\documentclass[12pt,a4paper]{article}
\usepackage[UTF8]{ctex}

\usepackage{cite}

\usepackage{geometry}
\usepackage{amsmath, amsthm, amssymb, bm, graphicx, mathrsfs}
\usepackage{graphicx}
\pagestyle{plain}

\usepackage{newtxtext}
\usepackage{mathptmx}
\DeclareMathAlphabet{\mathcal}{OMS}{cmsy}{m}{n}
\DeclareSymbolFont{largesymbols}{OMX}{cmex}{m}{n}

\usepackage[unicode, colorlinks]{hyperref}
\hypersetup{linkcolor = blue}

\geometry{right=2.5cm,left=2.5cm,top=2.5cm,bottom=2.5cm}
\usepackage{titlesec}

\newtheorem{theorem}{定理}[section]
\newtheorem{definition}[theorem]{定义} 
\newtheorem{proposition}[theorem]{命题}
\newtheorem{lemma}[theorem]{引理}
\newtheorem{example}[theorem]{例}
\newtheorem{corollary}[theorem]{推论}
\newtheorem{remark}[theorem]{注}

\newcommand{\e}{{\rm e}}
\newcommand{\ii}{{\rm i}}

\title{变分学总结}
\author{tron}
\date{\today}

\begin{document}

\maketitle
\tableofcontents

\section{经典理论}

简单地说, 变分学是研究泛函极值(以及更一般的临界值的一个数学分支). 设$\Omega \subseteq \mathbb{R}^n$是一个有界开集.
给定函数$L \in C^1(\overline{\Omega} \times \mathbb{R}^N \times \mathbb{R}^{nN})$ (通常称$L$为Lagrange函数, 即Lagrangian), 变分学主要研究如下形式的泛函:
\begin{equation}\label{1}
    I(u) = \int_{\Omega}L(x, u(x), \dot u(x)) \,{\rm d}x,
\end{equation}
其中$u \in M$, $M$是一个函数集合, 例如$C^1(\overline{\Omega})$等, 同时$M$由一些边值条件所规定. 有时$I$的积分中还可以含有高阶导数项,此时$M$也应做相应的改变.

变分问题的重要组成部分: \textbf{约束集合} + \textbf{目标泛函}.

\begin{example}[最速降线]
    在平面上给定两点$A = (x_1, y_1)$和$B = (x_2, y_2)$, 其中$x_1 < y_1, x_2 < y_2$.
    一个质点沿着一条连接这两点的``光滑''曲线仅受重力下滑. 设初速度为零, 问沿怎样的一条曲线滑行时间最短?
\end{example}

\begin{itemize}
    \item 约束集合. 为了问题的简单, 我们假设这条曲线有着显式表达式$y = u(x)$, 其中$u \in C^1[x_1, x_2]$.
    注意到这条曲线的起点和终点都是固定的, 由此我们设定约束函数集合为
    \begin{equation*}
        M = \{u \in C^1[x_1, x_2]\colon u(x_i) = y_i, i = 1, 2\}.
    \end{equation*}
    \item 目标泛函. 我们要在$M$中选取合适的函数$u^*$, 使得总时间$T = T(u)$达到极小.
    直接列出动力学方程:
    \begin{equation*}
        \begin{cases} 
            \displaystyle{\frac{1}{2}}mv^2 = mg(y_1 - u(x)), \\  
            v = \displaystyle{\frac{{\rm d}s}{{\rm d}t}}. 
        \end{cases}
    \end{equation*}
    联立得 
    \begin{equation*}
        T = T(u) = \int \,{\rm d}t = \frac{1}{\sqrt{2g}}\int_{x_0}^{x_1}\sqrt{\frac{1 + |\dot{u}(x)|^2}{2g(y_1 - u(x))}} \,{\rm d}x.
    \end{equation*}
\end{itemize}
至此, 我们便将最速降线问题转化成了一个变分学问题: 
\begin{align*}
    &\text{minimize}\ T(u) \\ 
    &\text{s.t.}\ u \in M.
\end{align*}

在以下几节中, 若无特别说明, 均假设$n = 1$.

\subsection{必要条件: Euler-Lagrange方程}

\textbf{函数极值的必要条件}. 设$\Omega$为$\mathbb{R}^N$中的开集, 函数$f \in C^1(\Omega)$在$x^* \in \Omega$处达到极小.
于是对任意的$h \in \mathbb{R}^n \smallsetminus \{0\}$, 存在充分小的$\varepsilon(h) > 0$, 使得当$0 < |\varepsilon| < \varepsilon(h)$时, $x^* + \varepsilon h \in \Omega$, 且$f(x^* + \varepsilon h) \geq f(x^*)$.
令$g_h(\varepsilon) = f(x^* + \varepsilon h)$, 由上述分析可知, $\varepsilon = 0$是单变量函数$g_h$的极小点, 故
\begin{equation*}
    0 = \dot g_h(0) = \nabla f(x^*) \cdot h.
\end{equation*}
再由$h$的任意性, 即得$\nabla f(x^*) = 0$.

\subsubsection{Euler-Lagrange方程}

对于泛函的情形我们也可以进行类似的处理. 首先做一些设定:

给定区间$J = [t_0, t_1] \subseteq \mathbb{R}$和开区域$\Omega \subseteq \mathbb{R}^N$, 取$L = L(x, u, p) \in C^1(J \times \Omega \times \mathbb{R}^N)$.
考虑约束集合 
\begin{equation*}
    M = \{u \in C^1(J)\colon u(t_i) = P_i, i = 1, 2\}
\end{equation*}
以及目标泛函 
\begin{equation*}
    I(u) =\int_JL(t, u(t), \dot{u}(t)) \,{\rm d}t.
\end{equation*}
称$u^*$是$I$的\textbf{极小点}, 若存在$u^*$的邻域$U$, 使得对任意的$u \in U \cap M$, 有$I(u) \geq I(u^*)$.
\begin{itemize}
    \item 函数的情形引入非零向量$h$ $\leadsto$ 泛函的情形引入测试函数$\varphi$.
    注意到对于函数自身正则性以及边值的约束, 一般取$\varphi \in C_0^1$.
    \item 邻域的刻画: 函数的情形引入充分小的$\varepsilon$ $\leadsto$ 泛函的情形是否能引入充分小的$\varepsilon = \varepsilon(\varphi)$来刻画?
    事实上, 注意到$u(J)$是紧的, 故$\varepsilon(\varphi)$的存在性可由$\mathbb{R}^n$的T4分离公理保证.
\end{itemize}

\begin{lemma}[du Bois-Reymond, 变分法基本引理]
    若$\psi \in C[t_0, t_1]$, 且
    \begin{equation*}
        \int_{t_0}^{t_1} \psi(t)\dot{\lambda}(t) \,{\rm d}t = 0, \quad \forall \lambda \in C_0^1[t_0, t_1],
    \end{equation*}
    则$\psi = {\rm const}$.
    \begin{proof}
        令
        \begin{equation*}
            c = \frac{1}{t_1 - t_0}\int_{t_0}^{t_1}\psi(t) \,{\rm d}t, \lambda(t) = \int_{t_0}^t(\psi(s) - c) \,{\rm d}s.
        \end{equation*}
        注意到此时$\lambda \in C_0^1[t_0, t_1]$, 故利用题设条件可得 
        \begin{equation*}
            \int_{t_0}^{t_1}(\psi(t) - c)^2 \,{\rm d}t = \int_{t_0}^{t_1}\dot{\lambda}(t)(\psi(t) - c) \,{\rm d}t = -c\int_{t_0}^{t_1}(\psi(t) - c) \,{\rm d}t = 0.
        \end{equation*}
        从而有$\psi = c$.
    \end{proof}
\end{lemma}

\begin{remark}
    上述du Bois-Reymond引理还有一些简单的推广形式. 如将$\psi \in C(J)$换成$\psi \in L^{\infty}(J)$或$\psi \in L^1(J)$, 相应地把$\lambda \in C_0^1(J)$换成$\lambda \in {\rm AC_0}(J)$或$\lambda \in C_c^{\infty}(J)$.
\end{remark}

\begin{proposition}
    设$u^* \in M$是泛函$I$在$M$上的一个极小点, 则它满足下列\textbf{积分形式的Euler-Lagrange方程}(简称\textbf{E-L方程}):
    \begin{equation*}
        \boxed{-\int_{t_0}^tL_{u_i}(s, u^*(s) , u^*(s)) \,{\rm d}s + L_{p_i}(t, u^*(t) , u^*(t)) = {\rm const}, \qquad \forall t \in J, 1 \leq i \leq n.}
    \end{equation*}
    \begin{proof}
        由前述分析可知, 对任意的$\varphi \in C_0^1[t_0, t_1]$, 存在充分小的$\varepsilon(\varphi) > 0$, 使得当$0 < |\varepsilon| < \varepsilon(\varphi)$时, 有$I(u^* + \varepsilon\varphi) \geq I(u^*)$.
        令$g_{\varphi}(\varepsilon) = I(u^* + \varepsilon\varphi)$. 题设条件表明$\dot g_{\varphi}(0) = 0$ (称$\dot g_{\varphi}(0)$为$I$对$\varphi$的\textbf{一阶变分}, 记作$\delta I(u^*, \varphi)$).
        利用分部积分公式, 进一步计算得
        \begin{align*}
            0 = g_{\varphi}'(0) &= \sum\limits_{i = 1}^N\int_J(L_{u_i}(t, u^*(t) , \dot u^*(t))\varphi_i(t) + L_{p_i}(t, u^*(t) , \dot u^*(t))\dot{\varphi_i}(t)) \,{\rm d}t \\  
            &= \sum\limits_{i = 1}^N\int_J\left(-\int_{t_0}^tL_{u_i}(s, u^*(s) , \dot u^*(s)) \,{\rm d}s + L_{p_i}(t, u^*(t) , \dot u^*(t))\right)\dot{\varphi_i}(t) \,{\rm d}t.
        \end{align*}
        由于上述结果对任意的$\varphi \in C_0^1[t_0, t_1]$均成立, 故利用du Bois-Reymond引理, 我们便得到了所证结论.
    \end{proof}
\end{proposition}

特别地, 若假设$L$和$u^*$具有更高的正则性, 如$L \in C^2$以及$u^* \in C^2$, 则$u^*$满足如下\textbf{微分形式的E-L方程}:
\begin{equation*}
    \boxed{L_{u_i}(t, u^*(t), \dot{u}^*(t)) - \frac{{\rm d}}{{\rm d}t}L_{p_i}(t, u^*(t), \dot{u}^*(t)) = 0.}
\end{equation*}
若$L \in C^1$且$u^* \in C^1$, 上述等式在广义导数意义下成立.

\begin{remark}
    事实上, E-L方程可适用于更广的函数类. 例如, 考虑Lipschitz函数类${\rm Lip}(J)$.
    注意到包含关系${\rm Lip}(J) \subseteq {\rm AC(J)}$, 故其导数几乎处处存在, 此时目标泛函中的积分按照Lebesgue积分意义理解.
    利用控制收敛定理和du Bois-Reymond引理, 我们仍可以导出积分形式的E-L方程. 
    
    特别地, 逐段$C^1$的函数是Lipschitz连续的, 因此积分形式的E-L方程对于逐段$C^1$的连续函数也成立.
\end{remark}

\begin{example}[质点运动方程]
    设$\mathbb{R}^3$中某质量为$m$的质点受外力$F$的作用, 其位置坐标为$x = (x_1, x_2, x_3)$, 则其动能$T = m|\dot x|^2/2$.
    若更设$F$有位势, 即存在函数$V$满足$-\nabla V = F$, 我们称
    \begin{equation*}
        L := T - V = \frac{1}{2}m|\dot x|^2 - V(x)
    \end{equation*}  
    为Lagrange函数\footnote{这是分析力学中的专有名词.}. 适当确定定义域$M$, 考虑泛函
    \begin{equation*}
        I(x) = \int_{t_1}^{t_2} L(x(t), \dot x(t)) \,{\rm d}t.
    \end{equation*}
    通过直接计算可知, $I$对应的E-L方程为$F = m\ddot x$, 即Newton第二定律所确定的运动轨道.
\end{example}

\subsubsection{变分导数}

从函数极值的角度来看, 若$x^*$是$f$的极小值点, 则我们只需要考虑$x^*$的一个小邻域内$f$的行为.
类似的, 尽管E-L方程的推导是在整个区间上进行的, 但对于任意的$\tau \in {\rm Int}\ J$, 在这一点的E-L方程只依赖于这点附近$L$的行为.
粗糙地说, 如果$u^*$在某点的某一邻域内达到``最优'', 那么$u^*$便满足整个区间上在这一点的E-L方程.

从以下极限过程看这种\textbf{局部性}: 引入\textbf{Euler-Lagrange算子}$E_L$:
\begin{equation*}
    \boxed{E_L(u)(t) := L_{u_i}(t, u^*(t), \dot{u}^*(t)) - \frac{{\rm d}}{{\rm d}t}L_{p_i}(t, u^*(t), \dot{u}^*(t)).}
\end{equation*}
以$N = 1$为例. 设$L \in C^2$, $u \in C^2$. 取$c \in (t_0, t_1)$, $\varphi \in C_0^1[t_0, t_1]$, 其中${\rm supp}\ \varphi \subseteq B_h(c)$.
直接计算得 
\begin{align*}
    \frac{I(u + \varphi) - I(u)}{\Delta\sigma} &=- \frac{\displaystyle\int_{c - h}^{c + h}\left(\int_{t_0}^tE_L(u + \theta\varphi)(s) \,{\rm d}s\right)\dot{\varphi}(t) \,{\rm d}t}{\Delta\sigma} \\ 
    &= \frac{\displaystyle\int_{c - h}^{c + h}E_L(u + \theta\varphi)(t)\varphi(t) \,{\rm d}t}{\Delta\sigma},
\end{align*}
其中$\theta \in (0, 1)$, $\Delta\sigma = \int_{B_h(c)}\varphi \,{\rm d}t$.
由简单的估计可知, 当$h \rightarrow 0$且$\sup_{\overline{B_h(c)}}|\dot\varphi| \rightarrow 0$时, 上式趋近于$E_L(u)$.
因此, 我们称E-L算子对函数$u$作用后在$t$点的值为$I$在$t$的\textbf{变分导数}.

\subsubsection{例}

如下考虑E-L方程的具体求解($N = 1$):

\begin{itemize}
    \item $L$不含$u$, 即$L = L(t, p)$. 此时E-L方程简化为 
    \begin{equation*}
        \frac{{\rm d}}{{\rm d}t}L_p(t, \dot{u}(t)) = 0.
    \end{equation*}
    若能从上述方程中解出$\dot u$, 那么也就可以通过积分得到$u$.
    \item $L$不含$p$, 即$L = L(t, u)$. 此时E-L方程化为函数方程$L_u(t, u) = 0$, 其解是一条或多条曲线.
    \item (自守系统) $L$不含$t$, 即$L = L(u, p)$. 在这种情况下, 引入\textbf{Hamilton量}$H(u, p) = pL_p(u, p) - L(u, p)$.
    通过直接计算, 我们有如下结果:
\end{itemize}

\begin{proposition}
    设$L \in C^2$且与$t$无关. 又设$u^* \in C^2$是对应E-L方程的解, 则
    \begin{equation*}
        H(u^*(t), \dot u^*(t)) \equiv {\rm const}, \qquad \forall\ t.
    \end{equation*}
\end{proposition}

利用上述命题, 我们可以在一些特殊情况下求解出$u^*$.

\begin{example}[最速降线-续]
    此时
    \begin{equation*}
        L(u, p) = \frac{1}{2g}\frac{\sqrt{1 + p^2}}{\sqrt{y_1 - u}}
    \end{equation*}
    与$t$无关. 利用上述命题, 我们有$(pL_p - L)|_{(u, \dot{u})} = {\rm const}$, 即存在常数$c$使得 
    \begin{equation*}
        -\frac{\sqrt{1 + \dot{u}^2}}{\sqrt{y_1 - u}} + \frac{\dot{u}^2}{\sqrt{(1 + \dot{u}^2)(y_1 - u)}} = c.
    \end{equation*}
    化简得 
    \begin{equation*}
        c^2(1 + \dot{u}^2)(y_1 - u) = 1.
    \end{equation*}
    令$k$为一待定常数, 引入参变量$\theta$, 令 
    \begin{equation*}
        \begin{cases} 
            x = x(\theta), \\  
            u = u(\theta) = y_1 - k(1 - \cos\theta), \end{cases}
    \end{equation*}
    则有 
    \begin{equation*}
        c^2\left(1 + k^2\frac{\sin^2\theta}{\dot{x}(\theta)}\right)k(1 - \cos\theta) = 1.
    \end{equation*}
    联想到半角公式, 故取$k = 1/2c^2$, 则可以解出 
    \begin{equation*}
        \dot{x}(\theta) = k(1 - \cos\theta).
    \end{equation*} 
    从而有
    \begin{equation*}
        \begin{cases} 
            x(\theta) = x_1 + k(\theta - \sin\theta), \\ 
            u(\theta) = y_1 - k(1 - \cos\theta), 
        \end{cases} \qquad \theta \in [0, \Theta],
    \end{equation*}
    其中$k$与$\Theta$通过
    \begin{equation*}
        \begin{cases} 
            x(\Theta) = x_2, \\  
            y(\Theta) = y_2 
        \end{cases}
    \end{equation*}
    确定.
\end{example}

\subsection{充分与必要条件: Legendre-Hadamard条件, Jacobi场}

一阶导/梯度 $\leadsto$ 一阶变分 $\Rightarrow$ E-L方程;
二阶导/Hessian阵 $\leadsto$ $\cdots$

设$\Omega$为$\mathbb{R}^N$中的开集, $f \in C^2(\Omega)$在$x_0 \in \Omega$处取得极小值.
由多元微积分的知识可知, 此时Hessian阵$(\partial_{ij}f)|_{x = x_0}$是半正定的; 反之, 若$(\partial_{ij}f)|_{x = x_0}$是正定的, 则$f$在$x_0$处取到极小值.

\subsubsection{二阶变分, Legendre-Hadamard条件}

设$L \in C^2(J \times \mathbb{R}^N \times \mathbb{R}^N)$, 
\begin{equation*}
    I(u) = \int_JL(t, u(t), \dot{u}(t)) \,{\rm d}t.
\end{equation*}
又设$u^* \in M$是对应E-L方程的解. 同1.1节的处理方法, 对任意的$\varphi \in C_0^1(J)$, 考虑单变量函数$g_{\varphi}(\varepsilon) = I(u^* + \varepsilon\varphi), 0 < |\varepsilon| < \varepsilon(\varphi)$.
则$g_{\varphi}$以$x = 0$为极小值点, 从而有$\ddot g_{\varphi}(0) \geq 0$. 直接计算得 
\begin{align*}
    \ddot{g}(0) &= \sum\limits_{i, j = 1}^N\int_J(L_{u_iu_j}(t, u^*(t), \dot{u}^*(t))\varphi_i(t)\varphi_j(t) + \\
    &L_{u_ip_j}(t, u^*(t), \dot{u}^*(t))\varphi_i(t)\dot{\varphi}_j(t) + L_{p_ip_j}(t, u^*(t), \dot{u}^*(t))\dot{\varphi}_i(t)\dot{\varphi}_j(t)) \,{\rm d}t.
\end{align*}
将上述等号右侧的表达式记为$\delta^2I(u^*, \varphi)$, 称为$I$在$u^*$处的\textbf{二阶变分}.
若引入函数矩阵
\begin{equation*}
    \begin{cases} 
        A_{u^*} = (L_{p_ip_j}(t, u^*(t), \dot{u}^*(t))), \\  
        B_{u^*} = (L_{p_iu_j}(t, u^*(t), \dot{u}^*(t))), \\  
        C_{u^*} = (L_{u_iu_j}(t, u^*(t), \dot{u}^*(t))).   
    \end{cases}
\end{equation*}
则二阶变分的表达式可以简化为 
\begin{equation*}
    \int_J(\dot\varphi^tA_{u^*}\dot{\varphi} + 2\dot{\varphi}^tB_{u^*}\varphi + \varphi^tC_{u^*}\varphi) \,{\rm d}t.
\end{equation*}

由上述分析, 我们有必要条件
\begin{equation*}
    u^*\text{极小} \Rightarrow \delta^2I(u^*, \varphi) \geq 0, \forall\varphi \in C_0^1(J).
\end{equation*}
上述必要条件可以进一步简化. 事实上, 三个矩阵$A_{u^*}, B_{u^*}, C_{u^*}$在判断$u^*$成为极小点中的地位是不平等的, 具体体现在当$\Vert \varphi \Vert_C$变化不大时, $\Vert \varphi \Vert_{C^1}$可以有很大的变化;
或者说, $\dot\varphi$比$\varphi$的影响更大.

\begin{proposition}[必要条件1]
    设$L \in C^2(J \times \mathbb{R}^N \times \mathbb{R}^N)$. 若$u^* \in M$是$I$的极小点, 则$A_{u^*}$是半正定的, 即
    \begin{equation}\label{2}
        \boxed{\xi^tA_{u^*}\xi = \sum\limits_{i, j = 1}^NL_{p_i, p_j}(\tau, u^*(\tau), \dot{u}^*(\tau))\xi_i\xi_j \geq 0, \qquad \forall \tau \in J, \xi \in \mathbb{R}^N.}
    \end{equation}
    称\eqref{2}为\textbf{Legendre-Hadamard条件}.
    \begin{proof}
        取$v \in C^1(\mathbb{R})\colon v|_{|s| \geq 1} = 0, \Vert \dot v \Vert_{L^2(\mathbb{R})}^2 = 1$.
        对任意的$\tau \in {\rm Int}(J), \xi \in \mathbb{R}^N$以及充分小的$\mu > 0$, 令 
        \begin{equation*}
            \varphi(t) = \xi\mu v\left(\frac{t - \tau}{\mu}\right).
        \end{equation*}
        将上述构造的$\varphi$代入至二阶变分的表达式中, 即得 
        \begin{equation*}
            0 \leq \delta^2I(u^*, \varphi) = \mu(\xi^tA_{u^*}\xi) + o(\mu) \qquad (\mu \rightarrow 0^+). 
        \end{equation*}
        上式即表明$A_{u^*}$是半正定的.
    \end{proof}
\end{proposition}

\subsubsection{Poincaré不等式}

\begin{proposition}[充分条件1]\label{prop1.10}
    若$u^* \in C_0^1(J)$满足E-L方程, 且存在$\lambda > 0$使得
    \begin{equation*}
        \delta^2I(u^*, \varphi) \geq \lambda\int_J(|\varphi|^2+|\dot\varphi|^2) \,{\rm d}t, \forall \varphi \in C_0^1(J),
    \end{equation*}
    则$u^*$是$I$的一个严格极小点.
    \begin{proof}
        对任意的$\varphi \in C_0^1(J)$, 令$g_{\varphi}(\varepsilon) = I(u^* + \varepsilon\varphi), 0 < |\varepsilon| < \varepsilon(\varphi)$.
        注意到$u^*$满足E-L方程, 故有 
        \begin{equation*}
            g(\varepsilon) - g(0) = g(\varepsilon) - g(0) - s\dot g(0) = \frac{s^2}{2}\ddot g(\theta \varepsilon) = \frac{s^2}{2}(\ddot g(\theta \varepsilon) - \ddot g(0)) + \frac{s^2}{2}\ddot g(0), 
        \end{equation*}
        其中$\theta = \theta(\varphi) \in (0, 1)$. 再注意到对于$\Vert \varphi \Vert_{C^1} \leq 1$, 当$s \rightarrow 0$时有 
        \begin{equation*}
            |A_{u^* + \varepsilon\varphi} - A_{u^*}| + |B_{u^* + \varepsilon\varphi} - B_{u^*}| + |C_{u^* + \varepsilon\varphi} - C_{u^*}| = o(1).
        \end{equation*}
        因此当$|\varepsilon| > 0$充分小时, 存在$\eta < \lambda$使得 
        \begin{equation*}
            \ddot g(\theta \varepsilon) - \ddot g(0) \geq -\eta\int_J(|\varphi|^2 + |\dot\varphi|^2) \,{\rm d}t.
        \end{equation*}
        结合题设条件, 我们便有
        \begin{equation*}
            I(u^* + \varepsilon\varphi) - I(u^*) = g(\varepsilon) - g(0) \geq \frac{\varepsilon^2}{2}(\lambda - \eta)\int_J(|\varphi|^2 + |\dot\varphi|^2) \,{\rm d}t > 0,
        \end{equation*} 
        从而$u^*$是$I$的严格极小点.
    \end{proof}
\end{proposition}

\begin{remark}
    由上述命题, 如果矩阵
    \begin{equation*}
        \begin{pmatrix} 
            A_{u^*} & B_{u^*} \\ 
            B_{u^*} & C_{u^*} 
        \end{pmatrix} 
    \end{equation*} 
    是正定的, 那么E-L方程的解$u^*$必是极小点.
\end{remark}

命题\ref{prop1.10}给出的充分条件还可以被进一步地简化.

\begin{lemma}[Poincaré不等式]\label{lma1.12}
    设$\varphi \in C_0^1[t_0, t_1]$, 则 
    \begin{equation}\label{3}
        \int_J|\varphi|^2 \,{\rm d}t \leq \frac{(t_1 - t_0)^2}{2}\int_J|\dot\varphi|^2 \,{\rm d}t.
    \end{equation}
    \begin{proof}
        注意到
        \begin{equation*}
            \varphi(t) = \int_{t_0}^t\dot\varphi(s) \,{\rm d}s,
        \end{equation*}
        故由Cauchy-Schwarz不等式可知, 
        \begin{equation*}
            |\varphi(t)|^2 \leq \left(\int_{t_0}^t|\dot\varphi(s)|^2 \,{\rm d}s\right)^2 \leq (t - t_0)\int_J|\dot\varphi(s)|^2 \,{\rm d}s.
        \end{equation*}
        积分得 
        \begin{equation*}
            \int_J|\varphi(t)|^2 \,{\rm d}t \leq \frac{(t_1 - t_0)^2}{2}\int_J|\dot\varphi|^2 \,{\rm d}t.
        \end{equation*}
    \end{proof}
\end{lemma}

\begin{remark}
    若$\varphi \in {\rm AC(J)}$, $\dot\varphi \in L^2(J)$且$\varphi(a) = 0$, 则Poincaré不等式\eqref{3}仍成立.
\end{remark}

由Poincaré不等式可知, 若存在$\lambda > 0$使得
\begin{equation*}
    \delta^2I(u^*, \varphi) \geq \lambda\int_J|\dot\varphi|^2 \,{\rm d}t, \qquad \forall \varphi \in C_0^1(J),
\end{equation*}
那么命题\ref{prop1.10}成立.

\begin{proposition}[充分条件2]\label{prop1.14}
    给定一个足够光滑的$L$, 设沿其E-L方程的解$u^*$满足\textbf{严格的Legendre-Hadamard条件}, 即$A_{u^*}$是正定的.
    若存在$\mu > 0$使得 
    \begin{equation*}
        \delta^2I(u^*, \varphi) \geq \mu\int_J|\varphi|^2 \,{\rm d}t, \qquad \forall \varphi \in C_0^1(J),
    \end{equation*}
    则存在$\lambda > 0$使得
    \begin{equation*}
        \delta^2I(u^*, \varphi) \geq \lambda\int_J(|\varphi|^2+|\dot\varphi|^2) \,{\rm d}t, \qquad \forall \varphi \in C_0^1(J).
    \end{equation*}
    从而$u^*$是$I$的一个严格极小值点.
    \begin{proof}
        由于$A_{u^*}$是正定的, 故存在$\alpha > 0$使得 
        \begin{equation*}
            \dot\varphi^tA_{u^*}\dot\varphi \geq \alpha|\dot\varphi|^2, \qquad \forall \varphi \in C_0^1(J).
        \end{equation*}
        从而存在正常数$C_1, C_2$使得
        \begin{align*} 
            \alpha\int_J|\dot\varphi|^2 \,{\rm d}t &\leq \delta^2I(u^*,\varphi) + \int_J(2| \dot\varphi^tB_{u^*}\varphi| + |\varphi^t C_{u^*}\varphi|) \,{\rm d}t \\ 
            &\leq \delta^2I(u^*,\varphi) + C_1\left(\left(\int_J|\dot\varphi|^2 \,{\rm d}t\right)^{1/2}\left(\int_J|\varphi|^2 \,{\rm d}t\right)^{1/2} + \int_J|\varphi|^2 \,{\rm d}t\right) \\ 
            &\leq \delta^2I(u^*,\varphi) + \frac{\alpha}{2}\int_J|\dot\varphi|^2 \,{\rm d}t + C_2\int_J|\varphi|^2 \,{\rm d}t, 
        \end{align*}
        这里我们用到了加权的初等不等式$\sqrt{ab} \leq (ka + b/k)/2, a, b > 0, k > 0$.
        由此即得 
        \begin{equation*}
            \int_J|\dot\varphi|^2 \,{\rm d}t \leq \frac{2}{\alpha}(1+C_2\mu^{-1})\delta^2I(u^*, \varphi), \qquad \forall \varphi \in C_0^1(J).
        \end{equation*} 
        这表明$u^*$是一个极小点.
    \end{proof}
\end{proposition}

\subsubsection{Jacobi场, 共轭点}

1.2.2节中列出的充分条件中仍含有任意函数$\varphi$, 还需要寻找更为精确的充分条件.
以下我们建立充分条件与严格Legendre-Hadamard条件之间的联系.

注意到二阶变分的具体表达式. 设$L \in C^3, u^*$是E-L方程的解. 令 
\begin{equation*}
    \boxed{\Phi_{u^*}(t, \xi, \eta) := \eta^tA_{u^*}\eta+2\xi^tB_{u^*}\eta+\xi^tC_{u^*}\xi, \qquad \forall(\xi, \eta) \in \mathbb{R}^N \times \mathbb{R}^N.}
\end{equation*}
我们称其为\textbf{附属的(accessory)Lagrange函数}. 此时$\delta^2I(u^*, \varphi)$可以看作是附属Lagrange函数相关的变分积分:
\begin{equation*}
    \delta^2I(u^*, \varphi) = Q_{u^*}(\varphi)=\int_J\Phi_{u^*}(t, \varphi(t), \dot\varphi(t)) \,{\rm d}t.
\end{equation*}
若设$u^*$是一个极小点, 则有$Q_{u^*}(\varphi) = \delta^2I(u^*, \varphi) \geq 0, \forall\varphi \in C_0^1(J)$.
注意到$Q_{u^*}(0) = 0$, 故$0$是$Q_{u^*}$的极小点. 更一般地, 我们将$Q_{u^*}$的定义域扩充到${\rm Lip}_0(J)$上, 导出它的积分形式的E-L方程:
\begin{equation*}
    A_{u^*}\dot\varphi(t) + B_{u^*}\varphi(t) - \int_{t_0}^t(B_{u^*}\dot\varphi(t) + C_{u^*}\varphi(t)) \,{\rm d}t = {\rm const}. 
\end{equation*}
事实上, 若$L$沿$u^*$满足满足严格的Legendre-Hadamard条件, 那么上述方程的解$\varphi \in C^2(J)$.
因此$\varphi$满足微分形式的E-L方程:
\begin{equation*}
    \boxed{J_{u^*}(\varphi)= \frac{{\rm d}}{{\rm d}t}(A_{u^*}\dot\varphi(t) + B_{u^*}\varphi(t)) - (B_{u^*}\dot\varphi(t) + C_{u^*}\varphi(t)) = 0, \quad t \in J.}
\end{equation*}
称此方程为\textbf{Jacobi方程}, 算子$J_{u^*}$为沿E-L方程的解$u^*$的\textbf{Jacobi算子}, Jacobi算子的任意一个$C^2$解为沿轨道$u^*$的一个\textbf{Jacobi场}.

\begin{proposition}[Jacobi场的刻画]\label{prop1.15}
    设$\varphi^*$是沿$u^*$的一个Jacobi场, 则$Q_{u^*}(\varphi^*) = 0$;
    反之, 若$\varphi^* \in {\rm Lip}_0(J)$满足$Q_{u^*}(\varphi^*) = 0$, 而且$Q_{u^*}(\varphi^*) \geq 0, \forall \varphi \in C_0^1(J)$, 则$\varphi^*$是沿$u^*$的一个Jacobi场.
    \begin{proof}
        若$\varphi^*$是沿$u^*$的一个Jacobi场, 通过直接计算可得
        \begin{align} 
            Q_{u^*}(\varphi^*) &= \int_J((\dot\varphi^*)^t(A_{u^*}\dot{\varphi}^* +B_{u^*}\varphi^*) + (\varphi^*)^t(B_{u^*}\dot\varphi^* + C_{u^*}\varphi^*)) \,{\rm d}t \\ 
            &= \int_J(\varphi^*)^t\left(-\frac{{\rm d}}{{\rm d}t}(A_{u^*}\dot\varphi^* + B_{u^*}\varphi^*) + (B_{u^*}\dot\varphi^* + C_{u^*}\varphi^*)\right) \,{\rm d}t \\  
            &= -\int_J (\varphi^*)^t J_{u^*}(\varphi^*) \,{\rm d}t = 0. 
        \end{align} 
        反之, 利用光滑函数逼近,我们有
        \begin{equation*}
            Q_{u^*}(\varphi) = 0, \qquad \forall \varphi \in {\rm Lip}_0(J). 
        \end{equation*}
        于是$\varphi^*$是$u^*$的一个极小点.由前述分析可知, $\varphi^*$也满足微分形式的E-L方程, 即$J_{u^*}(\varphi^*) = 0$. 
    \end{proof}
\end{proposition}

\begin{definition}
    设$u^*$是$I$的E-L方程的一个解. 称$(a, u^*(a))$与$(b, u^*(b))$是轨道$(t, u^*(t))$上的一对\textbf{共轭点}, 如果存在一个沿$u^*(t)$的非零Jacobi场$\varphi \in C_0^1[a, b]$.
    若在轨道$\{(t, u^*(t))\colon t \in (t_0, t_1]\}$上$(t_0, u^*(t_0))$没有共轭点,则称$u^*$没有共轭点.
\end{definition}

\begin{proposition}[必要条件2]
    设$u^*$是$I$的E-L方程的一个解, 且$L$沿$u^*$满足严格的Legendre-Hadamard条件.
    若$Q_{u^*}(\varphi) \geq 0, \forall \varphi \in C_0^1(J)$, 则不存在$a \in {\rm Int}(J)$使得$(a, u^*(a))$共轭于$(t_0, u^*(t_0))$.
    \begin{proof}
        若不然, 设存在$a \in {\rm Int}(J)$使得$(a, u^*(a))$是$(t_0, u^*(t_0))$的共轭点, 即存在$u^*$的非零Jacobi场$\xi \in C^2[t_0, a]$, 并满足$\xi(a) = \xi(t_0) = 0$.
        现令 
        \begin{equation*}
            \widetilde{\xi}(t) =  
            \begin{cases} 
                \xi(t) \quad &t \in  [t_0, a], \\  
                0 \quad &t \in (a, t_1], 
            \end{cases} 
        \end{equation*}
        则$\widetilde{\xi} \in {\rm Lip}(J)$, 且满足$\widetilde{\xi}(t_0) = \widetilde{\xi}(t_1) = 0$, 而且 
        \begin{equation*}
            Q_{u^*}(\widetilde{\xi}) = \int_{t_0}^a\Phi_{u^*}(t, \xi(t), \dot\xi(t)) \,{\rm d}t = 0.
        \end{equation*}
        由命题\ref{prop1.15}可知, $\widetilde{\xi} \in C^2(J)$且满足Jacobi方程, 即$J_{u^*}(\widetilde{\xi}) = 0$.
        由常微分方程初值问题的唯一性可知, $\widetilde{\xi} = 0$, 矛盾.
    \end{proof}  
\end{proposition}

当$N = 1$时, 上述命题的逆也成立.

\begin{proposition}[充分条件3, $N = 1$]
    设$u^*$是E-L方程的一个解, $L$沿$u^*$满足严格的Legendre-Hadamard条件.
    若在$J$上存在一个正的Jacobi场$\psi$, 则$u^*$是一个严格极小点。特别地,若$u^*$没有共轭点, 则在$J$上存在一个正的Jacobi场.
    \begin{proof}
        证明分三步进行.

        \emph{Step 1.} 若$u^*$没有共轭点, 那么便存在一个正的Jacobi场. 事实上, 设$\lambda$是一个Jacobi场, 其中$\lambda(t_0) = 0, \dot\lambda(t_0) = 1$.
        由假设条件可知, $\lambda(t) > 0, t \in (t_0, t_1]$. 再根据微分方程对初值的连续依赖性, 故存在一个Jacobi场$\psi$使得$\psi(t) > 0, t \in J$.

        \emph{Step 2.} 对任意的$\varphi \in C_0^1(J)$, 有 
        \begin{equation*}
            Q_{u^*}(\varphi)=\int_JA_{u^*}\psi^2\dot{\left(\frac{\varphi}{\psi}\right)}^2 \,{\rm d}t. 
        \end{equation*}
        记$\alpha = \varphi/\psi$, 则$\varphi = \alpha\psi, \dot\varphi = \dot\alpha\psi + \alpha\dot\psi$.
        从而有 
        \begin{equation*}
            A_{u^*}\dot\varphi^2+2B_{u^*}\dot\varphi\varphi+C_{u^*}\varphi^2=\alpha^2(A_{u^*}\dot\psi^2+2B_{u^*}\dot\psi\psi+C_{u^*}\psi^2)+2\dot\alpha\alpha(A_{u^*}\dot\psi+B_{u^*}\psi)+\dot\alpha^2A_{u^*}\psi^2. 
        \end{equation*}
        注意到$\psi$满足Jacobi方程, 故 
        \begin{align*}
            Q_{u^*}(\varphi) &= \int_J\left(\frac{{\rm d}(\psi\lambda^2)}{{\rm d}t}(A_{u^*}\dot\psi+B_{u^*}\psi) + \psi\lambda^2\frac{{\rm d}(A_{u^*}\dot\psi+B_{u^*}\psi) }{{\rm d}t} + A_{u^*}\dot\lambda^2\psi^2\right) \,{\rm d}t \\  
            &= \left.(\psi\lambda^2(A_{u^*}\dot\psi+B_{u^*}\psi))\right|_{t_0}^{t_1} + \int_JA_{u^*}\dot\lambda^2\psi^2\,{\rm d}t \\  
            &= \int_JA_{u^*}\dot\lambda^2\psi^2\,{\rm d}t.
        \end{align*}

        \emph{Step 3.} 记$\beta = \inf_J (A_{u^*}\psi^2) > 0$. 对任意的$\varphi \in C_0^1(J)$, 由Poincaré不等式可得 
        \begin{align*}
            Q_{u^*}(\varphi) = \int_JA_{u^*}\psi^2\dot{\left(\frac{\varphi}{\psi}\right)}^2\,{\rm d}t  \geq \beta\int_J\dot{\left(\frac{\varphi}{\psi}\right)}^2\,{\rm d}t &\geq \frac{2\beta}{|J|^2}\int_J\left(\frac{\varphi}{\psi}\right)^2 \,{\rm d}t 
            \\ &\geq \frac{2\beta}{|J|^2}\inf_J\left(\frac{1}{\psi^2}\right)\int_J\varphi^2 \,{\rm d}t.
        \end{align*}
        结合命题\ref{prop1.14}的结论, 我们便证得了所需结论.
    \end{proof}
\end{proposition}

\begin{example}
    设$M = \{u \in C^1[0, 1]\colon u(0) = 1, u(1) = b\}$. 考虑以下泛函 
    \begin{equation*}
        I(u) = \int_0^1(t\dot u + \dot u^2) \,{\rm d}t.
    \end{equation*}
    \begin{proof}[解]\let\qed\relax
        直接计算得$L_u = 0, L_p = 2p + t$, 故其E-L方程$2\ddot u + 1 = 0$有满足初值条件的解
        \begin{equation*}
            u^*(t) = -\frac{t^2}{4} + \left(b - a + \frac{1}{4}\right)t + a.
        \end{equation*}
        由于$L_{pp} = 2 > 0$, 并且对应的Jacobi方程$\ddot\varphi = 0$在$[0, 1]$上有一个正解, 所以$u^*$是一个严格极小点.
    \end{proof}
\end{example}

\subsection{强极小与极值场}

$C^1$拓扑 $\leadsto$ 考虑的因素更多, ``弱极小点''. $C = C^0$拓扑 $\leadsto$ 考虑的因素更少, ``强极小点''.

\begin{definition}
    设$J = [t_0, t_1], L \in C^1(J \times \mathbb{R}^N \times \mathbb{R}^N)$. 称$u \in C^1(J)$为 
    \begin{equation*}
        I = \int_JL(t, u(t), \dot u(t)) \,{\rm d}t
    \end{equation*}
    的\textbf{强(弱)极小点}, 如果存在$\varepsilon > 0$使得对一切满足
    \begin{equation*}
        \Vert \varphi \Vert_C < \varepsilon \qquad (\Vert \varphi \Vert_{C^1} < \varepsilon) 
    \end{equation*}
    的$\varphi \in C_0^1(J)$都有$I(u + \varphi) \geq I(u)$. 函数类$C^1$可以换成${\rm Lip}$, 并且用${\rm Lip}$模代替$C^1$模.
\end{definition}

前两节所讨论的极小点是弱极小点. ${\rm Lip}$意义下的弱极小点也是$C^1$意义下的弱极小点; 强极小点是弱极小点, 但反之不然.

\begin{example}
    设$M = {\rm Lip}_0[0, 1]$, 
    \begin{equation*}
        I(u)=\int_0^1(\dot u^2+\dot u^3) \,{\rm d}t.
    \end{equation*}
    注意到$I(0) = 0$, 且当$\Vert u \Vert_{{\rm Lip}} < 1/2$时, 有 
    \begin{equation*}
        I(u) - I(0) = \int_0^1\dot u^2(1 + \dot u) \,{\rm d}t \geq \frac{1}{2}\int_0^1\dot u^2 \,{\rm d}t \geq 0. 
    \end{equation*}
    故$u = 0$是弱极小点, 但不是强极小点. 事实上, 对充分小的$h > 0$, 令 
    \begin{equation*}
        u_h(x) =  
        \begin{cases} 
            \displaystyle-\frac{x}{h} \quad &x \in [0, h^2], \\  
            \displaystyle\frac{h(x - 1)}{1 - h^2} \quad &x \in [h^2, 1]. 
        \end{cases}
    \end{equation*}
    一方面, $\Vert u_h \Vert_C \leq h$; 另一方面, 直接计算得 
    \begin{equation} 
        (\dot u_h^2 + \dot u_h^3)(x) =  
        \begin{cases} 
            \displaystyle \frac{1}{h^2} - \frac{1}{h^3} \leq -\frac{1}{2h^3} \quad &x \in [0, h^2), \\  
            \displaystyle \left(\frac{h}{1 - h^2}\right)^2 + \left(\frac{h}{1 - h^2}\right)^3 \leq 2 \quad &x \in (h^2, 1]. 
        \end{cases} 
    \end{equation}
    因此
    \begin{equation*}
        I(u_h) - I(0) \leq 2 - \frac{1}{2h} \rightarrow -\infty \quad (h \rightarrow 0^+).
    \end{equation*}
    从而$u = 0$不是强极小点.
\end{example}

\subsubsection{必要条件: Weierstrass过度函数}

设$u^* \in C^1(J)$是E-L方程的解. 以下探究$u^*$成为强极小点的必要条件.
为此我们把$I(u^*)$与$I$在$u^*$的$C$拓扑临近的函数上的值作比较, 即构造适当的$\varphi \in C_0^1(J)$.

\begin{proposition}[必要条件]
    若$u^* \in C^1(J)$是$I$的一个强极小点, 则
    \begin{equation*}
        \mathfrak{E}_L(t, u(t), \dot u(t), \dot u(t) + \xi) \geq 0, \qquad \forall \xi \in \mathbb{R}^N, \tau \in J.
    \end{equation*}
    这里
    \begin{equation*}
        \boxed{\mathfrak{E}_L = \mathfrak{E}_L(t, u, p, q) := L(t, u, p) - L(t, u, q) - (q - p) \cdot L_p(t, u, p),}
    \end{equation*}
    并称之为\textbf{Weierstrass过度函数}.
    \begin{proof}
        对任意的$\xi \in \mathbb{R}^N, \tau \in {\rm Int}(J)$, 当$\lambda > 0$充分小时, 可使得$[\tau - \lambda^2, \tau+\lambda] \subseteq (t_0, t_1)$.
        令
        \begin{equation} 
            \psi_{\lambda}(x) =  
            \begin{cases} 
                0 \quad &s\in (\infty, -\lambda^2] \cup [\lambda, \infty), \\  
                s + \lambda^2 \quad &s \in [-\lambda^2, 0], \\  
                -\lambda s + \lambda^2 \quad &s \in [0, \lambda] 
        \end{cases} 
    \end{equation}
    和$\varphi_{\lambda}(t) = \xi\psi_{\lambda}(t - \tau)$. 注意到$\Vert \varphi_{\lambda}\Vert = O(\lambda^2)$. 若$u^*$是强极小点, 当$\lambda > 0$充分小时有$I(u + \varphi_{\lambda}) - I(u) \geq 0$.
    \begin{align*} 
        0 &\leq \int_J(L(t, u(t) + \varphi_{\lambda}(t), \dot u(t) + \dot\varphi_{\lambda}(t)) - L(t, u(t), \dot u(t))) \,{\rm d}t \\  
        &= \int_J(L(t, u(t) + \varphi_{\lambda}(t), \dot u(t) + \dot\varphi_{\lambda}(t)) \\
        - &L(t, u(t), \dot u(t)) - \varphi_{\lambda} (t) \cdot L_u(t, u(t), \dot u(t)) - \dot\varphi_{\lambda}(t) \cdot L_p(t, u(t), \dot u(t))) \,{\rm d}t \\ 
        &= \int_J F(t) \,{\rm d}t. 
    \end{align*}
    注意到$F$的具体表达式, 我们将上述积分拆成两个部分:
    \begin{equation*}
        \int_JF(t) \,{\rm d}t = \left(\int_{\tau - \lambda^2}^{\tau} + \int_{\tau}^{\tau + \lambda}\right)F(t) \,{\rm d}t.
    \end{equation*}
    一方面, 当$t \in [\tau, \tau + \lambda]$时, 注意到$\Vert \varphi_{\lambda}|_{[\tau, \tau + \lambda]}\Vert_C = O(\lambda)$以及$\Vert \dot\varphi_{\lambda}|_{[\tau, \tau + \lambda]}\Vert_C = O(\lambda), \lambda \rightarrow 0$.
    故由Taylor展开即得$F|_{[\tau, \tau + \lambda]} = O(\lambda^2) = o(1)$, 从而有 
    \begin{equation*}
        \lim\limits_{\lambda \rightarrow 0}\frac{1}{\lambda^2}\int_{\tau}^{\tau + \lambda} F(t) \,{\rm d}t = 0.
    \end{equation*}
    另一方面, 我们有 
    \begin{equation*}
        \lim\limits_{\lambda \rightarrow 0}\frac{1}{\lambda^2}\int_{\tau - \lambda^2}^{\tau} F(t) \,{\rm d}t = L(\tau, u(\tau), \dot u(\tau) + \xi) - L(\tau, u(\tau), \dot u(\tau)) - \xi\cdot L_p(\tau, u(\tau), \dot u(\tau)),
    \end{equation*}
    因此
    \begin{equation*}
        L(\tau, u(\tau), \dot u(\tau) + \xi) - L(\tau, u(\tau), \dot u(\tau)) - \xi \cdot L_p(\tau, u(\tau), \dot u(\tau)) \geq 0. 
    \end{equation*}
    上述不等式左侧的表达式等于$\mathfrak{E}_L(t, u(t), \dot u(t), \dot u(t) + \xi)$. 
    \end{proof}
\end{proposition}

\subsubsection{充分条件: 极值场}

基本思想: 对于一个给定的$C^1$函数$u$, 我们将其``嵌入''至一组``极值曲线''中, 这组``极值曲线''具有``统一的方向''.
注意到对所有的$\varphi \in C_0^1(J)$, $u$与$u + \varphi$具有相同的起点和终点, 如果能证明积分与道路的无关性, 我们可以将差值$I(u + \varphi) - I(u)$的表达式进一步地简化, 从而得到较为简洁的充分条件.

设$u^*$是E-L方程的解. 称$u^*$对应的图像$\{(t, u^*(t)) \colon t \in J\}$为一条\textbf{极值曲线}.
现设$u^*$可以延拓到更大的区间$J_1 = (a, b) \supseteq J$上, 又设$\{(t, u(t, \alpha))\colon t \in J_1, \alpha \in B_{\varepsilon_1}(0), \varepsilon_1 > 0\}$是$I$的一族足够光滑的极值曲线.

\begin{definition}
    设$\Omega$是$\{(t, u(t, \alpha))\colon t \in J_1, \alpha \in B_{\varepsilon}(0)\} \ (0 < \varepsilon < \varepsilon_1)$的一个单连通的开邻域, $\psi = \psi(t, u) \in C^1(\Omega)$是向量场.
    如果 
    \begin{itemize}
        \item 对任意的$u = u(t, \alpha)$, $u$满足方程$\partial_tu = \psi(t, u)$;
        \item $\det (\partial_{\alpha_i}u_j(t, \alpha)) \neq 0$;
        \item 对任意的$(t_1, u_1) \in \Omega$, 存在唯一的$\alpha_1 \in B_{\varepsilon_1}(0)$使得$u(t_1, \alpha_1) = u_1$;
        \item $u(t, 0) = u^*(t)$,
    \end{itemize}
    那么称$\Omega$为一个\textbf{极值场(或临界场)}, 并称$\psi$为其\textbf{方向场(或流)}.
\end{definition}

\begin{example}
    设$L_1 = p^2/2$, 则
    \begin{equation*}
        \Omega_1 = \{(t, mt + \lambda)\colon (t, \lambda) \in \mathbb{R} \times \mathbb{R}\}, \psi_1(t, u) = m
    \end{equation*}
    分别是$L_1$的一个极值场和方向场. 再考虑$L_2 = (p^2 - u^2)/2$. 对任意的开区间$O$, 令 
    \begin{equation*}
        \Omega_2 = \{(t, \sin(t + \lambda))\colon (t, \lambda) \in O \times (-1, 1)\}.
    \end{equation*}
    虽然极值曲线充满了整个$\Omega_2$, 但$\Omega_2$中每一点都有两个极值曲线通过, 所以$\Omega_2$不是极值场.
\end{example}

\textbf{Analysis:} 设极值曲线$\gamma^* = \{(t, u^*(t))\colon t \in J\} \subseteq \Omega$满足方程$\dot u = \psi(t, u)$, 其中$\psi$是极值场$\Omega$上的一个方向场.
我们选取与$\gamma^*$邻近的, 端点相同的$C^1$曲线$\gamma = \{(t, u(t))\colon t \in J\}$作比较.
将在区间上的积分转化为转化为路径积分, 如果积分与路径无关, 那么就有
\begin{align*} 
    I(u^*) &= \int_JL(t, u^*, \dot u^*) \,{\rm d}t \\ 
    &= \int_{\gamma^*}(L(t, u^*, \psi(t, u^*)) - \psi(t, u^*)\cdot L_p(t, u^*, \psi(t, u^*)) \,{\rm d}t + L_p(t, u^*, \psi(t, u^*))) \,{\rm d}u \\  
    &= \int_{\gamma}(L(t, u, \psi(t, u)) - \psi(t, u)\cdot L_p(t, u, \psi(t, u)) \,{\rm d}t + L_p(t, u, \psi(t, u))) \,{\rm d}u \\  
    &= \int_J(L(t, u, \psi(t, u)) + (\dot u - \psi(t, u))\cdot L_p(t, u, \psi(t, u))) \,{\rm d}t, 
\end{align*}
于是 
\begin{align*}
    I(u) - I(u^*) &= \int_J(L(t, u, \dot u) - L(t, u, \psi(t, u)) - (\dot u - \psi(t, u))\cdot L_p(t, u, \psi(t, u))) \,{\rm d}t \\  
    &= \int_J\mathfrak{E}_L(t, u, \psi(t, u), \dot u) \,{\rm d}t.
\end{align*} 
因此, 若对任意的$(t, u, p) \in \Omega \times  \mathbb{R}^N$有$\mathfrak{E}_L(t, u, \psi(t, u), p) \geq 0$, 那么$u^*$是一个强极小点.

在上述分析中用到了两个事实:

\begin{enumerate}
    \item \textbf{$u^*$对应的极值曲线$\gamma^*$可以嵌入到一个极值场中}. 具体地, 所谓``嵌入''是指, 存在开区间$J_1 \supseteq J$以及$u = u(t, \alpha) \in C^1(J_1 \times B_{\varepsilon})$, 使得对任意的$\alpha \in B_{\varepsilon}, u(t, \alpha)$是一条极值曲线, 其中$u^*(t) = u(t, 0)|_J$,
    而且$\{(t, u(t, \alpha))\colon t \in J_1, \alpha \in B_{\varepsilon}(0)\}$是一个极值场.
    \item \textbf{积分与路径无关}.
\end{enumerate}

以下我们验证这两个事实, 首先是第一个.

\begin{proposition}\label{prop1.25}
    设$L \in C^3(J \times \mathbb{R}^N \times \mathbb{R}^N), u^* \in C^2(J)$是其E-L方程的一个解. 又设$L$沿$u^*$满足严格Legendre-Hadamard条件.
    如果沿对应于$u^*$的极值曲线$\gamma^*$没有共轭点, 那么$\gamma^*$可以嵌入到一族极值曲线中, 并且由这族曲线确定的单连通区域$\Omega$是一个极值场.
    \begin{proof}
        先验证$N = 1$的情形. 首先, 由题设条件可知, $u^*$可以延拓到更大的区间$J_1 = (a, b) \supseteq J$上.
        相对于$\alpha \in \mathbb{R}$, 当$|\alpha| < \varepsilon_0$充分小时, 考虑初值问题 
        \begin{equation*}
            \begin{cases} 
                E_L(\varphi(\cdot, \alpha)) = 0, \\  
                \varphi(a, \alpha) = u^*(a), \\  
                \varphi_t(a, \alpha) = \dot u^*(a) + \alpha. 
            \end{cases}
        \end{equation*}
        由此我们得到一族解$\{\varphi(t, \alpha)\}$, 其中$t \in J_1, |\alpha| < \varepsilon_0$.
        再根据初值问题的唯一性, $\varphi(t, 0) = u^*(t)$.

        现定义
        \begin{equation*}
            \Omega_{\varepsilon} = \{(t, \varphi(t, \alpha))\colon t \in J_1, |\alpha| < \varepsilon\}, 
        \end{equation*}
        其中$\varepsilon < \varepsilon_1$. 通过直接计算可知, 
        \begin{equation*}
            \xi(t) = \left.\partial_{\alpha}\varphi(t, \alpha)\right|_{\alpha = 0}
        \end{equation*}
        是一个沿$u^*$的Jacobi场, 同时$\xi(a) = 0, \dot\xi(a) = 1$. 注意到$u^*$没有共轭点, 故我们可以选取合适的$a$和$b$, 使得$\xi(t) > 0, t \in (a, b) \supseteq J_1$.
        根据微分方程对于初值的连续依赖性, 故存在$0 < \varepsilon_1 < \varepsilon_0$, 使得 
        \begin{equation*}
            0 < \partial_{\alpha}\varphi(t, \alpha) \neq 0, \qquad \forall |\alpha| < \varepsilon, t \in J_1.
        \end{equation*}
        利用上述关系, 我们还可以引用隐函数定理, 故存在$0 < \varepsilon_2 < \varepsilon_1$, 使得对任意的$(t, u) \in \Omega_{\varepsilon_2}$, 方程$u = \varphi(t, \alpha)$存在唯一的$C^1$解$\alpha = w(t, u) \in B_{\varepsilon_2}(0)$.
        今令$\Omega = \Omega_{\varepsilon_2}$, 显然$\gamma^* \in \Omega$且$\Omega$是单连通的.
        最后我们只需寻找$\Omega$对应的方向场$\psi$. 事实上, 令 
        \begin{equation*}
            \psi(t, u) = \partial_t\varphi(t, w(t, u)),
        \end{equation*}
        则$\psi$在$\Omega$内处处有定义, 并且当$u = \varphi(t, \alpha)$时, 有 
        \begin{equation*}
            \dot u = \partial_t\varphi(t, \alpha) = \partial_t\varphi(t, w(t, u)) = \psi(t, u).
        \end{equation*}
        由此表明$\psi$是$\Omega$的一个方向场. 这便完成了$N = 1$的情形的证明.

        $N > 1$的情形是类似的. 对模长充分小的$\alpha \in \mathbb{R}^N$, 考虑满足初值条件 
        \begin{equation*}
            \partial_{\alpha_i}\varphi_j(a, \alpha) = 0, \quad \partial_{\alpha_i}\partial_t\varphi_j(a, \alpha) = \delta_{ij}, \qquad 1 \leq i, j \leq N
        \end{equation*}
        E-L方程的解. 再令$w_i(t) = \partial_{\alpha_i}\varphi(t, \alpha)|_{\alpha = 0}, i = 1, \cdots, N$.
        可以验证$w_i$是一个Jacobi场. 注意到$w_i(a) = 0, \partial_tw_i(a) = e_i , i = 1, \cdots, N$, 且$u^*$没有共轭点, 故总可以找到一个充分小的$\varepsilon^* > 0$, 使得$\det(\partial_{\alpha_i}\varphi_j(t, \alpha)) \neq 0, \forall (t, \alpha) \in J_1 \times B_{\varepsilon^*}(0)$.
        其余部分的证明是相同的.
    \end{proof}
\end{proposition}

现在考虑第二个事实. 定义 
\begin{equation*}
    \begin{cases} 
        R_i(t, u) = L_{p_i}(t, u, \psi(t, u)), \\  
        H(t, u) = \psi(t, u) \cdot L_p(t, u, \psi(t, u)) - L(t, u, \psi(t, u)) 
    \end{cases}
\end{equation*}
以及1-形式 
\begin{equation*}
    \boxed{\omega = \sum\limits_{i = 1}^NR_i \,{\rm d}u_i - H \,{\rm d}t}.
\end{equation*}
称$\omega$为\textbf{Hilbert积分不变量}. 显然, $\omega$是闭形式 $\Rightarrow$ 积分与路径无关.
以下引入更多概念来刻画这一条件.

\begin{definition}
    称极值场$\Omega$是\textbf{Mayer场}, 如果它满足如下相容性条件:
    \begin{equation*}
        \boxed{\partial_{u_i}L_{p_j}(t, u(t), \psi(t, u(t))) = \partial_{u_j}L_{p_i}(t, u(t), \psi(t, u(t))), \qquad \forall 1 \leq i, j \leq N.} 
    \end{equation*}
\end{definition}

\begin{proposition}[Mayer场的等价刻画]
    若$L \in C^2(J \times \mathbb{R}^N \times \mathbb{R}^N)$, 那么$(\Omega, \psi)$是一个Mayer场, 当且仅当${\rm d}\omega = 0$, 即$\partial_tR_i = -\partial_{u_i}H,1 \leq i \leq N$.
    \begin{proof}
        记$\widetilde{L} = \widetilde{L}(t) = L(t, u(t), \psi(t, u(t)))$. 类似地记$\widetilde{L_{u_i}}, \widetilde{L_{p_i}}$.
        利用条件$\dot u(t) = \psi(t, u(t))$和E-L方程$\widetilde{L_u} = \partial_t\widetilde{L_p}$, 可以得到$D_{\psi}\widetilde{L_p} = \widetilde{L_u}$, 其中 
        \begin{equation*}
            D_{\psi} = \partial_t + \sum\limits_{i = 1}^N\psi_i\partial_{u_i} + \sum\limits_{i = 1}^N\left(\partial_t\psi_i + \sum\limits_{k = 1}^N\psi_k\partial_{u_k}\psi_i\right)\partial_{p_i}. 
        \end{equation*}
        再通过直接计算可得
        \begin{equation*} 
            \begin{aligned} 
                \partial_tR_i &= \left(\partial_t + \sum_{j = 1}^N\partial_t\psi_j\partial_{p_j}\right)\widetilde{L_{p_i}}, \\ 
                \partial_{u_i}H &= \sum_{j = 1}^N\psi_j\partial_{u_i}\widetilde{L_{p_j}} - \widetilde{L_{u_i}}, 
            \end{aligned}  
            \qquad \forall 1 \leq i \leq N. 
        \end{equation*}
        可以验证, 相容性条件成立 $\Leftrightarrow$ $\partial_tR_i + \partial_{u_i}H = 0 =  D_{\psi}\widetilde{L_{p_i}} - \widetilde{L_{u_i}}$.
    \end{proof}
\end{proposition}

综上所述, \textbf{若$u^*$对应的极值曲线$\gamma^*$能够嵌入到一个Mayer场中}, 则上述两个事实均成立.

\begin{remark}
    由上述分析可知, 对于给定的Mayer场$(\Omega, \psi)$, 若设$\gamma$是连接$(t_0, u^*(t_0))$与$(t, u^*(t))$的任意一条曲线$(t_0 \leq t \leq t_1)$, 则线积分 
    \begin{equation*}
        \boxed{S(t, u) = \int_{\gamma}L_p \,{\rm d}u + (L - \psi \cdot L_p) \,{\rm d}t}
    \end{equation*}
    与$\gamma$无关. 称此积分为\textbf{Hilbert不变积分}.
\end{remark}

\begin{proposition}
    设$I$的E-L方程的解$u^*$对应的极值曲线$\gamma^*$能嵌入到一族曲线中去, 且这族曲线可以定义一个Mayer场$(\Omega, \psi)$.
    若
    \begin{equation*}
        \mathfrak{E}(t, u, \psi(t, u), p) \geq 0, \qquad \forall (t, u, p) \in \Omega \times \mathbb{R}^N,
    \end{equation*}
    则$u^*$是$I$的一个强极小点.
\end{proposition}

注意到\textbf{当$N = 1$时, 任何极值场都是Mayer场}, 故我们对上述充分条件有着更精准的刻画:

\begin{proposition}[充分条件, $N = 1$]\label{prop1.30}
    设$L \in C^3$, 并设其E-L方程的解$u^*$没有共轭点. 设$(\Omega, \psi)$为关于$u^*$的极值场.
    若$L$沿$u^*$满足严格Legendre-Hadamard条件, 则$u^*$是$I$的强极小点.
    \begin{proof}
        注意到对任意的$(t, u, p) \in \Omega \times \mathbb{R}$, 我们有 
        \begin{align*}
            \mathfrak{E}(t, u, \psi(t, u), p) &= L(t, u, p) - L(t, u, \psi(t, u)) - (p - \psi(t, u))L_p(t, u, \psi(t, u)) \\   
            &= L_{pp}(t, u, v) \geq 0, 
        \end{align*}
        其中$v$介于$p$和$\psi(t, u)$之间. 由此足以说明$u^*$是$I$的一个强极小点.
    \end{proof}
\end{proposition}

事实上, 对于高维的情形, 类似于命题\ref{prop1.30}的结论也是成立的.

\textbf{Analysis:} 给定Lagrange函数$L$和一族极值曲线$\{(t, \varphi(t, \alpha))\} \subseteq \Omega$, 其中$\Omega$是对应的极值场.
记$\overline{L}(t, \alpha) = L(t, \varphi(t, \alpha), \dot\varphi(t, \alpha))$, 其中$\dot\varphi(t, \alpha) = \partial_t\varphi(t, \alpha)$.
类似地记$\overline{L_{u_i}}, \overline{L_{p_i}}$. 直接计算得 
\begin{equation*}
    {\rm d}\omega = \sum_{i, \ell = 1}^N(\overline{L_{u_i}} - \partial_t\overline{L_{p_i}})\partial_{\alpha_\ell} {\rm d}\alpha_{\ell} \wedge {\rm d}t + \sum_{m, i, \ell = 1}^N\partial_{\alpha_m}\overline{L_{p_i}}\partial_{\alpha_{\ell}}\varphi_i {\rm d}\alpha_m \wedge {\rm d}\alpha_{\ell}.
\end{equation*}
现引入\textbf{Lagrange括号}
\begin{equation*}
    \boxed{[\alpha_\ell, \alpha_m] := \sum_{i = 1}^N(\partial_{\alpha_\ell}\overline{L_{p_i}}\partial_{\alpha_m}\varphi_i - \partial_{\alpha_m}\overline{L_{p_i}}\partial_{\alpha_\ell}\varphi_i).}
\end{equation*}
从而有
\begin{equation*}
    {\rm d}\omega = \sum_{i, \ell = 1}^NE_L(\varphi)_i\partial_{\alpha_\ell}\varphi_i {\rm d}\alpha_{\ell} \wedge {\rm d}t + \sum_{1 \leq \ell < m \leq N}[\alpha_\ell, \alpha_m] {\rm d}\alpha_{\ell} \wedge {\rm d}\alpha_m.
\end{equation*}
结合对Mayer场的等价刻画, 我们有

\begin{lemma}
    设$L \in C^3(\Omega \times \mathbb{R}^N \times \mathbb{R}^N)$. 又设$(\Omega, \psi)$是由一族极值曲线$\{\varphi(t, \alpha)\}$决定的极值场.
    则$(\Omega, \psi)$是一个Mayer场, 必须且只需 
    \begin{equation*}
        E_L(\varphi(\cdot, \alpha)) = 0, \quad \forall \alpha \in \mathbb{R}^N \quad \text{和} \quad [\alpha_{\ell}, \alpha_m] = 0, \quad \forall 1 \leq \ell, m \leq N.
    \end{equation*}
\end{lemma}

此外, 对于Lagrange括号, 有如下结果:

\begin{lemma}\label{lma1.32}
    设$L \in C^3(J \times \mathbb{R}^N \times \mathbb{R}^N)$, 且$(\Omega, \psi)$是一个极值场, 则 
    \begin{equation*}
        \partial_t[\alpha_\ell, \alpha_m] = 0, \qquad \forall 1 \leq \ell, m \leq N.
    \end{equation*}
    \begin{proof}
        利用E-L方程, 我们有 
        \begin{equation*}
            \partial_t[\alpha_\ell, \alpha_m] = \sum_{i = 1}^N\left(\partial_{\alpha_{\ell}}\overline{L_{u_i}}\partial_{\alpha_m}\varphi_i + \partial_{\alpha_{\ell}}\overline{L_{p_i}}\partial_{\alpha_m}\dot\varphi_i - \partial_{\alpha_m}\overline{L_{u_i}}\partial_{\alpha_{\ell}}\varphi_i - \partial_{\alpha_m}\overline{L_{p_i}}\partial_{\alpha_{\ell}}\dot\varphi_i\right).
        \end{equation*}
        将上式中的$\partial_{\alpha_{\ell}}\overline{L_{u_i}}, \partial_{\alpha_m}\overline{L_{u_i}}, \partial_{\alpha_{\ell}}\overline{L_{p_i}}, \partial_{\alpha_m}\overline{L_{p_i}}$展开, 即证得所需结论.
    \end{proof}
\end{lemma}

\begin{proposition}[充分条件, $N > 1$]
    设$L \in C^3(J \times \mathbb{R}^N \times \mathbb{R}^N)$. 如果其对应的E-L方程的解$u^*$没有共轭点, $L$沿$u^*$满足严格Legendre-Hadamard条件, 且 
    \begin{equation*}
        \mathfrak{E}(t, u, \psi(t, u), p) \geq 0, \quad \forall (t, u, p) \in \Omega \times \mathbb{R}^N. 
    \end{equation*}
    那么$u^*$是$I$的强极小点.
    \begin{proof}
        在命题\ref{prop1.25}的证明过程中我们构造了一个极值场$(\Omega, \psi)$, 若能证明此极值场是Mayer场, 那么命题的结论成立.
        事实上, 注意到初值条件$\partial_{\alpha_i}\varphi_j(a, \alpha) = 0, \forall i, j$, 则有$[\alpha_{\ell}, \alpha_m](a, \alpha) = 0, \forall \ell, m$.
        再根据引理\ref{lma1.32}, 则有$[\alpha_{\ell}, \alpha_m] = 0$. 这表明$(\Omega, \psi)$是一个Mayer场.
    \end{proof}
\end{proposition}

\begin{example}
    设 
    \begin{equation*}
        I(u) = \int_1^2(\dot u + t^2\dot u^2) \,{\rm d}t, \quad u \in M, 
    \end{equation*}
    其中$M = \{u \in C^1[1, 2]\colon u(1) = 1, u(2) = 2\}$. 验证 
    \begin{equation*}
        u^*(t) = -\frac{2}{t} + 3
    \end{equation*}
    是$I$的强极小点.
    \begin{proof}[解]
        直接计算得$L_u = L_{uu} = L_{pu} = 0, L_p = 1 + 2t^2p, L_{pp} = 2t^2$. 故其对应的E-L方程为
        \begin{equation*}
            \frac{\mathrm{d}}{\mathrm{d}t}(1 + 2t^2p) = 0.
        \end{equation*}
        显然$u^*$是满足E-L方程和边值条件的解, 且对应的Jacobi方程$t^2\dot\varphi(t) = \mathrm{const}$存在正解.
        现令 
        \begin{equation*}
            \Omega = \left\{\left(t, -\frac{2}{t} + \alpha\right)\colon (t, \alpha) \in (0, +\infty) \times \mathbb{R}\right\}, \quad \psi(t) = \frac{2}{t^2}. 
        \end{equation*}
        可以验证, $(\Omega, \psi)$是一个极值场, 且包含了$u^*$所对应的极值曲线. 注意到$L_{pp}(t, u, p) > 0, \forall(t, u, p) \in \Omega \times \mathbb{R}$, 因此由命题\ref{prop1.30}可知, $u^*$是$I$的一个强极小点.
    \end{proof}
\end{example}

\subsection{Hamilton-Jacobi理论}

\subsubsection{Hamilton方程组}

Hamilton方程组:
\begin{equation*}
    \begin{cases} 
        \dot\xi = -\partial_u H \\ 
        \dot u = \partial_{\xi}H. 
    \end{cases} 
\end{equation*}
其中$H = H(t, u(t), \xi(t))$. 利用Legendre变换, 上述方程组可以从变分的角度导出, 且其与E-L方程法有着深刻的联系.

\begin{definition}
    设$X$是赋范线性空间, 函数$\varphi\colon X \rightarrow (-\infty, +\infty]$满足
    \begin{equation*}
        \{x \in X\colon \varphi(x) < +\infty\} \neq \varnothing.
    \end{equation*}
    称函数 
    \begin{equation*}
        \boxed{\varphi^*\colon X^* \rightarrow (-\infty, +\infty], f \mapsto \sup_{x \in X}[\langle f, x\rangle - f(x)]}
    \end{equation*}
    为$\varphi$的\textbf{Legendre变换}(或称为$\varphi$的\textbf{共轭函数}), 其中$X^*$代表$X$的对偶空间.
\end{definition}

以上是Legendre变换的一般定义, 在这里我们只考虑标准欧式空间的情形: 若函数$f \in C^2(\mathbb{R}^N)$且其梯度$\xi = \nabla f(x)$有逆映射$\psi$, 则$f$的Legendre变换有表达式 
\begin{equation*}
    \boxed{f^*(\xi) = \xi \cdot x - f(x)= \xi \cdot \psi(\xi) - (f \circ \psi)(\xi).} 
\end{equation*}

\begin{remark}
    Legendre变换有着如下的几何意义: 记 
    \begin{equation*}
        G(f) = \{(x, y) \in \mathbb{R}^N \times \mathbb{R}\colon y = f(x)\}
    \end{equation*}
    为$f$的图像, 它在点$P = (x, y)$处的切平面
    \begin{equation*}
        S = \{(\alpha, \beta)\colon \beta - f(x) = \nabla f(x) \cdot (\alpha - x)\}. 
    \end{equation*}
    因此, 任取$S$上的一点$Q = (\alpha, \beta)$, 我们有 
    \begin{equation*}
        \beta - \nabla f(x) \cdot \alpha = f(x) - \nabla f(x) \cdot x,
    \end{equation*}
    从而有$\beta - \xi \cdot \alpha = -f^*(\xi)$. 这表明$-f^*(\xi)$是超平面$S$在$\beta$轴上的截距.
\end{remark}

\begin{proposition}
    Legendre变换有着如下简单性质:
    \begin{enumerate}
        \item 若$f \in C^k$, 则$f^* \in C^k$;
        \item $f^{**} = f$, 即Legendre变换是\textbf{自反的};
        \item $\left.(\partial_{\xi_i\xi_j}f^*(\xi))\right|_{\xi = \nabla f(x)} = (\partial_{x_ix_j}f(x))^{-1}$.
    \end{enumerate}
    \begin{proof}
        1. 注意到$x = \psi(\xi) \in C^{k - 1}$, 因此$f^* \in C^{k - 1}$. 另一方面, 注意到等式 
        \begin{equation*}
            \nabla f^*(\xi) = \psi(\xi) + \xi \cdot \nabla\psi(\xi) - \nabla f(\psi(\xi)) \cdot \psi(\xi) = \psi(\xi),
        \end{equation*}
        由此表明$f^* \in C^k$.

        2. 由1的证明过程可知$x = \psi(\xi) = \nabla f^*(\xi)$, 故有 
        \begin{equation*}
            f^{**}(x) = \xi \cdot x - f^*(\xi) = f(x).
        \end{equation*}

        3. 注意到$x = (\nabla f^*)(\nabla f(x))$, 等式两边取梯度, 即得 
        \begin{equation*}
            I_N = \left.(\partial_{\xi_i\xi_j}f^*(\xi))\right|_{\xi = \nabla f(x)} \cdot (\partial_{x_ix_j}f(x)).
        \end{equation*}
    \end{proof}
\end{proposition}

上述命题中展现了一些对称性的结果:

\begin{equation*}
    \boxed{f(x) + f^*(\xi) = \xi \cdot x, \quad \xi = \nabla f(x), \quad x = f^*(\xi).}
\end{equation*}

\begin{example}
    设$f(x) = x^p/p, x \geq 0, p > 1$. 直接计算得$f^*(\xi) = \xi^{p'}/p', \xi \geq 0$, 其中$1/p + 1/p' = 1$.
    注意到Legendre变换的定义, 我们便有 
    \begin{equation*}
        x\xi \leq \frac{1}{p}x^p + \frac{1}{p'}\xi^{p'}, \qquad x, \xi \geq 0.
    \end{equation*}
    这便是经典的Young不等式.
\end{example}

以下我们利用Legendre变换来推导Hamilton方程组. 给定$L \in C^2(\mathbb{R} \times \mathbb{R}^N \times \mathbb{R}^N)$, 并假设$\det(L_{p_ip_j}(t, u, p)) \neq 0$.
令$\xi = \xi = L_p(t, u, p)$, 根据隐函数定理, 我们可以局部地解出$p = \varphi(t, u, \xi)$, 即 
\begin{equation*}
    p_i = \varphi_i(t, u, \xi), \quad 1 \leq i \leq N.
\end{equation*} 
现固定$(t, u)$, 将$L$看作是$p$的函数, 并对其(关于$p$)作Legendre变换:
\begin{equation*}
    \boxed{H(t, u, \xi):= L^*(t, u, \xi) = (\xi \cdot p - L(t, u, p))|_{p = \varphi(t, u, \xi)}.}
\end{equation*}
称$H$是\textbf{Hamilton函数}(Hamiltonian). 从上述分析中可以看出, $H$无非是$L$的Legendre变换.
注意到Legendre变换是自反的, 故$L$也可以看作是$H$的Legendre变换.

接下来考虑$L$对应的E-L方程, 并将其改写成方程组的形式:
\begin{equation*}
    \begin{cases} 
        \dot u(t) = p(t), \\  
        \displaystyle\frac{{\rm d}}{{\rm d}t}L_p(t, u(t), p(t)) - L_u(t, u, p(t)) = 0,
    \end{cases}
\end{equation*}
设其解为$(u(t), p(t))$. 在等式$H(t, u, \xi)= \left.(\xi \cdot p - L(t, u, p))\right|_{p = \varphi(t, u, \xi)}$两边作微分, 即得 
\begin{equation*}
    H_t \,{\rm d}t + H_u \cdot {\rm d}u + H_{\xi} \cdot {\rm d}\xi = -L_t {\rm d}t - L_u \cdot {\rm d}u + p \cdot {\rm d}\xi,
\end{equation*}
从而有 
\begin{equation*}
    H_t = -L_t, \quad H_u = -L_u, \quad H_{\xi} = p.
\end{equation*}
若令$\xi(t) = L_p(t, u(t), p(t))$, 结合上述等式, 便有 
\begin{equation*}
    \dot\xi(t) = L_u(t, u(t), \dot u(t)) = L_u(t, u(t), p(t)) = -H_u(t, u(t), \xi(t)) 
\end{equation*}
与$\dot u(t) = p(t) = H_{\xi}(t, u(t), \xi(t))$. 综上所述, 我们有
\begin{equation*}
    \boxed{\begin{cases} 
        \dot\xi(t) = -H_u(t, u(t), \xi(t)), \\ 
        \dot u(t) = H_{\xi}(t, u(t), \xi(t)). 
    \end{cases}}
\end{equation*}
这便是经典的\textbf{Hamilton方程组}, 简称H-S.

\begin{remark}
    在上述分析中, 我们先假设$(u(t), p(t))$是E-L方程的解, 从而推导出$(u(t), \xi(t))$是H-S的解.
    反之, 对于给定的H-S的解$(u(t), \xi(t))$, 注意到关系$p(t) = \dot u(t)$和$\xi(t) = L_p(t, u(t), p(t))$, 我们有 
    \begin{equation*}
        \frac{{\rm d}}{{\rm d}t}L_p(t, u(t), \dot u(t)) = \dot\xi(t) = -H_u(t, u(t), \xi(t)) = L_u(t, u(t), \dot u(t)). 
    \end{equation*}
    这表明$(u(t), p(t))$是E-L方程的解. 综上, 我们得到了E-L方程和H-S二者之间的一个一一对应关系.
\end{remark}

以下考察H-S对应的变分积分. 可以验证, H-S是泛函 
\begin{equation*}
    \boxed{F(u, \xi) = \int_J(\dot u \cdot \xi - H) \,{\rm d}t}
\end{equation*}
所对应的E-L方程, 相应的1-形式是 
\begin{equation*}
    \boxed{\alpha = \xi \cdot \,{\rm d}u - H \,{\rm d}t,}
\end{equation*}
我们称其为\textbf{Poincaré-Cartan积分不变量}. 注意到$H$是$L$的Legendre变换, 因此泛函$F$与$I$表达式中的被积函数实际上是\textbf{同一个函数在不同变量下的表示}, 而Poincaré-Cartan积分不变量则就是上一节中提到的Hilbert积分不变量.

\begin{remark}
    Hamilton方程组对应的泛函$F$不是下方有界的, 从而没有最小值, 因此H-S的解是相应泛函的``临界点''.
    在实际问题中, 我们视实际情况从而选用E-L方程或H-S进行求解.
\end{remark}

\begin{example}
    对于有$n$个自由度的质点组, 记位置坐标$q = (q_1, \cdots, q_N)$, 则其动能$T = T(q) = \sum_{i, j = 1}^Na_{ij}\dot q_i\dot q_j/2$, 其中$(a_{ij})$是正定阵.
    设位能$V = V(q)$, 则Lagrange函数$L = T - V$. 通过直接计算可知, $L$对应的E-L方程为 
    \begin{equation*}
        \frac{{\rm d}}{{\rm d}t}\sum_{j = 1}^na_{ij}\dot q_j = -\partial_{q_i}V(q) \quad (i = 1, \cdots, N).
    \end{equation*}
    此时Hamilton函数 
    \begin{equation*}
        H(q, \xi) = \frac{1}{2}\sum_{i, j = 1}^Na^{ij}\xi_i\xi_j + V(q)
    \end{equation*}
    是这个质点组的能量, 其中$(a^{ij}) = (a_{ij})^{-1}$. 对应的Hamilton方程组为 
    \begin{equation*}
        \begin{cases} 
            \dot\xi_i = -\partial_{q_i}V(q), \\  
            \dot q_i = \sum_{j = 1}^Na^{ij}\xi_j,  
        \end{cases} 
        \quad i= 1, \cdots, N.
    \end{equation*}
    一般地, 若设$(u(t), p(t))$是E-L方程的解, 直接计算得 
    \begin{equation*}
        \boxed{\frac{{\rm d}}{{\rm d}t}H(u(t), \xi(t)) = 0.}
    \end{equation*}
    即\textbf{Hamilton方程组的解曲线都保持在同一个等值面上}. 将此结果运用到上述分析中, 即得: 在运动过程中质点组的能量守恒.
\end{example}

\subsubsection{Hamilton-Jacobi方程}

对于给定的Hamilton函数$H = H(t, u, \xi)$, 称一阶偏微分方程 
\begin{equation*}
    \boxed{\partial_tS(t, u) + H(t, u, \nabla_uS(t, u)) = 0}
\end{equation*}
为\textbf{Hamilton-Jacobi方程}(简称\textbf{H-J方程}), 其中$S = S(t, u)$是定义在$\mathbb{R}^{1 + N}$上的函数.
以下我们结合Mayer场和Legendre变换等概念, 导出此方程, 并探究其与H-S之间的联系.

先引入一些概念. 给定一个Lagrange函数$L$. 对于一个极值场$(\Omega, \psi)$, 其上有一个对应的1-形式, 即Hilbert不变积分因子:
\begin{equation*}
    \omega = L_p(t, u, \psi(t, u)) \cdot {\rm d}u - (\psi(t, u) \cdot L_p(t, u, \psi(t, u)) - L(t, u, \psi(t, u))) {\rm d}t.
\end{equation*}
我们已经知道, $(\Omega, \psi)$是一个Mayer场, 当且仅当$\omega$是闭的. 因此, 在一个Mayer场上我们可以由$\omega$定义出$\Omega$上的一个单值函数$g$:
\begin{equation*}
    \boxed{g(t, u) := g(t_0, u_0) + \int_{\gamma}\omega,}
\end{equation*}
其中$\gamma$是连接$(t_0, u_0)$与$(t, u)$的任意一条曲线. 我们称此单值函数$g$为\textbf{程函}.
此外, 由程函定义可知, $g$满足如下方程组:
\begin{equation*}
    \boxed{\begin{cases} \
        \nabla_ug(t, u) = L_p(t, u, \psi(t, u)), \\  
        \partial_tg(t, u) = L(t, u, \psi(t, u)) - \psi(t, u) \cdot L_p(t, u, \psi(t, u)). 
    \end{cases}}
\end{equation*}
称此方程组为\textbf{Carathéodory方程组}.

\begin{example}
    设$\gamma = (t, u(t)) \subseteq (\Omega, \psi)$是一条极值曲线. 直接计算得 
    \begin{equation*}
        g(t_2, u(t_2)) - g(t_1, u(t_1)) = \int_{\gamma}\omega = \int_JL(t, u(t), \dot u(t)) \,{\rm d}t.
    \end{equation*}
    由此表明, \textbf{程函在同一极值曲线上两点的差等于Lagrange函数沿这条曲线的积分}.
    在光学中, Lagrange函数表示光在传播中瞬时走过的路程除以速度, 沿这条曲线的积分就等于光线从$(t_1, u(t_1))$传播到$(t_2, u(t_2))$所经历的时间.
    由上述分析可知, 程函的等值面$\{g(t, u) = {\rm const}\}$可以用来表示从一点发出的一束光线的等时面, 即波阵面.
\end{example}

以下我们导出H-J方程. 给定Mayer场$(\Omega, \psi)$, 其对应的程函$g$满足Carathéodory方程组.
将$\xi = L_p(t, u, \psi(t, u))$代入至Carathéodory方程组中, 即得  
\begin{equation*}
    \partial_tg(t, u) + H(t, u, \nabla_ug(t, u)) = 0,
\end{equation*}
其中$H$是$L$的Legendre变换, 即Hamilton函数. 这表明$g$满足H-J方程.

\begin{remark}
    注意到$H$是$L$的Legendre变换, 对于给定的Hamilton函数$H$和对应的H-S的一组解$(u(t), \xi(t))$, 令$p(t) = H_{\xi}(t, u(t), \xi(t))$, 则$(u(t), p(t))$便是E-L方程的解. 再利用等式
    \begin{equation*}
        L(t, u(t), p(t)) = \dot u(t) \cdot \xi(t) - H(t, u(t), \xi(t)),
    \end{equation*} 
    我们便可以写出Lagrange函数$L$. 因此 
    \begin{equation*}
        g(t, u) = g(t_0, u(t_0)) + \int_{t_0}^t L(t, u(t), p(t)) \,{\rm d}t
    \end{equation*}
    便是H-J方程的一个解. 上述分析表明, 我们可以从任取初值得到的H-S的所有解导出H-J的解.
\end{remark}

事实上, 我们也可以从H-J方程的解导出H-S的解.

\begin{definition}
    设$g = g(t, u; \lambda_1, \cdots, \lambda_N)$是H-J方程一族依赖于$N$个参数$(\lambda_1, \cdots, \lambda_N) \in \Lambda$的解, 其中$\Lambda \in \mathbb{R}^N$是一个区域.
    如果$\det(g_{u_i\lambda_j}) \neq 0$, 那么称$g$为一个\textbf{完全积分}.
\end{definition}

\begin{theorem}[Jacobi]
    设$C^2$函数$g = g(t, u; \lambda_1, \cdots, \lambda_N)$是H-J方程的一个完全积分. 若依赖于$2N$个参数$(\alpha, \beta) = (\alpha_1, \cdots, \alpha_N, \beta_1, \cdots, \beta_N)$的函数 
    \begin{equation*}
        \begin{cases} 
            u = U(t, \alpha, \beta), \\  
            p = P(t, \alpha, \beta)  
        \end{cases}
    \end{equation*}
    满足方程
    \begin{equation}\label{9}
        \begin{cases} 
            g_{\alpha_i}(t, U(t, \alpha, \beta), \alpha) = -\beta_i, \\  
            P_i(t, \alpha, \beta) = g_{u_i}(t, U(t, \alpha, \beta), \alpha),  
        \end{cases}  
        \quad  i = 1, \cdots, N,
    \end{equation}
    那么$(U, P)$便是H-S的一族解.
    \begin{proof}
        先在H-J方程的两端对$\alpha_i$求偏导, 即得 
        \begin{equation*}
            g_{t, \alpha_i} + \sum_{k = 1}^NH_{\xi_k}(t, u, \nabla_ug)g_{u_k, \alpha_i} = 0, \quad i = 1, \cdots, N.
        \end{equation*}
        将等式$u = U(t, \alpha, \beta)$代入至上式, 并利用\eqref{9}的第二个等式, 便有 
        \begin{equation}\label{10}
            g_{t, \alpha_i}(t, U, \alpha) + \sum_{k = 1}^NH_{\xi_k}(t, U, P)g_{u_k\alpha_i}(t, U, \alpha) = 0.
        \end{equation}
        再对\eqref{9}的第一个方程等式两边对$t$求偏导, 我们有
        \begin{equation}\label{11}
            g_{t, \alpha_i}(t, U, \alpha) + \sum_{k = 1}^Ng_{\alpha_iu_k}(t, U, \alpha)\dot U_k(t, \alpha, \beta) = 0.
        \end{equation} 
        联立\eqref{10}和\eqref{11}, 并注意到$g$是完全积分, 从而有 
        \begin{equation*}
            \dot U_k(t, \alpha, \beta) =H_{\xi_k}(t, U(t, \alpha, \beta), P(t, \alpha, \beta)), \quad k = 1, \cdots, N.
        \end{equation*}
        这是H-S的一组方程. 另一方面, 对H-J方程等式两边对$u_i$求偏导, 并将等式$u = U(t, \alpha, \beta), P(t, \alpha, \beta) = \nabla_ug(t, U(t, \alpha, \beta), \alpha)$代入, 得 
        \begin{equation*}
            -H_{u_i}(t, U, P) = g_{t, u_i}(t, U, \alpha) + \sum_{k = 1}^Ng_{u_iu_k}(t, U, \alpha)\dot U_k(t, \alpha, \beta).
        \end{equation*}
        对\eqref{9}的第二个方程等式两边对$t$求偏导, 我们有 
        \begin{equation*}
            \dot P_i(t, \alpha, \beta) = g_{u_i, t}(t, U, \alpha) + \sum_{k = 1}^Ng_{u_iu_k}(t, U, \alpha)\dot U_k(t, \alpha, \beta).
        \end{equation*}
        从而我们得到 
        \begin{equation*}
            \dot P_k(t, \alpha, \beta) = -H_{u_k}(t, U, P), \quad k = 1, \cdots, N.
        \end{equation*}
        这便是H-S的另一组方程.
    \end{proof}
\end{theorem}

由Jacobi定理可知, 我们可以用H-J方程的解写出H-S的解. 具体方法如下: 设$g$是一个完全积分, 先解$N$个函数方程 
\begin{equation*}
    g_{\alpha_i}(t, u, \alpha) = -\beta_i, \quad i = 1, \cdots, N,
\end{equation*}
由此得到 
\begin{equation}\label{12}
    u = U(t, \alpha, \beta). 
\end{equation}
因此
\begin{equation}\label{13}
    p = P(t, \alpha, \beta) = \nabla_ug(t, U(t, \alpha, \beta), \alpha).
\end{equation}
从而$(u, p)$就是H-S的解.

\begin{remark}
    注意到完全积分与通解的意义是不同的. 若考虑H-J方程的Cauchy问题, 由唯一性可知, 它的通解里应该含有一个任意函数$\varphi = \varphi(u)$, 而不仅仅是$2N$个独立参数.
    但是, H-S初值问题的解可以由H-J方程的一个完全积分$g$所确定. 具体地, 我们考虑H-S方程的初值问题 
    \begin{equation}\label{14}
        \begin{cases} 
            \dot u = H_{\xi}(t, u, \xi), \\ 
            \dot \xi = -H_u(t, u, \xi), \\  
            u(0) = u_0, \xi(0) = \xi_0,  
        \end{cases}
    \end{equation}
    其中$u_0, \xi_0$是任意常数. 如果$g$是一个完全积分, 那么$\det(g_{u_i\alpha_j}) \neq 0$, 我们因此可以使用隐函数定理对方程 
    \begin{equation*}
        \xi_0 = \nabla_ug(0, u_0, \alpha), 
    \end{equation*}
    解出$\alpha_0 = \alpha(u_0, \xi_0)$. 再令 
    \begin{equation*}
        \beta_0 = -\nabla_{\alpha}g(0, u_0, \alpha_0), 
    \end{equation*}
    并将$(\alpha_0, \beta_0)$作为初值代入至\eqref{12}和\eqref{13}中, 我们便得到了初值问题\eqref{14}的解.
\end{remark}

\subsubsection{例}

\begin{example}[光在介质中的传播]
    设在介质中一点$(t, u) \in \mathbb{R} \times \mathbb{R}^N$的介质密度是$\rho = \rho(t, u)$.
    以真空光速为单位, 若在此点的光速为$1/\rho(t, u)$, 则对应的Lagrange函数为 
    \begin{equation*}
        L(t, u, p) = \rho(t, u)\sqrt{1 + p^2}.
    \end{equation*}
    从而有 
    \begin{equation*}
        H(t, u, \xi) = -\sqrt{\rho(t, u)^2 - \xi^2}.
    \end{equation*}
    此时程函$g$满足H-J方程 
    \begin{equation*}
        \partial_tg = \sqrt{\rho^2 - |\nabla u|^2},
    \end{equation*}
    对应的方向场 
    \begin{equation*}
        \psi(t, u) = H_{\xi}(t, u, \nabla_ug) = \frac{\nabla_ug}{\sqrt{\rho^2 - |\nabla_ug|^2}} = \frac{\nabla_ug}{\partial_tg}.
    \end{equation*}
    即有 
    \begin{equation*}
        (\dot t, \dot u) = (1, \dot u) = (\partial_tg)^{-1}(\partial_tg, \nabla_ug).
    \end{equation*}
    由上述结果可知, 积分曲线$(t, u)$空间中沿波阵面$\{g(t, u) = {\rm const}\}$的法方向.
    上述分析表明: 光线垂直于波阵面.
\end{example}

\begin{example}[简谐振动]
    给定Lagrange函数 
    \begin{equation*}
        L= \frac{1}{2}(mp^2 - ku^2),
    \end{equation*}
    其中$m$与$k$都是正常数. 通过直接计算可知, 其对应的Hamilton函数 
    \begin{equation*}
        H(t, u, p) = \frac{1}{2}\left(\frac{p^2}{m} + ku^2\right), 
    \end{equation*}
    且对应的H-S 
    \begin{equation}\label{15}
        \begin{cases} 
            \displaystyle\dot u = \frac{p}{m}, \\  
            \dot p = -ku 
        \end{cases}
    \end{equation}
    有解
    \begin{equation*}
        \begin{cases} 
            \displaystyle u = C\sin\left(\sqrt{\frac{k}{m}}(t + t_0)\right), \\  
            \displaystyle p = C\sqrt{mk}\cos\left(\sqrt{\frac{k}{m}}(t + t_0)\right), 
        \end{cases}
    \end{equation*}
    其中$t_0, C$是任意常数. 我们现在利用Jacobi定理, 通过H-J方程把\eqref{15}的解写出来.
    考虑一个特殊的$g(t, u, \alpha) = \varphi(u, \alpha) - \alpha t$, 其中$\alpha$是一个参数, $\varphi$是一个待定的函数.
    将此代入至H-J方程中, 得
    \begin{equation*}
        \frac{1}{2}\left(\frac{\varphi_u^2}{m} + ku^2\right) = \alpha,
    \end{equation*} 
    即$\varphi_u = \sqrt{m(2\alpha - ku^2)}$. 解出来有 
    \begin{equation*}
        g(t, u, \alpha) = \int_0^u\sqrt{m(2\alpha - kv^2)} \,{\rm d}v - \alpha t.
    \end{equation*}
    此时$g_{\alpha u} = 2m \neq 0$. 现考虑方程 
    \begin{equation*}
        -\beta = g_{\alpha}(t, u, \alpha) = \sqrt{\frac{m}{k}}\arcsin\left(\sqrt{\frac{k}{2\alpha}}u\right) - t,
    \end{equation*}
    解得 
    \begin{equation*}
        u = \sqrt{\frac{2\alpha}{k}}\sin\left(\frac{k}{m}(t - \beta)\right). 
    \end{equation*}
    将上式代入至$g$中, 即得 
    \begin{equation*}
        p = g_u = \sqrt{2\alpha m}\cos\left(\frac{k}{m}(t - \beta)\right).
    \end{equation*}
    这便是\eqref{15}带有两个参数$\alpha, \beta$的解.
\end{example}

\subsection{含多重积分的变分问题}

在这一节中, 我们将前几节中所讨论的结果推广到高维的情形. 先引入如下记号:
\begin{align*}
    x &= (x_i)_{1 \leq i \leq n} = (x_1, \cdots, x_n) \in \mathbb{R}^n, \\  
    u &= (u^m)_{1 \leq m \leq N} = (u_1, \cdots, u_N) \in \mathbb{R}^N, \\  
    p &= (p_i^m)_{1 \leq i \leq n, 1 \leq m \leq N} \in \mathbb{R}^{nN}, \\  
    \nabla u &= (\partial_iu^m)_{1 \leq i \leq n, 1 \leq m \leq N} \in \mathbb{R}^{nN}.
\end{align*}
给定$\mathbb{R}^n$中带有$C^1$边界的有界区域$\Omega$, Lagrange函数$L = L(x, u, p) \in C^2(\overline{\Omega} \times \mathbb{R}^N \times \mathbb{R}^{nN})$以及边界上的函数$\Phi \in C^1(\partial\Omega)$.
考虑泛函
\begin{equation*}
    I(u) = \int_{\Omega} L(x, u(x), \nabla u(x)) \,{\rm d}x
\end{equation*}
在边值条件$u \in M = \{v \in C^1(\overline{\Omega})\colon v|_{\partial\Omega} = \Phi\}$下的极小值.

\subsubsection{Euler-Lagrange方程}

类似于单重积分的情形($n = 1$), 称$u^* \in M$是$I$在$M$上的\textbf{极小点}, 如果 
\begin{equation*}
    I(u) \geq I(u^*), \qquad \forall u \in U \cap M,
\end{equation*}
其中$U$是$u^*$在$M$中的一个邻域. 取$C^1$拓扑时, 称$u^*$为\textbf{弱极小点}; 取$C$拓扑时, 称$u^*$为\textbf{强极小点}.

\begin{lemma}[du Bois-Reymond, 变分学基本引理]
    设$u \in L^1_{{\rm loc}}(\Omega)$, 且对任意的$\varphi \in C_c^{\infty}(\Omega)$有 
    \begin{equation*}
        \int_{\Omega}u(x)\varphi(x) \,{\rm d}x = 0,
    \end{equation*}
    则$u(x) = 0$, a.e. $x \in \Omega$.
    \begin{proof}
        设$\{\eta_n\}$是一族光滑化子. 令$g_n = g \ast \eta_n$, 其中$g \in L^{\infty}(\mathbb{R}^n)$且${\rm supp} \ g \subseteq \Omega$.
        显然$g \in L^1(\mathbb{R}^n)$, 且当$n$充分大时, 有$g_n \in C_c^{\infty}(\Omega)$, 从而
        \begin{equation*}
            \int_{\Omega}u(x)g_n(x) \,{\rm d}x = 0.
        \end{equation*}
        此外, 由恒等逼近的理论可知, $\Vert g_n - g \Vert_{L^1} \rightarrow 0, n \rightarrow \infty$.
        因此存在子列, 不妨记为$\{g_n\}$, 使得$g_n \rightarrow g$, a.e. 注意到 
        \begin{equation*}
            \Vert g_n\Vert_{L^{\infty}} = \Vert g \ast \eta_n \Vert_{L^{\infty}} \leq \Vert \eta_n \Vert_{L^1}\Vert g \Vert_{L^{\infty}} = \Vert g \Vert_{L^{\infty}},
        \end{equation*}
        故由控制收敛定理可得 
        \begin{equation}\label{16}
            \int_{\Omega}u(x)g(x) \,{\rm d}x = 0.
        \end{equation}
        今在\eqref{16}中取 
        \begin{equation*}
            g(x) =  
            \begin{cases} 
                {\rm sgn}\ u(x) \quad &x \in K, \\   
                0 \quad &x \in \mathbb{R}^n \smallsetminus K,  
            \end{cases}
        \end{equation*}
        其中$K$是$\Omega$内的紧集. 由此我们可以得到$u(x) = 0$, a.e. $x \in K$.
        注意到$K$是任意的, 故$u = 0$, a.e.
    \end{proof}
\end{lemma}

以下利用高维形式的变分法基本引理来推导出E-L方程. 设$L \in C^2, u^* \in C^2$.
对任意的$\varphi \in C_0^1(\Omega)$, 考虑一元函数$g(\varepsilon) = I(u^* + \varepsilon\varphi)$.
由于$u^*$是极小点, 则一阶变分$\delta I(u^*, \varphi) = \dot g(0) = 0$, 即 
\begin{align*}
    0 &= \sum_{m = 1}^N\int_{\Omega}\left(L_{u^m}(\tau)\varphi^m(x) + \sum_{i = 1}^nL_{p^m_i}(\tau)\partial_i\varphi^m(x)\right)\,{\rm d}x \\  
    &= \sum_{m = 1}^N\int_{\Omega}\left(L_{u^m}(\tau) - \sum_{i = 1}^n\partial_iL_{p^m_i}(\tau)\right)\varphi^m(x) \,{\rm d}x,
\end{align*}
其中$\tau = (x, u^*(x), \nabla u^*(x))$. 再利用变分法基本引理, 我们便得到
\begin{equation*}
    \boxed{L_{u^m}(\tau) - \sum_{i = 1}^n\partial_iL_{p^m_i}(\tau) =0, \qquad 1 \leq m \leq N,}
\end{equation*}
即\textbf{Euler-Lagrange方程}. 上式还可以等价地写为 
\begin{equation*}
    \boxed{L_{u^m}(\tau) - {\rm div}\ L_{p^m}(\tau) = 0, \qquad 1 \leq m \leq N.}
\end{equation*}
或直接简记为$L_u(\tau) - {\rm div}\ L_p(\tau) = 0$. 由此可以看出,当$n = 1$时, 上述E-L方程与我们第一节所导出的E-L方程是相同的.
若$u^*$的光滑性较差, 上述E-L方程应在广义导数的意义下理解.

类似地, 称算子$E_L\colon u \mapsto v = (v_1, \cdots, v_n)$, 其中 
\begin{equation*}
    \boxed{v_m  = L_{u^m}(\tau) - \sum_{i = 1}^n\partial_iL_{p^m_i}(\tau), \qquad m = 1, \cdots, N}
\end{equation*}
为关于$L$的\textbf{Euler-Lagrange算子}.

\begin{example}
    设$N = 1, L(p) = |p|^2/2 = (p_1^2 + \cdots + p_n^2)/2$. 对于泛函 
    \begin{equation*}
        I(u) = \int_{\Omega}L(p) \,{\rm d}x = \frac{1}{2}\int_{\Omega}|\nabla u|^2 \,{\rm d}x,
    \end{equation*}
    其对应的E-L方程为$\nabla \cdot \nabla u = 0$, 即 
    \begin{equation*}
        \Delta u(x) = 0, \qquad \forall x \in \Omega.
    \end{equation*}
    这便是Laplace方程.
\end{example}

\begin{example}
    用$\mathbb{R} \times \mathbb{R}^3$表示时空连续统, 时空中任意一点的坐标是$(t, x) = (t, x_1, x_2, x_3)$, 其中$t$表示时间, $x = (x_1, x_2, x_3)$表示空间位置.
    如果我们用$u = u(t, x)$表示在时空区域$\Omega \subseteq \mathbb{R} \times \mathbb{R}^3$内弹性波的位移, 那么弹性波的动能是 
    \begin{equation*}
        T(u) = \frac{1}{2}\int_{\Omega}|\partial_tu(t, x)|^2 \,{\rm d}t\,{\rm d}x, 
    \end{equation*}
    势能 
    \begin{equation*}
        U(u) =  \frac{1}{2}\int_{\Omega}|\nabla_xu(t, x)|^2 \,{\rm d}t\,{\rm d}x,
    \end{equation*}
    由此对应的Lagrange函数 
    \begin{equation*}
        I(u) = T(u) - U(u) = \frac{1}{2}\int_{\Omega}(|\partial_tu(t, x)|^2 - |\nabla_xu(t, x)|^2) \,{\rm d}t\,{\rm d}x. 
    \end{equation*}
    通过直接计算可知, $I$对应的E-L方程为 
    \begin{equation*}
        \square u := \partial_t^2u - \Delta u = 0.
    \end{equation*}
    这便是经典的波动方程. 类似地, 如果还有内力或外力存在, 那么在势能中可以再添加一些其他项. 例如:
    \begin{equation*}
        U(u) =  \frac{1}{2}\int_{\Omega}(|\nabla_xu(t, x)|^2 + m^2|u(t, x)|^2) \,{\rm d}t\,{\rm d}x,
    \end{equation*} 
    其中$m > 0$是一个常数. 此时对应的E-L方程为
    \begin{equation*}
        \square u - m^2u = 0.
    \end{equation*} 
    这是Klein-Gordon方程. 又如
    \begin{equation*}
        U(u) =  \int_{\Omega}\left(\frac{1}{2}|\nabla_xu(t, x)|^2 +\frac{1}{4} |u(t, x)|^2\right) \,{\rm d}t\,{\rm d}x.
    \end{equation*} 
    此时对应的E-L方程为 
    \begin{equation*}
        \square u + u^3 = 0.
    \end{equation*}
    这是一个非线性波动方程.
\end{example}

\begin{example}[极小曲面]
    设$\Omega \subseteq \mathbb{R}^n$. 给定函数$u \in C^1(\overline{\Omega})$, 其对应的超曲面$\{(x, u(x))\colon x \in \overline{\Omega}\}$的面积是
    \begin{equation*}
        A(u) = \int_{\Omega}\sqrt{1 + |\nabla u(x)|^2} \,{\rm d}x.
    \end{equation*}
    在给定边值$u|_{\partial\Omega} = \Phi$的条件下, 我们要寻求$u$使得面积$A = A(u)$达到极小. 将$A$看作是关于$U$的泛函, $A$对应的E-L方程为 
    \begin{equation*}
        {\rm div}\ \frac{\nabla u(x)}{\sqrt{1 + |\nabla u(x)|^2}} = 0, \qquad \forall x \in \Omega.
    \end{equation*}
    这便是极小曲面所满足的方程. 注意到平均曲率有表达式
    \begin{equation*}
        H = \frac{1}{n}{\rm div}\ \frac{\nabla u(x)}{\sqrt{1 + |\nabla u(x)|^2}},
    \end{equation*}
    由此可知, 平均曲率等于零的方程就是极小曲面的方程. 我们有时候也将此作为极小曲面的定义, 即平均曲率为零的曲面.
\end{example}

\subsubsection{必要条件: Legendre-Hadamard条件}

二阶变分$\delta^2I(u^*, \varphi) = \ddot g(0)$. 直接计算得 
\begin{align*}
    \delta^2I(u^*, \varphi) &= \sum_{m, n = 1}^N\int_{\Omega} F(x) \,{\rm d}x,
\end{align*}
其中 
\begin{equation*}
    F(x) = L_{u^mu^\ell}(\tau)\varphi^m(x)\varphi^\ell(x) + 2\sum_{i = 1}^nL_{u^mp^\ell_i}(\tau)\varphi^m(x)\partial_i\varphi^{\ell}(x) + \sum_{i, j = 1}^nL_{p^m_ip^{\ell}_j}(\tau)\partial_i\varphi^m(x)\partial_j\varphi^\ell(x),
\end{equation*}
而$\tau = (x, u^*(x), \nabla u^*(x))$. 为了书写的简便, 引入记号 
\begin{align*}
    A_{u^*} &= (a_{ij}^{m\ell}) = (L_{p^m_ip^{\ell}_j}(\tau)), \\ 
    B_{u^*} &= (b_j^{m\ell}) = (L_{u^mp^\ell_j}(\tau)), \\ 
    C_{u^*} &= (c^{m\ell}) = (L_{u^mu^\ell}(\tau)),
\end{align*}
并记
\begin{equation*}
    Q_{u^*}(\varphi) = \delta^2I(u^*, \varphi) = \int_{\Omega}(A_{u^*}(\nabla\varphi, \nabla\varphi) + 2B_{u^*}(\nabla\varphi, \varphi) + C_{u^*}(\varphi, \varphi)) \,{\rm d}x.
\end{equation*}
显然, 若$u^*$是一个(弱)极小点, 则对任意的$\varphi \in C_0^1(\Omega)$有$Q_{u^*}(\varphi) \geq 0$.

\begin{proposition}
    设$\Omega \subseteq \mathbb{R}^n$是一个有界区域, Lagrange函数$L \in C^2(\overline{\Omega} \times \mathbb{R}^N \times \mathbb{R}^{nN})$.
    若$u^*$是对应的E-L方程的解, 则\textbf{Legendre-Hadamard条件}
    \begin{equation}\label{17}
        \boxed{\sum_{m, \ell = 1}^N\sum_{i, j = 1}^nL_{p_i^mp_j^\ell}(\tau)\xi^m\xi^\ell\eta_i\eta_j \geq 0, \quad \forall (x, \xi, \eta) \in \Omega \times \mathbb{R}^N \times \mathbb{R}^n.} 
    \end{equation}
    成立.
    \begin{proof}
        对任意的$x_0 \in \Omega$, 取向量值函数$v \in C_c^{\infty}(B_1(0))$. 当$\mu > 0$充分小时, 令 
        \begin{equation*}
            \varphi(x) = \mu v\left(\frac{x - x_0}{\mu}\right). 
        \end{equation*}
        将其代入至二阶变分的具体表达式中, 即得 
        \begin{align*}
            0 \leq \mu^n&\int_{B_1(0)}(A_{u^*}(x_0 + \mu y)(\nabla v(y), \nabla v(y)) \\&
            + 2\mu B_{u^*}(x_0 + \mu y)(\nabla v(y), v(y)) + \mu^2C_{u^*}(x_0 + \mu y)(v(y), v(y))) \,{\rm d}y.
        \end{align*}
        令$\mu \rightarrow 0$, 即得
        \begin{equation}\label{18}
            \sum_{m, \ell = 1}^N\sum_{i, j = 1}^nA_{u^*}(x_0)\int_{B_1(0)}\partial_iv^m(y)\partial_jv^n(y) \,{\rm d}y \geq 0.
        \end{equation}
        现取函数$\rho \in C_c^{\infty}(B_1(0))$满足$\Vert \rho \Vert_{L^2(B_1(0))} = 1$.
        对任意的$t > 0, \xi \in \mathbb{R}^N, \eta \in \mathbb{R}^n$, 分别将 
        \begin{equation*}
            v_1(y) = \xi\cos(t\eta \cdot y)\rho(y), 
        \end{equation*}
        和
        \begin{equation*}
            v_2(y) = \xi\sin(t\eta \cdot y)\rho(y)
        \end{equation*}
        代入至\eqref{18}中再相加, 即得 
        \begin{equation*}
            \sum_{m, \ell = 1}^N\sum_{i, j = 1}^nA_{u^*}(x_0)\xi^m\xi^\ell\eta_i\eta_j + O(t^{-1}) \geq 0 \qquad (t \rightarrow +\infty), 
        \end{equation*}
        从而有 
        \begin{equation*}
            \sum_{m, \ell = 1}^N\sum_{i, j = 1}^nA_{u^*}(x_0)\xi^m\xi^\ell\eta_i\eta_j \geq 0.
        \end{equation*}
        这便是Legendre-Hadamard条件\eqref{17}.
    \end{proof}
\end{proposition}

\begin{remark}
    若使用秩1矩阵的符号
\begin{equation*}
    \pi = (\pi_i^m) = (\xi^m\eta_i), 
\end{equation*}
那么\eqref{17}可以等价写为 
\begin{equation*}
    \sum_{m, \ell = 1}^N\sum_{i, j = 1}^nL_{p_i^mp_j^\ell}(\tau)\pi_i^m\pi_j^\ell \geq 0, \qquad \forall \pi, {\rm rank}(\pi) = 1.
\end{equation*}
\end{remark}

类似地, 若存在$\lambda > 0$使得 
\begin{equation*}
    \boxed{\sum_{m, \ell = 1}^N\sum_{i, j = 1}^nL_{p_i^mp_j^\ell}(\tau)\xi^m\xi^\ell\eta_i\eta_j \geq \lambda|\xi|^2|\eta|^2, \quad \forall (x, \xi, \eta) \in \Omega \times \mathbb{R}^N \times \mathbb{R}^n.}
\end{equation*}
那么我们称其为\textbf{严格Legendre-Hadamard条件}. 利用秩1矩阵的符号, 上式可以等价写为 
\begin{equation*}
    \sum_{m, \ell = 1}^N\sum_{i, j = 1}^nL_{p_i^mp_j^\ell}(\tau)\pi_i^m\pi_j^\ell \geq \lambda\Vert \pi\Vert^2, \qquad \forall \pi, {\rm rank}(\pi) = 1. 
\end{equation*}
这里$\Vert \pi\Vert$代表$\pi$的Frobenius范数:
\begin{equation*}
    \Vert \pi \Vert = \left(\sum_{m, \ell = 1}^N\sum_{i, j = 1}^n(\pi_i^m)^2\right)^{1/2}.
\end{equation*}

\subsubsection{充分条件}

先将Jacobi场的概念推广到高维情形: 设$L \in C^3$, $u^*$是一个极小点. 将$Q_{u^*}(\varphi)$看作是关于$\varphi$的变分积分, 写出其对应的E-L方程:

\begin{equation*}
    \boxed{J_{u^*}(\varphi) = \sum_{\ell = 1}^N\left(\sum_{i = 1}^n\partial_i\left(\sum_{j = 1}^na_{ij}^{m\ell}\partial_j\varphi^\ell + b_i^{m\ell}\varphi^\ell\right) - \left(\sum_{j = 1}^nb_j^{m\ell}\partial_j\varphi^m + c^{m\ell}\varphi^\ell\right)\right) = 0,}
\end{equation*}
其中$j = 1, \cdots, N$. 这是一个齐次二阶偏微分方程组. 称此方程为\textbf{Jacobi方程}, 并称$J_{u^*}$为沿$u^*$的\textbf{Jacobi算子}.
Jacobi方程的任一$C^2$解为沿$u^*$的\textbf{Jacobi场}.

以下探究$u^*$成为强极小点的充分条件:

\begin{proposition}[充分条件1]
    设$u^* \in M$满足E-L方程, 并且存在$\lambda > 0$使得 
    \begin{equation}\label{19}
        Q_{u^*}(\varphi) \geq \lambda\int_{\Omega}(|\varphi|^2 + |\nabla\varphi|^2) \,{\rm d}x, \quad \forall \varphi \in C_0^1(\Omega),
    \end{equation}
    则$u^*$是$I$的一个严格极小点.
    \begin{proof}
        与$n = 1$的情形一样.
    \end{proof}
\end{proposition}

对于1维的情形, 引入了共轭点的概念, 利用严格Legendre-Hadamard条件和Poincaré不等式来对条件\eqref{19}进行简化.
对于高维的情形, 由于共轭点的概念无法推广到高维, 故我们需要采取其它的手段. 以下我们旨在给出命题\ref{prop1.14}在高维情形的推广.

\begin{lemma}[G\r arding不等式]
    设$(a_{ij}^{m\ell}(x))$是$\overline{\Omega} \subseteq \mathbb{R}^n$上的一致连续函数, 且存在$\sigma > 0$使得 
    \begin{equation*}
        \sum_{m, \ell = 1}^N\sum_{i, j = 1}^na_{ij}^{m\ell}(x)\xi^m\xi^\ell\eta_i\eta_j \geq \sigma|\xi|^2|\eta|^2, \quad \forall (x, \xi, \eta) \in \Omega \times \mathbb{R}^N \times \mathbb{R}^n.
    \end{equation*}
    则存在$\beta, C_0 > 0$使得 
    \begin{equation*}
        \int_{\Omega}\sum_{m, \ell = 1}^N\sum_{i, j = 1}^na_{ij}^{m\ell}(x)\partial_i\varphi^m\partial_j\varphi^\ell \,{\rm d}x \geq \beta\int_{\Omega}|\nabla\varphi(x)|^2 - M\int_{\Omega}|\varphi(x)|^2 \,{\rm d}x, \quad \forall \varphi \in C_0^1(\Omega).
    \end{equation*}
    \begin{proof}
        我们分为三种情况进行证明. 

        当$N = 1$时, 结论是显然的.

        当$N > 1$且$(a_{ij}(x))$恒为常数使, 我们让$\varphi$在$\Omega$外定义为零, 使其在全空间上由定义, 从而我们可以考虑$\varphi$的Fourier变换:
        \begin{equation*}
            \widehat{\varphi}(\xi) = \int_{\mathbb{R}^n}\varphi(x)\mathrm{e}^{-2\pi\mathrm{i}x \cdot \xi} \,{\rm d}x. 
        \end{equation*}
        注意到等式$(\partial_i\varphi)^{\wedge}(\xi) = (2\pi\mathrm{i}\xi_i)\widehat{\varphi}(\xi), \forall i$, 从而由题设条件和Plancherel定理可得, 
        \begin{align*}
            \sum_{m, \ell = 1}^N\sum_{i, j = 1}^n\int_{\Omega}a_{ij}^{m\ell}\partial_i\varphi^m\partial_j\varphi^\ell \,{\rm d}x &\geq \sum_{m, \ell = 1}^N\sum_{i, j = 1}^n\int_{\Omega}a_{ij}^{m\ell}\widehat{\partial_i\varphi^m}\overline{\widehat{\partial_j\varphi^\ell}} \,{\rm d}\xi \\ 
            &= 4\pi^2 \sum_{m, \ell = 1}^N\sum_{i, j = 1}^n\int_{\Omega}a_{ij}^{m\ell}\xi_i\xi_j\partial_i\widehat{\varphi}^m\partial_j\widehat{\varphi}^\ell \,{\rm d}\xi \\ 
            &\geq 4\pi^2\sigma\int_{\mathbb{R}^n}|\xi\widehat{\varphi}(\xi)|^2 \,{\rm d}\xi \\  
            &= \sigma\int_{\mathbb{R}^n}|\widehat{\nabla\varphi}(\xi)|^2 \,{\rm d}\xi = \sigma\int_{\mathbb{R}^n}|\nabla\varphi(x)|^2 \,{\rm d}x. 
        \end{align*}

        最后我们考虑变系数的情形. 由$(a_{ij}(x))$的一致连续性可知, 对任意的$\varepsilon > 0$, 存在$\delta > 0$, 使得当开集$U \subseteq \overline{\Omega}$的直径${\rm diam}\ U < \delta$时, 有
        \begin{equation*}
            \sup_{x, y \in U}|a_{ij}^{m\ell}(x) - a_{ij}^{m\ell}(y)| < \varepsilon.
        \end{equation*}
        对任意的$x \in \overline{\Omega}$取球$B_{\delta/2}(x)$. 显然有$\bigcup_{x \in \overline{\Omega}}B_{\delta/2}(x) \supseteq \overline{\Omega}$.
        再由$\overline{\Omega}$的紧性可知, 存在$x_1, \cdots x_k$使得$\bigcup_{\alpha = 1}^kB_{\delta/2}(x_{\alpha}) \supseteq \overline{\Omega}$.
        令$B_{\alpha} = B_{\delta/2}(x_{\alpha}) \cap \Omega$, 从而有$\Omega = \bigcup_{\alpha = 1}^kB_{\alpha}$, 其中${\rm diam}\ B_{\alpha} < \delta, \forall \alpha$.
        我们现在取$\{B_{\alpha}\}_{\alpha = 1}^k$对应的一个单位分解, 即函数族$\{w_{\alpha}\}_{\alpha = 1}^k$满足如下条件:
        \begin{itemize}
            \item $w_{\alpha} \in C_c^{\infty}(B_{\alpha}), 0 \leq w_{\alpha} \leq 1, \forall \alpha$;
            \item $\sum_{\alpha = 1}^kw_{\alpha}^2 = 1$.
        \end{itemize}
        从而有 
        \begin{align*}
            \sum_{m, \ell = 1}^N\sum_{i, j = 1}^n\int_{\Omega}a_{ij}^{m\ell}(x)\partial_i\varphi^m\partial_j\varphi^\ell \,{\rm d}x &= \sum_{m, \ell = 1}^N\sum_{i, j = 1}^n\sum_{\alpha = 1}^k\int_{\Omega}a_{ij}^{m\ell}(x)w_{\alpha}^2(x)\partial_i\varphi^m\partial_j\varphi^\ell \,{\rm d}x \\ 
            &= S_1 + S_2,
        \end{align*}
        其中 
        \begin{align*}
            S_1 &=  \sum_{m, \ell = 1}^N\sum_{i, j = 1}^n\sum_{\alpha = 1}^k\int_{\Omega}a_{ij}^{m\ell}(x_{\alpha})w_{\alpha}^2(x)\partial_i\varphi^m\partial_j\varphi^\ell \,{\rm d}x, \\ 
            S_2 &=  \sum_{m, \ell = 1}^N\sum_{i, j = 1}^n\sum_{\alpha = 1}^k\int_{\Omega}(a_{ij}^{m\ell}(x) - a_{ij}^{m\ell}(x_{\alpha}))w_{\alpha}^2(x)\partial_i\varphi^m\partial_j\varphi^\ell \,{\rm d}x.
        \end{align*}
        一方面, 
        \begin{equation*}
            S_2 = o(1)\int_{\Omega}(|\varphi|^2 + |\nabla\varphi|^2) \,{\rm d}x \quad (\varepsilon \rightarrow 0).
        \end{equation*}
        另一方面, 由前述常系数的情形的分析, 有 
        \begin{align*}
            S_1 &= \sum_{m, \ell = 1}^N\sum_{i, j = 1}^n\sum_{\alpha = 1}^k\int_{\Omega}a_{ij}^{m\ell}(x_{\alpha})\partial_i(\varphi^mw_{\alpha})\partial_j(\varphi^\ell w_{\alpha}) \,{\rm d}x - S_3 \\ 
            &\geq \sigma\int_{\Omega}\sum_{\alpha = 1}^k|\nabla(w_{\alpha}\varphi)|^2 \,{\rm d}x + O(\Vert \nabla \varphi\Vert_{L^2}\Vert \varphi \Vert_{L^2} + \Vert \varphi \Vert_{L^2}^2) \\  
            &\geq \sigma'\int_{\Omega}|\nabla\varphi|^2 \,{\rm d}x - M\int_{\Omega}|\varphi|^2 \,{\rm d}x.
        \end{align*}
        综上所述, 我们有 
        \begin{equation*}
            \sum_{m, \ell = 1}^N\sum_{i, j = 1}^n\int_{\Omega}a_{ij}^{m\ell}(x)\partial_i\varphi^m\partial_j\varphi^\ell \,{\rm d}x \geq (\sigma'+ o(1))\int_{\Omega}|\nabla\varphi(x)|^2 - M\int_{\Omega}|\varphi(x)|^2  \qquad (\varepsilon \rightarrow 0).
        \end{equation*}
        这便完成了引理的证明.
    \end{proof}
\end{lemma}

\begin{proposition}[充分条件2]
    设$L \in C^2$满足严格Legendre-Hadamard条件. 若$u^*$是E-L方程的一个解, 且存在$\mu > 0$使得 
    \begin{equation*}
        Q_{u^*}(\varphi) \geq \mu\int_{\Omega}|\varphi|^2 \,{\rm d}x, \quad \forall \varphi \in C_0^1(\Omega),
    \end{equation*}
    则存在$\lambda > 0$使得 
    \begin{equation*}
        Q_{u^*}(\varphi) \geq \lambda\int_{\Omega}(|\varphi|^2 + |\nabla\varphi|^2) \,{\rm d}x, \quad \forall\varphi \in C_0^1(\Omega).
    \end{equation*}
    从而$u^*$是$I$的一个极小点.
    \begin{proof}
        利用G\r arding不等式, 参照命题\ref{prop1.14}的证明过程即可.
    \end{proof}
\end{proposition}

最后我们再从特征值的角度给出极小值点的刻画. 设$u^* \in C^1(\overline{\Omega})$是E-L方程的解. 称 
\begin{equation*}
    \boxed{\lambda_1 = \inf\left\{Q_{u^*}(\varphi)\colon \varphi \in C_0^1(\Omega), \int_{\Omega}|\varphi|^2 \,{\rm d}x = 1\right\}}
\end{equation*}
为Jacobi算子的\textbf{第一特征值}.

\begin{proposition}
    设$L \in C^2$满足严格Legendre-Hadamard条件, 又设$u^* \in M$是$I$的一个弱极小点, 则$\lambda_1 \geq 0$;
    反之, 若$\lambda_1 > 0$, 则$u^*$是$I$的一个严格弱极小点.
\end{proposition}

\subsection{约束极值问题}

\subsubsection{等式约束}

\begin{example}[等周问题]
    在平面上给定封闭曲线的弧长, 问什么样的曲线围成的面积最大? 为了问题的简单, 我们假设这条封闭曲线有参数表示
    \begin{equation*}
        r = r(t) = (x(t), y(t)),
    \end{equation*}
    其中$t \in [0, 2\pi]$. 此曲线围成的面积 
    \begin{equation*}
        S = S(r)= \frac{1}{2}\int x \,{\rm d}y - y \,{\rm d}x = \frac{1}{2}\int_0^{2\pi}(x\dot y - y\dot x) \,{\rm d}t, 
    \end{equation*}
    其长度为 
    \begin{equation*}
        L = \int_0^{2\pi}\sqrt{\dot x^2 + \dot y^2} \,{\rm d}t.
    \end{equation*}
    如果给定长度为$l$, 那么我们的问题就是在约束$L = l$之下, 求曲线$(x(t), y(t))$使得面积$S$达到极大值.
\end{example}

函数的条件极值问题: \textbf{Lagrange乘子法}. 设$\Omega \subseteq \mathbb{R}^n$是一个开集, 函数$f, g \in C^1(\overline{\Omega})$.
又设$g^{-1}(0) \neq \varnothing$且$\dot g(x) \neq 0$. 如果存在$x_0 \in \Omega$使得$f$在约束条件$g(x) = 0$下达到极小值, 那么便存在一个Lagrange乘子$\lambda \in \mathbb{R}$, 使得 
\begin{equation*}
    \nabla f(x_0) + \lambda\nabla g(x_0) = 0.
\end{equation*}
对于泛函的情形, 我们也有类似的Lagrange乘子法可用.

\begin{proposition}\label{prop1.60}
    给定$L, G \in C^2(\overline{\Omega} \times \mathbb{R}^N \times \mathbb{R}^{nN})$和$ \Phi \in C^1(\partial\Omega)$, 定义$M$上的泛函 
    \begin{equation*}
        I(u) = \int_{\Omega}L(x, u(x), \nabla u(x)) \,{\rm d}x 
    \end{equation*}
    和 
    \begin{equation*}
        N(u) = \int_{\Omega}G(x, u(x), \nabla u(x)) \,{\rm d}x. 
    \end{equation*}
    设$N^{-1}(0) \cap M \neq \varnothing$. 若$u^* \in M$是$I$在约束$N(u) = 0$下的极小点, 且存在$\varphi^* \in C_0^1(\Omega)$使得$\delta N(u^*, \varphi^*) \neq 0$, 则存在$\lambda \in \mathbb{R}$(此时$\lambda$也被称作\textbf{Lagrange乘子})使得
    \begin{equation*}
        \delta I(u^*, \varphi) + \lambda\delta(u^*, \varphi) = 0, \quad \forall \varphi \in C_0^1(\Omega).
    \end{equation*}
    等价地, 若记$Q = L + \lambda G$为调整后的Lagrange函数, 则$u^*$满足$Q$对应的E-L方程:
    \begin{equation*}
        \boxed{{\rm div}\ Q_p(x, u^*(x), \nabla u^*(x)) = Q_u(x, u^*(x), \nabla u^*(x)).}
    \end{equation*}
    \begin{proof}
        对任意的$\varphi \in C_0^1(\Omega)$, 考虑通过$u^*$, 并由$\varphi$和$\varphi^*$张成的平面:
        \begin{equation*}
            \pi = \{u^* + \varepsilon\varphi + \tau\varphi^*\colon (\varepsilon, \tau) \in \mathbb{R}^2\}
        \end{equation*}
        以及函数 
        \begin{align*}
            \Phi(\varepsilon, \tau) &= I(u^* + \varepsilon\varphi + \tau\varphi^*), \\   
            \Psi(\varepsilon, \tau) &= N(u^* + \varepsilon\varphi + \tau\varphi^*). 
        \end{align*}
        注意到$\Psi(0, 0) = N(u^*) = 0, \partial_{\tau}\Psi(0, 0) = \delta N(u^*, \varphi^*) \neq 0$, 故由隐函数定理可知, 当$r > 0$充分小时, 方程$\Psi(\varepsilon, \tau) = 0$在$B_r(0, 0)$上有唯一$C^1$解: $\tau = \tau(\varepsilon)$.
        由此表明$N^{-1}(0) \cap \pi \neq \varnothing$. 现令 
        \begin{equation*}
            g(\varepsilon) = \Phi(\varepsilon, \tau(\varepsilon)) = I(u^* + \varepsilon\varphi + \tau(\varepsilon)\varphi^*).
        \end{equation*}
        由题设条件可知, $u^*$是$I$在$N^{-1}(0) \cap M$上的极小点. 对应地, $0$是$g$的极小点, 从而有 
        \begin{align*}
            0 = \dot g(0) &= \partial_{\varepsilon}\Phi(0, 0) + \partial_{\tau}\Phi(0, 0)\dot\tau(0) \\ 
            &= \delta I(u^*, \varphi) + \delta I(u^*, \varphi^*)\left(-\frac{\delta N(u^*, \varphi)}{\delta N(u^*, \varphi^*)}\right) \\  
            &= \delta I(u^*, \varphi) + \lambda\delta N(u^*, \varphi), 
        \end{align*}
        其中 
        \begin{equation*}
            \lambda = -\frac{\delta I(u^*, \varphi_0)}{\delta N(u^*, \varphi_0)}
        \end{equation*}
        是一个常数. 这便得到了所需结论.
    \end{proof}
\end{proposition}

同理, 我们也可以考虑多个约束的泛函极值问题.

\begin{example}[等周问题-续]
    此时调整后的Lagrange函数为 
    \begin{equation*}
        Q = \frac{1}{2}(x\dot y - y\dot x) + \lambda \sqrt{\dot x^2 + \dot y^2}, 
    \end{equation*}
    其中$\lambda \in \mathbb{R}$. 对应的E-L方程为 
    \begin{equation*}
        \begin{cases} 
            \displaystyle\dot x = \lambda\frac{{\rm d}}{{\rm d}t}\frac{\dot y}{\sqrt{\dot x^2 + \dot y^2}}, \\  
            \displaystyle\dot y = \lambda\frac{{\rm d}}{{\rm d}t}\frac{\dot x}{\sqrt{\dot x^2 + \dot y^2}}. 
        \end{cases}
    \end{equation*}
    由此解出 
    \begin{equation*}
        \begin{cases} 
            \displaystyle x - c_1 = -\lambda\frac{\dot y}{\sqrt{\dot x^2 + \dot y^2}}, \\ 
            \displaystyle y - c_2 = \lambda\frac{\dot x}{\sqrt{\dot x^2 + \dot y^2}}, 
        \end{cases}
    \end{equation*}
    其中$c_1, c_2$是常数. 显然, 这是圆的方程:
    \begin{equation*}
        (x - c_1)^2 + (y - c_2)^2 = \lambda^2,
    \end{equation*}
    其半径$r = \lambda = l/2\pi$, 圆心为$(c_1, c_2)$.
\end{example}

上述我们考虑的是\textbf{积分形式的约束}, 以下我们考虑\textbf{等式约束}. 具体地, 给定函数$F \in C^1(\overline{\Omega} \times \mathbb{R}^N \times \mathbb{R}^{nN})$, 我们要在约束 
\begin{equation*}
    F(x, u(x), \nabla u(x)) = 0, \quad \forall x \in \Omega
\end{equation*}
下, 求泛函 
\begin{equation*}
    I(u) = \int_{\Omega} L(x, u(x), \nabla u(x)) \,{\rm d}x, \quad x \in M 
\end{equation*}
的极值. 事实上, 对于\textbf{完整(holonomic)约束}, 即$F$只依赖于$u$的情形, 我们也有类似的Lagrange乘子法可用.

\begin{proposition}\label{prop1.62}
    设$\Omega$是$\mathbb{R}^n$中的有界区域. 设$L \in C^2(\overline{\Omega} \times \mathbb{R}^N \times \mathbb{R}^{nN}), F \in C^2(\mathbb{R}^N)$.
    又设$u^* \in M$是在约束$F(u(x)) = 0$下的极小点, 并且$u^*$在有限个逐片$C^1$的$(n - 1)$维超曲面之外是$C^2$的.
    若对于任意的$x \in \overline{\Omega}, \nabla F(u^*(x)) \neq 0$, 那么存在$\lambda \in C(\overline{\Omega})$, 使得$u^*$满足对应于调整后的Lagrange函数$Q = L + \lambda F$的E-L方程:
    \begin{equation}\label{20}
        \boxed{L_{u_m} + \lambda M_{u_m} = \sum_{i = 1}^n\partial_{x_i}L_{p_i^m}, \quad 1 \leq m \leq N.}
    \end{equation}
    \begin{proof}
        同命题\ref{prop1.60}的证明思路类似, 我们先利用隐函数定理, 将有约束问题转化为无约束问题, 从而构造出局部定义的连续函数$\lambda$, 最后再将它们粘连起来, 成为一个整体定义的连续函数.

        任意固定$x_0 \in \Omega$, 我们选取充分小的$ r > 0$使得$\nabla_x F(u^*(x)) \neq 0, \forall x \in B_r(x_0) \subseteq \Omega$.
        注意到 
        \begin{equation*}
            \nabla_xF(u^*(x)) = \nabla_uF(u^*(x)) \cdot \nabla u^*(x),
        \end{equation*}
        因此$\nabla F(u^*) \neq 0, \forall u \in u^*(B_r(x_0))$. 不失一般性, 设$F_{u^N}(u^*) \neq 0, \forall u \in u^*(B_r(x_0))$.
        利用隐函数定理, 我们可以局部地解出$u^N\colon u^N = U(\widetilde{u})$, 其中$\widetilde{u} = (u^1, \cdots, u^{N - 1}), U \in C^2$.
        现令 
        \begin{equation*}
            \Lambda(x, \widetilde{u}, \widetilde{p}) = L(x, \widetilde{u}, U(\widetilde{u}), \widetilde{p}, p^N),
        \end{equation*}
        其中
        \begin{align*}
            \widetilde{p} &= (p_i^m)_{1 \leq i \leq n, 1 \leq m \leq N - 1}, \\ 
            p^N &= (p_i^N)_{1 \leq i \leq n} = \left(\sum_{m = 1}^{N -1}U_{u^m}p_i^m\right)_{1 \leq i \leq n}.
        \end{align*}
        由题设条件可知, 若$u^*$是在约束$F(u^*) = 0$下$I$的极小点, 则$\widetilde{u^*}$一定是
        \begin{equation*}
            J(\widetilde{u}) = \int_{B_r(x_0)}\Lambda(x, \widetilde{u}(x), \nabla\widetilde{u}(x)) \,{\rm d}x
        \end{equation*}
        的极小点. 后者对应的E-L方程为
        \begin{equation}\label{21}
            L_{u^m} + L_{u^N}U_{u^m} + \sum_{i = 1}^nL_{p_i^N}\partial_{u^m}p_i^N = \sum_{i = 1}^n\partial_{x_i}(L_{p_i^m} + L_{p_i^N}U_{u^m}), \quad \forall m = 1, \cdots, N - 1.
        \end{equation} 
        直接计算可得 
        \begin{equation*}
            \partial_{u^m}p_i^N = \sum_{\ell = 1}^{N - 1}\partial_{u^m}U_{u^{\ell}}p_i^\ell =\partial_{x_i}U_{u^m},
        \end{equation*}
        从而有  
        \begin{equation*}
            \sum_{i = 1}^n\partial_{x_i}(L_{p_i^m} + L_{p_i^N}U_{u^m}) = \sum_{i = 1}^n(\partial_{x_i}L_{p_i^m} + U_{u^m}\partial_{x_i}L_{p_i^N} +L_{p_i^N}\partial_{u^m}p_i^N).
        \end{equation*}
        因此\eqref{21}式化为 
        \begin{equation}\label{22}
            L_{u^m} + U_{u^m}\left(L_{u^N} - \sum_{i = 1}^n\partial_{x_i}L_{p_i^N}\right) = \sum_{i = 1}^n\partial_{x_i}L_{p_i^m}.
        \end{equation}
        注意到 
        \begin{equation*}
            U_{u^m} = -\frac{F_{u^m}}{F_{u^N}}, 
        \end{equation*}
        因此, 若在$B_r(x_0)$上定义 
        \begin{equation*}
            \lambda_{B_r(x_0)} = \frac{1}{F_{u^N}}({\rm div}\ L_{p^N} - L_{u^N}),
        \end{equation*}
        则\eqref{22}可写为 
        \begin{equation*}
            L_{u^m} + \lambda_{B_r(x_0)} F_{u^m} = {\rm div}\ L_{p^m}, \quad m = 1, \cdots, N - 1.            
        \end{equation*}
        这便是(在$B_r(x_0)$上)局部的E-L方程.

        最后, 由上述的构造可知, $\Omega = \bigcup_{x \in \Omega}B_r(x)$, 且每个球$B_r(x)$对应的函数$\lambda_{B_r(x)}$都是连续的.
        进一步地, 当两个这样的小球, 设为$B_{r_1}(x_1)$和$B_{r_2}(x_2)$, 相交非空时, 显然有
        \begin{equation*}
            \lambda_{B_{r_1}(x_1)} = \lambda_{B_{r_2}(x_2)}. 
        \end{equation*}
        因此, 由粘接引理可知, 存在$\lambda \in C(\Omega)$, 使得$\lambda|_{B_r(x)} = \lambda_{B_r(x)}, \forall x \in \Omega$, 且满足\eqref{20}式.
        注意到$u^*$在边界处的值是已知的, 故由\eqref{20}的具体表达式可知, 我们可以将$\lambda$延拓到$\overline{\Omega}$上, 使之成为$\overline{\Omega}$上的连续函数.
        显然延拓后的$\lambda$即为所求。
    \end{proof}
\end{proposition}

与积分形式的约束类似, 我们也可以考虑不止一个约束函数的情形. 与积分约束的情形相比, 这里的Lagrange乘子$\lambda$不是常数, 而是定义在$\overline{\Omega}$上的连续函数.

\begin{example}[球面上的测地线]
    我们以条件约束的观点来讨论球面上的测地线问题. 具体地, 设曲线有参数表示$u= u(t) = (x(t), y(t), z(t)), a \leq t \leq b$, 约束为 
    \begin{equation*}
        f(x, y, z) = x^2 + y^2 + z^2 = 1,
    \end{equation*}
    我们要在此约束下求解泛函 
    \begin{equation*}
        I(u) = \int_a^b \sqrt{\dot x^2 + \dot y^2 + \dot z^2} \,{\rm d}t
    \end{equation*}
    的极值. 令$u = (x, y, z), p = (\xi, \eta, \zeta)$. 引入Lagrange乘子$\lambda = \lambda(t)$, 从而得到调整后的Lagrange函数 
    \begin{equation*}
        Q = \sqrt{\xi^2 + \eta^2 + \zeta^2} + \lambda(x^2 + y^2 + z^2 - 1) = |p| + \lambda(|u|^2 - 1).
    \end{equation*}
    注意到约束$f$的表达式只依赖于$u$, 故由命题\eqref{prop1.62}可知, $\lambda$满足$Q$所对应的E-L方程:
    \begin{equation*}
        \frac{{\rm d}}{{\rm d}t}\frac{\dot u}{|\dot u|} = 2\lambda u,
    \end{equation*}
    其中$|u| = 1$. 进一步地, 我们有
    \begin{equation*}
        \frac{{\rm d}}{{\rm d}t}\left(\frac{\dot u}{|\dot u|} \times u\right) = \left(\frac{{\rm d}}{{\rm d}t}\frac{\dot u}{|\dot u|}\right) \times u + \frac{\dot u}{|\dot u|} \times \dot u = 0,
    \end{equation*}
    即$v= \frac{\dot u}{|\dot u|} \times u$是常向量. 由此可知, $u$必须位于垂直于常向量$v$且过原点的平面上.
    因此$u$必是大圆的一部分.
\end{example}

\begin{example}[到球面的调和映射]
    设$\mathbb{B}$是$\mathbb{R}^3$中的单位球, $\mathbb{S} = \partial\mathbb{B}$是单位球面, $u =(u_1, u_2, u_3)\colon \mathbb{B} \rightarrow \mathbb{S}$.
    若$u^*$是下列约束问题
    \begin{equation*}
        \min\{I(u)\colon M(u) = 0\},
    \end{equation*}
    的解, 其中  
    \begin{align*}
        I(u) &= \int_{\Omega}|\nabla u|^2 \,{\rm d}x = \sum_{m = 1}^3\sum_{i = 1}^3\int_{\Omega}|\partial_{x_i}u_m|^2 \,{\rm d}x, \\  
        M(u) &= |u|^2 - 1 = u_1^2 + u_2^2 + u_3^2 - 1,
    \end{align*}
    那么我们称$u^*$为从$\mathbb{B}$到$\mathbb{S}$的\textbf{调和映射}. 以下我们导出$u^*$满足的方程.
    首先写出调整后的Lagrange函数对应的E-L方程:
    \begin{equation}\label{23}
        -\Delta u = \lambda u.
    \end{equation} 
    其中$\lambda \in C(\mathbb{B}), \Delta u = (\Delta u_1, \Delta u_2, \Delta u_3)$.
    在等式$u \cdot u = 1$两边求导, 我们有
    \begin{equation*}
        u \cdot \partial_m u = 0,
    \end{equation*}
    其中$\partial_mu = (\partial_{x_m}u_1, \partial_{x_m}u_2, \partial_{x_m}u_3)$.
    再次求导, 并将得到的等式相加, 即得
    \begin{equation}\label{24}
        u \cdot \Delta u +|\nabla u|^2 =0.
    \end{equation}
    联立\eqref{23}和\eqref{24}, 得 
    \begin{equation*}
        \lambda = -u \cdot \Delta u = |\nabla u|^2.
    \end{equation*}
    由此我们导出了调和映射所满足的方程:
    \begin{equation*}
        -\Delta u = u|\nabla u|^2.
    \end{equation*}
\end{example}

\subsubsection{不等式约束}

设$\Omega$是$\mathbb{R}^n$上的有界区域, $M$为定义在$\overline{\Omega}$上的$C^1$函数的集合.
给定$M$的一个凸子集$C$和Lagrange函数$L$, 我们考虑如下约束问题:
\begin{equation*}
    \min\left\{I(u) = \int_{\Omega}L(x, u(x), \nabla (u(x))) \,{\rm d}x \colon u \in C\right\}. 
\end{equation*}
若$u^* \in M$是一个极小点, 那么对于任意的$v \in C$, 由于$C$是凸集, 则$tv + (1 - t)u \in C, \forall t \in [0, 1]$, 从而有 
\begin{equation*}
    I(tv + (1 - t)u) \geq I(u), \quad \forall t \in [0, 1]. 
\end{equation*}
由此可以推出 
\begin{equation*}
    \delta I(u, v - u) = \lim_{t \rightarrow 0^+}\frac{I(u + t(v - u)) - I(u)}{t} \geq 0,
\end{equation*}
即对任意的$v \in C$, 我们有 
\begin{equation*}
    \boxed{\int_{\Omega}(L_u(x, u(x), \nabla u(x))(v(x) - u(x)) + L_p(x, u(x), \nabla u(x)) \cdot (\nabla v(x) - \nabla u(x))) \,{\rm d}x \geq 0.}
\end{equation*}
称上述不等式为\textbf{变分不等式}.

\begin{example}[障碍问题]
    设$\Omega \subseteq \mathbb{R}^2$是一个有界区域. 给定函数$\varphi \in C^1(\partial\Omega), \psi \in C^1(\overline{\Omega}), f \in C(\overline{\Omega})$.
    在$\Omega$上我们考虑一张薄膜$u$, 它的边界固定: $u|_{\partial\Omega} = \varphi$, 并受外力$f$的作用, 但不能越过``障碍'', 即$u(x) \leq \psi(x), \forall x \in \overline{\Omega}$.
    具体地, 我们要寻求薄膜的平衡位置 
    \begin{equation*}
        u \in M = \{u \in PWC^1(\overline{\Omega})\colon u|_{\partial\Omega} = \varphi\}
    \end{equation*}
    在不等式约束 
    \begin{equation*}
        u(x) \leq \psi(x), \quad \forall x \in \overline{\Omega}, u \in M
    \end{equation*}
    下, 使薄膜的能量 
    \begin{equation*}
        I(u) = \int_{\Omega}\left(\frac{1}{2}|\nabla u(x)|^2 - f(x)u(x)\right) \,{\rm d}x
    \end{equation*}
    达到极小值. 注意到$C = \{u \in PWC^1(\overline{\Omega})\colon u|_{\partial\Omega} = \varphi, u(x) \leq \psi(x), \forall x \in \overline{\Omega}\}$是一个凸集, 故该变分问题对应的变分不等式为
    \begin{equation*}
        \int_{\Omega}(\nabla u\nabla(v - u) - f(v - u)) \,{\rm d}x \geq 0, \quad \forall v \in C.
    \end{equation*} 
\end{example}

\subsection{应用: Noether定理}

为了表达式的简洁, 本节我们在公式推导有时会使用Einstein求和约定.

\subsubsection{一阶变分的推广}

我们先对经典的一阶变分$\delta I(u^*, \varphi)$进行推广. 在前几节我们讨论的变分问题中, $M$中的函数均具有相同的定义域.
事实上, 这种限制不是必需的.

设$\Omega$为$\mathbb{R}^n$中的有界区域, 并给定$u \in C^1(\Omega)$和Lagrange函数$L \in C^2(\overline{\Omega} \times \mathbb{R}^N \times \mathbb{R}^{nN})$.
现引入一族单参数微分同胚$\eta_{\varepsilon} = \eta(\cdot, \varepsilon)\colon \Omega \rightarrow \Omega_{\varepsilon}$,  其中$\Omega_{\varepsilon} = \eta_{\varepsilon}(\Omega) \subseteq \mathbb{R}^n, \eta_0 = \eta( \cdot, 0) = {\rm id}$.
若设$\partial_{\varepsilon}\eta_{\varepsilon}(x)|_{\varepsilon = 0} = \bar{X}(x)$, 那么
\begin{equation}\label{25}
    \eta_{\varepsilon}(x) = x + \varepsilon\bar{X}(x) + o(\varepsilon) \qquad (\varepsilon \rightarrow 0).
\end{equation}
我们再考虑一族从$\Omega_{\varepsilon}$映到$\mathbb{R}^N$的函数$v_{\varepsilon} = v( \cdot, \varepsilon), |\varepsilon| < \varepsilon_0$, 其中$v_0 = u$.
设$\partial_{\varepsilon}\eta_{\varepsilon}(x)|_{\varepsilon = 0} = \bar{X}(x)$, 则有
\begin{equation*}
    v_{\varepsilon}(x) = u(x) + \varepsilon\varphi(x) + o(\varepsilon) \qquad (\varepsilon \rightarrow 0).
\end{equation*}
现记 
\begin{equation*}
    I = I(u, \Omega) = \int_{\Omega}L(x,u(x), \nabla u(x)) \,{\rm d}x,
\end{equation*}
我们考虑$(u_{\varepsilon}, \Omega_{\varepsilon})$在$I$上的取值. 即
\begin{align*}
    \Phi(\varepsilon) = I(v_{\varepsilon}, \Omega_{\varepsilon}) &= \int_{\Omega_{\varepsilon}}L(y, v_{\varepsilon}(y), \nabla v_{\varepsilon}(y)) \,{\rm d}y \\
    &= \int_{\Omega}L(\eta_{\varepsilon}(x), v_{\varepsilon}(\eta_{\varepsilon}(x)), \nabla_y v_{\varepsilon}(\eta_{\varepsilon}(x))) \det(\partial_{x_i}\eta_{\varepsilon}^j) \,{\rm d}x.
\end{align*} 
进一步地, 由表达式\eqref{25}和行列式求导法则可知, 
\begin{equation*}
    \det(\partial_i\eta_{\varepsilon}^j)|_{\varepsilon = 0} = 1, \quad \left.\frac{{\rm d}}{{\rm d}\varepsilon}\det(\partial_{x_i}\eta_{\varepsilon}^j)\right|_{\varepsilon = 0} = \partial_i\bar{X}^i = {\rm div}\ \bar{X},
\end{equation*}
从而当$u \in C^2$时, 我们有
\begin{align*}
    \dot\Phi(0) &= \int_{\Omega}(\partial_iL(\tau)\bar{X}^i(x) + L_{u^m}(\tau) \varphi^m(x) + L_{p^m_i}(\tau)\partial_i\varphi^m(x) + L(\tau){\rm div}\ \bar{X}(x)) \,{\rm d}x \\  
    &= \int_{\Omega}({\rm div}(L\bar{X})(\tau) + L_{u^m}(\tau) \varphi^m(x) - \partial_iL_{p_i^m}(\tau)\varphi^m(x) + {\rm div}(L_{p^m} \varphi^m)(\tau)) \,{\rm d}x,
\end{align*}
其中$\tau = (x, u(x), \nabla u(x))$. 化简得 
\begin{equation}\label{26}
    \boxed{\dot \Phi(0) = \int_{\Omega}(E_L(u)^m\varphi^m + {\rm div}(L \bar{X} + L_{p^m}\varphi^m)) \,{\rm d}x.}
\end{equation}
\eqref{26}式也可以看作是一种推广形式的一阶变分. 我们将其记为$\delta^*I(u; \varphi, \bar{X})$.

\begin{example}[Pohozaev恒等式]
    给定有光滑边界的有界区域$\Omega \subseteq \mathbb{R}^n$和函数$g \in C^1(\mathbb{R})$, 考虑下列非线性椭圆方程:
    \begin{equation}\label{27}
        \begin{cases} 
            -\Delta u = g(u) \quad &{\rm in}\ \Omega, \\  
            u = 0 \quad &{\rm on}\ \partial\Omega. 
        \end{cases} 
    \end{equation}
    则当$n \geq 3$时, 它的解满足下列\textbf{Pohozaev恒等式}:
    \begin{equation*}
        \frac{n - 2}{2}\int_{\Omega}|\nabla u|^2 \,{\rm d}x - n\int_{\Omega}G(u) \,{\rm d}x + \frac{1}{2}\int_{\partial\Omega}\left|\frac{\partial u}{\partial\nu}\right|^2(x \cdot \nu) \,{\rm d}S = 0, 
    \end{equation*}
    其中$G$是$g$的原函数, $G(0) = 0$; $\nu$是$\partial\Omega$的单位外法向量. 事实上, 取Lagrange函数 
    \begin{equation*}
        L = \frac{1}{2}p^2 - G(u)
    \end{equation*}
    以及$M = C_0^1(\Omega)$. 容易验证, $L$对应的E-L方程正是\eqref{27}. 不妨假设$0 \in \Omega$. 
    考虑一族单参数微分同胚$\eta_{\varepsilon}\colon \Omega \rightarrow \Omega_{\varepsilon}, x \mapsto (1 + \varepsilon)x$, 其中$\Omega_{\varepsilon} = \eta_{\varepsilon}(\Omega)$. 显然有$\eta_0 = {\rm id}$.
    现设$u \in M$是\eqref{27}的解. 令$v_{\varepsilon} = u \circ \eta_{\varepsilon}$. 显然有$v_0 = u$.
    通过直接计算可知, $\partial_{\varepsilon}\eta_{\varepsilon}|_{\varepsilon = 0} = 1, \partial_{\varepsilon}v_{\varepsilon}|_{\varepsilon = 0} = -x \cdot \nabla u$.
    利用$\delta^*I(u; \varphi, \bar{X})$的具体表达式, 我们有 
    \begin{equation*}
        \left.\frac{{\rm d}}{{\rm d}\varepsilon}I(v_{\varepsilon}, \Omega_{\varepsilon})\right|_{\varepsilon = 0} = \int_{\Omega}{\rm div}\left(\left(\frac{1}{2}|\nabla u|^2 - G(u)\right)x - \nabla u(x \cdot \nabla u)\right) \,{\rm d}x.
    \end{equation*}
    一方面, 注意到$u$满足方程\eqref{27}, 故等式右端可以化为
    \begin{equation}\label{28}
        \begin{aligned}
            \int_{\Omega}&\left(\frac{n}{2}|\nabla u|^2 - nG(u) + x \cdot \nabla\left(\frac{1}{2}|\nabla u|^2 -G(u)\right) - \Delta u(x \cdot \nabla u) - \nabla u \cdot \nabla(x \cdot \nabla u)\right) \,{\rm d}x \\ 
            &= \int_{\Omega}\left(\frac{n}{2}|\nabla u|^2 - nG(u) + \frac{x}{2} \cdot \nabla (|\nabla u|^2) - \frac{x}{2} \cdot \nabla (|\nabla u|^2) - |\nabla u|^2\right) \,{\rm d}x \\  
            &= \int_{\Omega}\left(\frac{n - 2}{2}|\nabla u|^2 - nG(u)\right) \,{\rm d}x.
        \end{aligned}
    \end{equation}
    这里我们用到了公式
    \begin{equation*}
        \nabla (f \cdot g) = f \times (\nabla \times g) + g \times (\nabla \times f) + (f \cdot \nabla)g + (g \cdot \nabla)f, 
    \end{equation*}
    其中$f, g$均为向量场. 另一方面, 注意到条件$G(0) = 0$且$u|_{\partial\Omega} = 0$, 故由散度定理可得 
    \begin{equation}\label{29}
        \begin{aligned}
            \frac{{\rm d}}{{\rm d}\varepsilon}I(v_{\varepsilon},\Omega_{\varepsilon})\Bigg|_{\varepsilon = 0} &= \int_{\partial\Omega}\left(\frac{1}{2}|\nabla u|^2(x \cdot \nu) - (\nabla u \cdot \nu)(x \cdot \nabla u)\right) \,{\rm d}S \\  
            &= \int_{\partial\Omega}\left(\frac{1}{2}\left|\frac{\partial u}{\partial\nu}\right|^2(x \cdot \nu) - (\nabla u \cdot \nabla u)(x \cdot \nu)\right) \,{\rm d}S \\ 
            &= -\frac{1}{2}\int_{\partial\Omega}\left|\frac{\partial u}{\partial\nu}\right|^2(x \cdot \nu) \,{\rm d}x,
        \end{aligned}
    \end{equation}
    其中$\nu$是$\partial\Omega$的单位外法向量. 联立\eqref{28}和\eqref{29}, 即得欲证恒等式.
\end{example}

\subsubsection{Noether定理}

粗略地说, Noether定理表明: 若变分积分$I$在某单参数变换群下保持不变, 则对于$I$的极值点$u^*$有某种守恒律成立.

在前一节的讨论中, 我们固定了$u$, 让$x$在一定范围内作形变. 现在我们考虑更一般的情况, 即在相空间$(x, u)$中作形变.
取$\mathbb{R}^n$中的有界区域$\Omega$. 给定向量场\footnote{这里特指微分流形中的概念.}
\begin{equation}\label{30}
    X = X^i(x, u)\frac{\partial}{\partial x_i} + U^m(x, u)\frac{\partial}{\partial u^m},
\end{equation}
我们可以找到一族定义在$\Omega \times \mathbb{R}^N$上的单参数变换群$\{\phi_{\varepsilon}\}_{|\varepsilon| < \varepsilon_0}$, 其中$\phi_0 = {\rm id}$.
进一步地, 若设$\phi_{\varepsilon}(x, u) = (Y(x, u, \varepsilon), W(x, u, \varepsilon))$, 则有 
\begin{equation*}
    \begin{cases} 
        X(x, u) = \partial_{\varepsilon}Y(x, u, \varepsilon)|_{\varepsilon = 0}, \\  
        U(x, u) = \partial_{\varepsilon}W(x, u, \varepsilon)|_{\varepsilon = 0}. 
    \end{cases}
\end{equation*}
现对任意的$u \in C^1(\overline{\Omega})$, 我们令
\begin{equation*}
    \begin{cases} 
        \eta(x, \varepsilon) = Y(x, u(x), \varepsilon), \\  
        \omega(x, \varepsilon) = W(x, u(x), \varepsilon). 
    \end{cases}
\end{equation*}
则有$\eta(x, 0) = x, \omega(x, 0) = u(x)$. 再令
\begin{equation}\label{31}
    \begin{cases} 
        \bar{X}(x) = \partial_{\varepsilon}\eta(x, \varepsilon)|_{\varepsilon = 0} = X(x, u(x)), \\  
        \bar{U}(x) = \partial_{\varepsilon}\omega(x, \varepsilon)|_{\varepsilon = 0} = U(x, u(x)), 
    \end{cases} 
\end{equation}
从而有 
\begin{equation*}
    \eta_{\varepsilon}(x) = \eta(x, \varepsilon) = x+ \varepsilon\bar{X}(x) + o(\varepsilon) \qquad (\varepsilon \rightarrow 0). 
\end{equation*}
若记$\Omega_{\varepsilon} = \eta_{\varepsilon}(\Omega)$, 则由上述表达式可知, $\eta_{\varepsilon}\colon \Omega \rightarrow \Omega_{\varepsilon}$是一族微分同胚, 其中$|\varepsilon| < \varepsilon_0, \varepsilon_0 > 0$充分小.
此外, 注意到$\eta_{\varepsilon}$有逆映射$\xi_{\varepsilon} = \eta_{-\varepsilon}$, 从而有
\begin{equation*}
    \xi_{\varepsilon}(x) = x - \varepsilon\bar{X}(x) + o(\varepsilon), \quad \varepsilon \rightarrow 0.
\end{equation*} 
由此我们可以导出一族从$\Omega_{\varepsilon}$映到$\mathbb{R}^N$的映射:
\begin{equation*}
    v_{\varepsilon}(x) = \omega(\xi_{\varepsilon}(x), \varepsilon). 
\end{equation*}
且有$v_0(y) = \omega(\xi_0(y), 0) = \omega(x, 0) = u(x)$. 若设$\partial_{\varepsilon}v(x, \varepsilon)|_{\varepsilon = 0} = \varphi(x)$, 则有  
\begin{equation*}
    \bar{U}(x) = \partial_{\varepsilon}v(\eta_{\varepsilon}(x), \varepsilon)|_{\varepsilon = 0} = \varphi(x) + \sum_{i = 1}^n\partial_iu(x)\bar{X}^i(x).
\end{equation*}
即 
\begin{equation}\label{32}
    \varphi = \bar{U} - \sum_{i = 1}^n\partial_iu\bar{X}^i. 
\end{equation}
这样一来, 结合\eqref{31}和\eqref{32}, 并将式子中出现的$\bar{X}$和$\varphi$代入至$\delta^*I(u; \varphi, \bar{X})$的表达式中, 我们便得到了在\textbf{一般的局部单参数变换群下所满足的恒等式}.
进一步地, 若变分积分$I(u_{\varepsilon}, \Omega_{\varepsilon})$与$\varepsilon$无关(此时我们也称$I$关于$\{\phi_{\varepsilon}\}$是不变的), 那么$\delta^*I(u; \varphi, \bar{X})$.
特别地, 对于任意的$x \in \Omega$,  我们取$\Omega = B_r(x)$, 其中$r > 0$充分小, 从而有
\begin{equation*}
    \frac{1}{|B_r(x)|}\int_{B_r(x)}(E_L(u)^m\varphi^m + {\rm div}(L\bar{X} + L_{p^m}\varphi^m)) \,{\rm d}x = 0.
\end{equation*}
在上述等式两边令$r \rightarrow 0$, 并注意到$x \in \Omega$的任意性, 我们便有
\begin{equation*}
    \boxed{E_L(u)^m\varphi^m + {\rm div}(L\bar{X} + L_{p^m}\varphi^m) = 0.}
\end{equation*}
称上述恒等式为\textbf{Noether恒等式}, 它便是守恒律的体现.

\begin{theorem}[Noether定理]
    设局部单参数变换群$\{\phi_{\varepsilon}\}$是由向量场\eqref{30}生成的, 又设泛函
    \begin{equation*}
        I(u) = \int_{\Omega}L(x, u(x), \nabla u(x)) \,{\rm d}x.
    \end{equation*}
    则对任意的$u \in C^2(\Omega)$, 有恒等式\eqref{26}成立, 其中
    \begin{equation*}
        \begin{cases} 
            \bar{X}(x) = X(x, u(x)), \\  
            \bar{U}(x) = U(x, u(x)), \\  
            \displaystyle \varphi(x) = \bar{U}(x)  - \sum_{i = 1}^n\partial_i\bar{X}^i(x).  
        \end{cases}
    \end{equation*}
    进一步地, 若$\{\phi_{\varepsilon}\}$关于$I$是不变的, 那么有如下的Noether恒等式成立L
    \begin{equation*}
        E_L(u)^m\varphi^m + {\rm div}(L\bar{X} + L_{p^m}\varphi^m) = 0.
    \end{equation*}
\end{theorem}

\begin{remark}
    当$u \in C^2$是$I$的弱极小点时, 由Noether恒等式可知, 微分形式 
    \begin{equation*}
        \omega = \sum_{i = 1}^n\left(L\bar{X}^i + \sum_{m = 1}^NL_{p_i^m}\left(\bar{U}^m - \sum_{j = 1}^n\partial_ju^m\bar{X}^j\right)\right) {\rm d}x_1 \wedge \cdots {\rm d}x_{i - 1} \wedge {\rm d}x_{i + 1} \wedge \cdots {\rm d}x_n
    \end{equation*}
    是闭的, 即${\rm d}\omega = 0$. 这也是守恒律的一种体现.
\end{remark}

我们还可以对Noether恒等式作进一步的简化. 引入\textbf{Hamilton能量动量张量}(简称\textbf{能动张量})$T(x, u, p) = (T_i^j(x, u, p))$, 其中
\begin{equation*}
    T_i^j = \sum_{m = 1}^Np_i^mL_{p_j^m} - \delta_{ij}L.
\end{equation*} 
若设$u \in C^2$是极小点, 使用该记号, Noether恒等式也可以写为
\begin{equation*}
    \boxed{\sum_{i = 1}^n\partial_i\left(\sum_{m = 1}^NL_{p_i^m}\bar{U}^m - \sum_{j = 1}^nT_j^i\bar{X}^j\right) = 0.}
\end{equation*}

\begin{example}
    考虑$k$个质点所构成的系统, 其质量分别为$m_1, \cdots, m_k$.
    位置坐标$X = (X_1, \cdots, X_k)$, 其中$X_i = (x_i, y_i, z_i)$是第$i$个质点的空间坐标.
    则系统的总动能为
    \begin{equation*}
        T = \frac{1}{2}\sum_{i = 1}^km_i|\dot X_i(t)|^2 = \frac{1}{2}\sum_{i = 1}^km_i(\dot x_i^2 + \dot y_i^2 + \dot z_i^2),
    \end{equation*}
    势能为 
    \begin{equation*}
        V = -k\sum_{i < j}\frac{m_im_j}{|X_i - X_j|} = -k\sum_{i < j}\frac{m_im_j}{((x_i - x_j)^2 + (y_i - y_j)^2 + (z_i - z_j)^2)^{1/2}}.
    \end{equation*}
    Lagrange函数$L = T - V$, 对应的变分积分为 
    \begin{equation*}
        I(X) = \int_{t_0}^{t_1}L(X(t), \dot X(t)).
    \end{equation*}
    \begin{itemize}
        \item \textbf{空间平移群}. 设$\{S_{\varepsilon}\}$是空间坐标依赖于参数$\varepsilon$的一族变换:
        \begin{equation*}
            \widetilde{t} = t, \quad \widetilde{x_i} = x_i + \varepsilon, \quad\widetilde{y_i} = y_i, \quad\widetilde{z_i} = z_i \qquad (1 \leq i \leq k).
        \end{equation*}
        注意到$L$在这组变换下有相同的表达式, 故$I$关于$\{S_{\varepsilon}\}$是不变的.
        此时对应的向量场$X = 0$, 而$ U = (e_1, \cdots, e_1)$, 其中$e_1 = (1, 0, 0)$. 
        此时由Noether恒等式可得
        \begin{equation*}
            \sum_{i = 1}^km_i\dot x_i = {\rm const}.
        \end{equation*}
        同理, 对$y, z$方向作平移, 我们也可以得到类似的等式. 由此可得 
        \begin{equation*}
            \sum_{i = 1}^km_i\dot X_i(t) = \sum_{i = 1}^km_i(\dot x_i(t), \dot y_i(t), \dot z_i(t)) = {\rm const}.
        \end{equation*}
        即\textbf{动量守恒}.
        \item \textbf{时间平移群}. 设$\{T_{\varepsilon}\}$是空间坐标依赖于参数$\varepsilon$的一族变换:
        \begin{equation*}
            \widetilde{t} = t + \varepsilon, \widetilde{X_i} = X_i \qquad (1 \leq i \leq k).
        \end{equation*}
        由于$I$与$t$无关, 故$I$关于$\{T_{\varepsilon}\}$是不变的. 此时$X = 1, U = 0$, 故由Noether恒等式得到
        \begin{equation*}
            H = pL_p - L = {\rm const}. 
        \end{equation*}
        这是\textbf{能量守恒}.
        \item \textbf{单参数转动群}. 设$\{R_{\varepsilon}\}$是时空坐标依赖于$\varepsilon$的一族变换:
        \begin{equation*}
            \widetilde{t} = t, \quad \widetilde{x_i} = x_i\cos\varepsilon + y_i\sin\varepsilon, \quad \widetilde{y_i} = -x_i\sin\varepsilon + y_i\cos\varepsilon, \quad \widetilde{z_i} = z_i \ (1 \leq i \leq k).
        \end{equation*}
        可以验证, $I$关于$\{R_{\varepsilon}\}$是不变的. 直接计算得
        \begin{equation*}
            X = 0, U = (Z_1, \cdots, Z_k), 
        \end{equation*}
        其中$Z_i= (y_i, -x_i, 0), 1 \leq i \leq k$. 因此, 由Noether恒等式可知
        \begin{equation*}
            \sum_{i = 1}^km_i(y_i\dot x_i - x_i\dot y_i) = {\rm const}. 
        \end{equation*}
        类似地, 对于平面$yOz$和$zOx$上的转动, 也有类似的等式. 从而有
        \begin{equation*}
            \sum_{i = 1}^km_i X_i \times \dot X_i = {\rm const}.
        \end{equation*}
        上述等式即代表\textbf{角动量守恒}.
    \end{itemize}
\end{example}

\subsubsection{内极小}

本节我们从另一个角度来探究泛函极值的必要条件.

\emph{Motivation:} Noether定理 $\leadsto$ 自变量$x$可以作``形变''. 
$u + \varepsilon\varphi$ $\leadsto$ E-L方程, etc; $u \circ \eta_{\varepsilon}$ $\leadsto$ ? 

具体地, 设$\eta_{\varepsilon}\colon \overline{\Omega} \rightarrow \overline{\Omega}$是一个微分自同胚:
\begin{equation*}
    y = \eta_{\varepsilon}(x) = x + \varepsilon\bar{X}(x) + o(\varepsilon),
\end{equation*}
其中$\bar{X}|_{\partial\Omega} = 0$. 对于给定的函数$u \in C^1(\overline{\Omega})$, 令$v_{\varepsilon} = u \circ \xi_{\varepsilon}$, 从而有 
\begin{align*}
    I(v_{\varepsilon}, \Omega) &= \int_{\Omega}L(y, v_{\varepsilon}(y), \nabla u(y)) \,{\rm d}x,\\  
    &= \int_{\Omega}(\eta_{\varepsilon}(x), u(x), \partial_y\xi_{\varepsilon}(y)\nabla u(x))\det(\partial_i\eta_{\varepsilon}^j) \,{\rm d}x. 
\end{align*}
因此, 对于任意的$u \in C^2$, 我们有
\begin{align*}
    \left.\frac{{\rm d}}{{\rm d}\varepsilon}I(v_{\varepsilon}, \Omega)\right|_{\varepsilon = 0} &= \int_{\Omega}(\partial_iL\bar{X}^i  -L_{p_j^m}\partial_j\bar{X}^i\partial_iu^m + L\partial_i\bar{X}^i) \,{\rm d}x \\
    &= \int_{\Omega}(\partial_iL -\partial_i(L) +\partial_j(L_{p_j^m}\partial_iu^m))\bar{X}^i \,{\rm d}x \\  
    &= -\int_{\Omega}E_L(u) \cdot \left(\partial_iu\bar{X}^i\right) \,{\rm d}x. 
\end{align*} 
上述推导过程中无非用到了分布积分公式和复合求导法则. 注意到在第三个表达式中, $\partial_iL$代表取值, 而$\partial_i(L)$代表求导.

\begin{definition}
    称$u \in C^1(\overline{\Omega})$为$I$的一个\textbf{内极小点}, 如果对于任意的$\bar{X} \in C_0^1(\Omega)$, $I$在$v_{\varepsilon}$的变换下满足
    \begin{equation*}
        \boxed{\left.\frac{{\rm d}}{{\rm d}\varepsilon}I(v_{\varepsilon}, \Omega)\right|_{\varepsilon = 0}  = 0.}
    \end{equation*} 
\end{definition}

从上述推导过程中我们可以看出, 若$u \in C^2$是一个内极小点, 则有 
\begin{equation*}
    \boxed{E_L(u) \cdot \partial_iu = 0, \qquad \forall 1 \leq i \leq n.}
\end{equation*}
这便是\textbf{内极小点的必要条件}. 此外, 我们还可以得出, $C^2$的内极小点一定是弱极小点.

\subsubsection{例}

\begin{example}
    设$L(t, u, p) = t^2(p^2 - u^6/3)$. 又设$\phi_{\varepsilon}\colon \mathbb{R} \times \mathbb{R}^N \rightarrow \mathbb{R} \times \mathbb{R}^N , (t, u) \mapsto(Y, W)$, 其中 
    \begin{equation*}
        Y(t, u, \varepsilon) = (1 + \varepsilon)t, \quad W(t, u, \varepsilon) = \frac{u}{\sqrt{1 + \varepsilon}}. 
    \end{equation*}
    求证:
    \begin{enumerate}
        \item $L$对应的变分积分 
        \begin{equation*}
            I(u) = \int_0^1L(t, u, p) \,{\rm d}t
        \end{equation*}
        是$\{\phi_{\varepsilon}\}$不变的.
        \item 若设$u$为$I$的E-L方程的解, 则有 
        \begin{equation*}
            \frac{t^3}{3}u^6 + t^3\dot u^2 + t^2u\dot u = {\rm const}.
        \end{equation*}
    \end{enumerate}
    \begin{proof}
        1. 由题设条件, 直接计算可得, 形变$\eta_{\varepsilon}(t) = Y(t, u(t), \varepsilon) = (1 + t)\varepsilon$, 其诱导的映射 
        \begin{equation*}
            v_{\varepsilon}(y) = W(\eta_{\varepsilon}^{-1}(y), u(\eta_{\varepsilon}^{-1}(y)), \varepsilon) = \frac{u\left(\frac{y}{1 + \varepsilon}\right)}{\sqrt{1 + \varepsilon}}.
        \end{equation*}
        从而有 
        \begin{align*}
            I(v_{\varepsilon}, \Omega_{\varepsilon}) &= \int_0^{1 + \varepsilon}y^2\left(\frac{\dot u\left(\frac{y}{1 + \varepsilon}\right)}{(1 + \varepsilon)^3} - \frac{1}{3}\frac{u\left(\frac{y}{1 + \varepsilon}\right)^6}{(1 + \varepsilon)^3}\right) \,{\rm d}y \\  
            &= \int_0^1t^2\left(\dot u(t) - \frac{1}{3}u(t)^6\right) \,{\rm d}t =I(u, \Omega).
        \end{align*}
        由此表明$I$关于$\{\phi_{\varepsilon}\}$是不变的.

        2. 直接计算得$\bar{X}(t) = \partial_{\varepsilon}\eta_{\varepsilon}(t)|_{\varepsilon = 0} = t$, 
        \begin{equation*}
            \varphi(t) = \partial_{\varepsilon}W(x, u(x), \varepsilon)|_{\varepsilon = 0} - \dot u \bar{X} = -\frac{1}{2}u(t) - t\dot u(t).
        \end{equation*}
        将上述计算结果代入至Noether恒等式中, 并注意到$u$是E-L方程的解, 我们有
        \begin{equation*}
            t^3\left(\dot u^2 - \frac{1}{3}u^6\right) - 2t^2\dot u\left(\frac{1}{2}u + t\dot u\right) = {\rm const}.
        \end{equation*} 
        化简即得欲证等式.
    \end{proof}
\end{example}


\section{直接方法}

\subsection{引言}

传统方法 $\leadsto$ E-L方程, etc. 但仍有许多缺陷:
\begin{itemize}
    \item E-L方程为ODE/PDE, 大多数情况下无法写出其解析解;
    \item 前述理论大多都是基于E-L方程的解存在, 且满足一定正则性的前提下的; 如$C^1, C^2, PWC^1$.
\end{itemize}

直接方法: 直接从定义出发, 找极小化序列, 并验证极小点的存在性.

\subsubsection{Dirichlet原理}

\begin{example}
    考虑Laplace方程
    \begin{equation}\label{33}
        \begin{cases}
            \Delta u = 0 \quad &{\rm in} \ \Omega, \\ 
            u|_{\partial\Omega} = \varphi \quad &{\rm on}\ \partial\Omega.
        \end{cases}
    \end{equation}
    其可以看作Dirichlet积分
    \begin{equation}\label{34}
        D(u) = \int_{\Omega}|\nabla u|^2 \,{\rm d}x
    \end{equation}
    在$M = \{u \in C^2(\overline{\Omega})\colon u|_{\partial\Omega} = \varphi\}$上的E-L方程.
\end{example}

证明Laplace方程边值问题\eqref{33}有解 $\leadsto$ Dirichlet积分\eqref{34}在$M$上存在极小值.
为说明极小点的存在性, Riemann使用了如下的\textbf{Dirichlet原理}:

\begin{itemize}
    \item 因为$D$有下界, 所以必有下确界. 现取一列函数$\{u_n\} \subseteq M$使得$D(u_n) \rightarrow \inf_{u \in M}D(u)$.
    因为$\{u_n\}$是有界的, 所以有收敛子列$u_{n_k} \rightarrow u^*$. 这个$u^*$就是要找的极小点: $D(u^*) = \inf_{u \in M}D(u)$.
\end{itemize}

1870年, Weierstrass举出了以下反例:

\begin{example}
    考虑下列极值问题:
    \begin{equation*}
        I(u) = \int_{-1}^1x^2\dot u^2 \,{\rm d}x, \qquad M = \{u \in C^1[-1, 1]\colon u(-1) = -1, u(1) = 1\}.
    \end{equation*}
    显然有$I \geq 0$. 另一方面, 对$\varepsilon > 0$取 
    \begin{equation*}
        u_{\varepsilon}(x) = \frac{\arctan \displaystyle\frac{x}{\varepsilon}}{\arctan \displaystyle\frac{1}{\varepsilon}}.
    \end{equation*}
    我们有 
    \begin{equation*}
        I(u_{\varepsilon}) = \frac{\varepsilon^2}{\arctan^2\displaystyle\frac{1}{\varepsilon}}\int_{-1}^1\frac{x^2}{(x^2 + \varepsilon^2)^2} \,{\rm d}x \rightarrow 0 \qquad (\varepsilon \rightarrow 0).
    \end{equation*}
    由此表明$\inf_MI = 0$. 但若存在$u^* \in M$使得$I(u^*) = 0$, 则由$I$的具体表达式可知, $\dot u = 0$, 从而有$u^* = {\rm const}$, 这与边值条件矛盾!
\end{example}

事实上, 有界集不一定是列紧集, 且若极小化序列存在收敛子列, 其极限点也未必是极小点.

\begin{itemize}
    \item 问题的提出: 给定一个拓扑空间$X$, 设函数$f\colon X \rightarrow \mathbb{R}$下方有界, 从而有下确界$m = \inf_{x \in X}f(x)$.
    取极小化序列$\{x_j\} \subseteq X$使得$f(x_j) \rightarrow m$. 在什么条件下$\{x_j\}$存在收敛子列收敛到极小值点?    
\end{itemize}

\textbf{Analysis}: 考虑下方水平集 
\begin{equation*}
    f_t = \{x \in X\colon f(x) \leq t\},
\end{equation*}
其中$t \in \mathbb{R}$. 若存在$t > m$使得$f_t$是列紧的, 那么由于当$N$充分大时有$\{x_j\colon j \geq N\} \subseteq f_t$, 故有子列$x_{k_j} \rightarrow x^* \in f_t \subseteq X$. 
进一步地, 若$f$在$x = x^*$处还满足条件 
\begin{equation}\label{35}
    f(x^*) \leq \liminf\limits_{k \rightarrow \infty}f(x_{j_k}),
\end{equation}
则有$f(x^*) = m = \inf_Mf$. 特别地, 若$f$是\textbf{序列下半连续的}, 即条件$x_n \rightarrow x$蕴含$f(x) \leq \liminf f(x_n)$, 则不等式\eqref{35}成立.

\begin{itemize}
    \item 下半连续性. 一般定义: 称$f$是下半连续的, 若对任意的$t \in \mathbb{R}$, 下方水平集$f_t$是闭的.
    特别地, 对于度量空间, 下半连续 $\Leftrightarrow$ 序列下半连续.
    \item 列紧性. 对于有限维赋范线性空间的情形, 闭 + 有界 $\Rightarrow$ 列紧. 因此, 对于下半连续函数$f$, 只要添加\textbf{强制性条件}:
    \begin{equation*}
        f(x) \rightarrow +\infty \qquad (\Vert x \Vert \rightarrow +\infty),
    \end{equation*}
    则可保证$f_t$是有界的. 对于无限维的情形, 有界闭集不一定是列紧的, 如单位球面$\mathbb{S} = \{\Vert x \Vert = 1\}$.
    \item 选取合适的空间. $C^1$不是合适的空间. 一方面, 从$D(u_n)$有界不一定能推出$\{u_n\}$有界; 另一方面, 即使点列是$C^1$有界的, 也未必存在$C^1$收敛的点列.
\end{itemize}

\subsubsection{泛函分析初步: 弱拓扑}

\begin{definition}\label{def2.3}
    设$X$是赋范线性空间. 称序列$\{x_n\}$在$X$中\textbf{弱收敛}到$x$(记作$x_n \rightharpoonup x$), 如果对任意的$f \in X^*$, 有$\langle f, x_n - x\rangle \rightarrow 0$, 其中$X^*$为$X$的对偶空间.
    称序列$\{f_n\}$在$X^*$中\textbf{弱\textsuperscript{*}收敛}到$f$(记作$f_n \rightharpoonup^* f$), 如果对任意的$x \in X$, 有$\langle f_n - f, x\rangle \rightarrow 0$.
\end{definition}

我们可以通过如下方式来定义弱拓扑和弱\textsuperscript{*}拓扑: 

\begin{itemize}
    \item 对任意的$f \in X^*$, 定义映射$\varphi_f\colon X \rightarrow \mathbb{R}\colon x \mapsto \langle f, x\rangle$.
    那么$X$上的弱拓扑即为使得映射族$\{\varphi_f\}_{f \in X^*}$均连续的最粗糙的拓扑; 类似地, 对任意的$x \in X$, 考虑映射$\phi_x\colon X^* \rightarrow \mathbb{R}\colon f \mapsto \langle f, x\rangle$.
    则$X^*$上的弱\textsuperscript{*}拓扑即为使得映射族$\{\phi_x\}_{x \in X}$均连续的最粗糙的拓扑.
    \item 容易验证, 在上述定义下, $X$(或$X^*$)上的弱收敛(或弱\textsuperscript{*}收敛)的具体表现形式即为定义\ref{def2.3}中所定义的的那样.
    \item 取$x_0 \in X$, 若设$U$为$x_0$的在弱拓扑意义下的邻域, 则存在$f_1, \cdots, f_k \in X^*$以及$\varepsilon > 0$, 使得 
    \begin{equation*}
        U = \{x \in X\colon |\langle f_j, x - x_0\rangle| < \varepsilon, \forall j = 1, \cdots, k\}.
    \end{equation*}
    类似地, 若设$V$为$f_0 \in X^*$在弱\textsuperscript{*}拓扑意义下的邻域, 则存在$x_1, \cdots, x_l \in X$以及$\delta > 0$, 使得 
    \begin{equation*}
        V = \{f \in X^*\colon |\langle f - f_0, x_m\rangle| < \delta, \forall m = 1, \cdots, l\}.
    \end{equation*}
\end{itemize}

\begin{remark}
    在$X^*$中既有弱收敛, 又有弱\textsuperscript{*}收敛的概念. 注意到连续嵌入$X \hookrightarrow  X^{**}$, 所以弱收敛蕴含着弱\textsuperscript{*}收敛.
    特别地, 若$X$是自反的, 则$X^*$中的弱收敛和弱\textsuperscript{*}收敛没有区别.
\end{remark}

\begin{example}\label{ex2.5}
    设$D = [0, 1]^N$是$\mathbb{R}^N$中的立方体. 设$\varphi \in L^p(D), 1 \leq p \leq \infty$.
    将$\varphi$作周期延拓, 并令$\varphi_n(x) = \varphi(nx), \forall n \in \mathbb{Z}_{\geq 0}$, 以及 
    \begin{equation*}
        \bar{\varphi} = \int_D\varphi(x) \,{\rm d}x,
    \end{equation*}
    则
    \begin{equation*}
        \varphi \rightharpoonup \bar{\varphi} \ \text{in}\ L^p(D) \qquad (1 \leq p < \infty),
    \end{equation*}
    以及 
    \begin{equation*}
        \varphi \rightharpoonup^* \bar{\varphi}\ \text{in}\ L^{\infty}(D).
    \end{equation*}
\end{example}

有界不一定蕴含列紧性, 但若设原空间是自反的, 则有界性蕴含弱列紧性. 

\begin{theorem}[Banach-Alaoglu]\label{Alaoglu}
    设$X$是Banach空间, 则$X^*$中的闭单位球
    \begin{equation*}
        \mathbb{B}_{X^*} = \{f \in X^*\colon \Vert f \Vert \leq 1\}
    \end{equation*}
    是弱\textsuperscript{*}紧的.
\end{theorem}

\begin{proposition}
    设$X$是Banach空间. 则$X^*$是可分的, 当且仅当$\mathbb{B}_X$在弱拓扑下是可度量化的.
    对偶地, $X$是可分的, 当且仅当$\mathbb{B}_{X^*}$在弱\textsuperscript{*}拓扑下是可度量化的.
\end{proposition}

由上述结果可知, 若设$X$是可分的Banach空间, 此时由于$\mathbb{B}_{X^*} \subseteq X^*$是可度量化的, 故$\mathbb{B}_{X^*}$是列紧的, 即$\mathbb{B}_{X^*}$中的任意点列均有弱\textsuperscript{*}收敛子列.
特别地, 若$X$还是自反的, 注意到$X$可以看作是$X^*$的对偶空间, 故$X$中的闭单位球$\mathbb{B}_X$也是列紧的. 
因此, \textbf{若$X$是自反且可分的Banach空间, 则它的任意有界序列必有一弱收敛子序列}.

事实上, 上述论断中对$X$的可分性假设可以去掉. 我们先利用Banach-Alaoglu证明以下结论:

\begin{theorem}\label{Kakutani}
    设$X$是自反的Banach空间, 那么$\mathbb{B}_X$是弱紧的. 
    \begin{proof}
        设$J$为$X$到$X^{**}$的典范映射. 由于$X$是自反的, 故$\mathbb{B}_{X^{**}} = J(\mathbb{B}_X)$.
        再根据定理\ref{Alaoglu}可知, $\mathbb{B}_{X^{**}}$是弱\textsuperscript{*}紧的, 因此我们只需证明$J^{-1}$是从$X^{**}$(赋予弱\textsuperscript{*}拓扑)到$X$(赋予弱拓扑)的连续映射.
        事实上, 注意到对任意的$f \in X^*$和$\xi \in X^{**}$, 有 
        \begin{equation*}
            \langle f, J^{-1}\xi\rangle = \langle \xi, f\rangle,
        \end{equation*}
        由此即可说明$J^{-1}$是连续的. 这便证得了所需结论.
    \end{proof}
\end{theorem}

综上所述, 我们便有如下结果:

\begin{corollary}\label{coro2.9}
    自反Banach空间$X$中的任意有界序列必有一弱收敛的子序列.
    \begin{proof}
        设$\{x_n\}$为$X$中任意有界序列. 令
        \begin{equation*}
            M_0 = {\rm span} \{x_n\} = \left\{y \in X\colon y = \sum_Ja_ix_i, a_i \in \mathbb{R}, \# J < \infty.\right\}
        \end{equation*}
        由题设条件可知, $M = \overline{M_0}$是自反的. 进一步地, 若记$M'$为$\{x_n\}$在$\mathbb{Q}$上张成的线性子空间, 则显然$M'$在$M_0$稠密, 由此表明$M$还是可分的.
        故$M^*$也是可分, 且自反的. 若记$\mathbb{B}_M$为$M$中的闭单位球, 由上述分析可知, $\mathbb{B}_M$是可度量化的.
        此外, 由定理\ref{Kakutani}可知, $\mathbb{B}_M$还是弱紧的, 故$\mathbb{B}_M$是弱列紧的.
        由此足以说明$\{x_n\}$存在弱收敛子序列.
    \end{proof}
\end{corollary}

特别地, 任一Hilbert空间和$L^p \ (1 < p < \infty)$空间均是自反的; 在这些空间上, 由推论\ref{coro2.9}可知, 有界性蕴含着弱列紧性.

\subsubsection{Dirichlet原理-续}

以下是直接方法的基本定理: 

\begin{theorem}\label{th2.10}
    设$X$是自反的Banach空间, $E \subseteq X$是弱序列闭的非空子集.
    若$f \colon E \rightarrow \mathbb{R}$是弱序列下半连续, 并且是强制的, 则$f$在$E$上达到极小值.
    \begin{proof}
        取极小化序列$\{x_n\}\colon f(x_n) \rightarrow \inf_Ef$. 由于$f$是强制的, 则$\{x_n\}$是有界的.
        根据推论\ref{coro2.9}, $\{x_n\}$有弱\textsuperscript{*}收敛子列: $x_{k_n} \rightharpoonup^* x^*$. 再由题设条件可知, $x^* \in E$.
        注意到$f$是弱序列下半连续的, 从而有 
        \begin{equation*}
            f(x^*) \leq \liminf\limits_{n \rightarrow \infty}f(x_{k_n}) \leq \inf_Ef.
        \end{equation*}
        由此表明$f(x^*) = \inf_Ef$.
    \end{proof}
\end{theorem}

若要运用上述定理解决变分问题, 我们需要注意如下几点:

\begin{itemize}
    \item 选取合适的函数空间. 我们通常选取自反的Banach空间, 或更一般地, 可分Banach空间的对偶空间.
    \item 对应的泛函对于此函数空间上的拓扑是弱序列下半连续, 且是强制的.
\end{itemize}

以下我们使用定理\ref{th2.10}来验证Dirichlet原理.

首先是选取合适的空间. 注意到$C^1(\overline{\Omega})$不是自反的, 且Dirichlet积分$D(u)$按$C^1$拓扑也不是强制的, 故我们需要选取其它的空间.
注意到$D(u)$的具体表达式, 我们在$C^1(\overline{\Omega})$上引入如下范数 
\begin{equation*}
    \Vert u \Vert_{H^1} := \left(\int_{\Omega}(|\nabla u|^2 + |u|^2) \,{\rm d}x\right)^{1/2}.
\end{equation*}
$C^1(\overline{\Omega})$在此范数下不是完备的. 我们将其完备化空间记为$H^1(\Omega)$.
可以验证, 这是一个Hilbert空间, 其上的内积定义为 
\begin{equation*}
    (u, v)_{H^1} := \int_{\Omega}(\nabla u \cdot \nabla v + uv) \,{\rm d}x. 
\end{equation*}
对应地, 我们把$C_0^1(\Omega)$在$H^1(\Omega)$中的闭包记为$H_0^1(\Omega)$. 注意到$D(u)$不是$H^1(\Omega)$上的范数, 然而, 下述引理表明, 当$\Omega$是一个有界区域时, $D(u)$可以看作是$H_0^1(\Omega)$上的范数:

\begin{lemma}[Poincaré不等式]
    设$\Omega \subseteq \mathbb{R}^n$是一个有界区域, 则对任意的$u \in C_0^1(\Omega)$, 存在常数$C = C(\Omega) > 0$, 使得 
    \begin{equation*}
        \Vert u \Vert_{L^2} \leq C\Vert \nabla u \Vert_{L^2}.
    \end{equation*}
    \begin{proof}
        取$\mathbb{R}^n$中的立方体$D = \times_{i = 1}^n[a_i, b_i] (a_i, b_i \in \mathbb{R})$, 使得$D \supseteq \Omega$. 对任意的$\varphi \in C_0^1(\Omega)$, 令 
        \begin{equation*}
            \widetilde{\varphi}(x) = \varphi(x)\chi_{\Omega}(x), \qquad x \in D.
        \end{equation*}
        记$x = (x_1, \widetilde{x})$. 由引理\ref{lma1.12}可知, 存在只依赖于$\Omega$的常数$C_1$, 使得
        \begin{equation*}
            \int_J |\widetilde{\varphi}(x_1, \widetilde{x})| \,{\rm d}x_1 \leq C_1\int_J|\partial_{x_1}\widetilde{\varphi}(x_1, \widetilde{x})|^2 \,{\rm d}x_1,
        \end{equation*}
        其中$J$是$D$沿$x_1$方向的投影. 在上述不等式两侧依次对$x_2, x_3, \cdots, x_n$积分, 并重复使用引理\ref{lma1.12}, 注意到$\widetilde{\varphi}$的具体表达式, 即证得所需结论.
    \end{proof}
\end{lemma}

以下我们在$H^1(\Omega)$上验证Dirichlet原理的正确性.
对给定的$\varphi = C^1(\overline{\Omega})$, 在$M = \varphi + H_0^1(\Omega)$上, 其Dirichlet积分有表达式 
\begin{equation*}
    D(u) = D(\varphi) + D(v) + 2\int_{\Omega}\nabla u_0 \cdot \nabla \varphi \,{\rm d}x.
\end{equation*}
其中$u = \varphi + v \in M$.
\begin{itemize}
    \item $D(u)$是强制的. 由上述表达式可以看出, 只需说明当$D(v) \rightarrow +\infty$时, 有$D(u) \rightarrow +\infty$.
    事实上, 由Schwarz不等式和加权的初等不等式可得 
    \begin{equation*}
        \left|\int_{\Omega}\nabla v \cdot \nabla \varphi \,{\rm d}x\right| \leq \sqrt{D(v)D(\varphi)} \leq D(v) + \frac{1}{4}D(\varphi),
    \end{equation*}
    从而有 
    \begin{equation*}
        D(u) \geq \frac{1}{2}D(v) - D(\varphi).
    \end{equation*}
    由此即表明$D$是强制的.
    \item $D(u)$是弱序列下半连续的. 在$H^1(\Omega)$中取序列$\{u_n\}$, 其中$u_n = \varphi + v_n, v_n \in H_0^1(\Omega)$. 
    显然, 
    \begin{equation*}
        u_n \rightharpoonup u \ \text{in}\ H^1 \Leftrightarrow v_n \rightharpoonup v \ \text{in}\ H_0^1,
    \end{equation*}
    这里$u = \varphi + v$. 注意到
    \begin{equation*}
        \psi \mapsto \int_{\Omega}\nabla \psi \cdot \nabla \varphi \,{\rm d}x
    \end{equation*}
    可以看作是$H_0^1(\Omega)$的连续线性泛函, 由弱收敛的定义以及Riesz表示定理可知, 
    \begin{equation*}
        \int_{\Omega}\nabla v_n \cdot \nabla \varphi \,{\rm d}x \rightarrow \int_{\Omega}\nabla v \cdot \nabla \varphi \,{\rm d}x.
    \end{equation*}
    同理有 
    \begin{equation*}
        \int_{\Omega}\nabla v_n \cdot \nabla v \,{\rm d}x \rightarrow \int_{\Omega}\nabla v \cdot \nabla v \,{\rm d}x = D(v).
    \end{equation*}
    此时Schwarz不等式给出 
    \begin{equation*}
        D(v) = \lim\limits_{n \rightarrow \infty}\int_{\Omega}\nabla v_n \cdot \nabla v \,{\rm d}x \leq \liminf\limits_{n \rightarrow \infty}D(v)^{1/2}D(v_n)^{1/2}, 
    \end{equation*}
    即$D(v) \leq \liminf D(v_n)$. 由此足以说明$D(u)$的弱序列下半连续性.
\end{itemize}

综上所述, 我们利用定理\ref{th2.10}证明了Dirichlet积分可以在$H^1(\Omega)$上达到极小值, 即验证了Dirichlet原理.

最后指出, 一个微分方程的解并不能总是用直接方法求得. 以下反例属于Hadamard: 令 
\begin{equation*}
    u(r, \theta) = \sum_{m = 1}^{\infty}\frac{r^{m!}\sin m!\theta}{m^2}, \qquad (r, \theta) \in [0, 1] \times [0, 2\pi].
\end{equation*}
可以验证, $u$是单位圆内的调和函数, 但由于$\Vert \nabla u\Vert_{L^2} = \infty$, 其对应的Dirichlet积分$D(u) = \infty$.

\subsection{Sobolev空间初步}

从上一节的分析中可以看出, 若要使用定理\ref{th2.10}来验证极小点的存在性, 选取的函数空间应该是某个Banach空间的对偶空间(特别地, 自反空间), 这要求该空间至少是完备的;
此外, 由于泛函是含导数的变分积分, 则此空间应至少包含函数的一阶导数(或更弱意义下的导数)信息. 以下介绍的Sobolev空间便满足这些要求.

\subsubsection{基本定义和性质}

\emph{Motivation}: 分部积分公式. 若$u \in C^k(\Omega)$, 则对任意的$\alpha \in \mathbb{Z}^n_{\geq 0}\colon |\alpha| \leq k$和$\varphi \in C_0^{\infty}(\Omega)$, 我们有
\begin{equation*}
    \int_{\Omega}\partial^{\alpha}u\varphi \,{\rm d}x = (-1)^{|\alpha|}\int_{\Omega}u\partial^{\alpha}\varphi \,{\rm d}x.
\end{equation*}

\begin{definition}
    设$u, v \in L^1_{{\rm loc}}(\Omega)$, $\alpha \in \mathbb{Z}^n_{\geq 0}$.
    称$v$为$u$的$\alpha$阶\textbf{广义导数}, 如果对于任意的$\varphi \in C_0^{\infty}(\Omega)$, 有
    \begin{equation*}
        \boxed{\int_{\Omega}v\varphi \,{\rm d}x = (-1)^{|\alpha|}\int_{\Omega}u\partial^{\alpha}\varphi \,{\rm d}x.}
    \end{equation*}
    记作$v = D^{\alpha}u$.
\end{definition}

\begin{definition}
    设$p \in [1, \infty]$, $k \in \mathbb{Z}_{\geq 0}$. 定义 
    \begin{equation*}
        \boxed{W^{k, p}(\Omega) := \{u \in L^p(\Omega)\colon D^{\alpha}u \in L^p(\Omega), \forall \alpha\in \mathbb{Z}_{\geq 0}\colon |\alpha| \leq k\}.}
    \end{equation*}
    并规定其上的范数为 
    \begin{equation*}
        \Vert u \Vert_{k, p} := 
        \begin{cases}
            \displaystyle\left(\sum_{|\alpha| \leq k}\int_{\Omega}|D^{\alpha}u|^p \,{\rm d}x\right)^{1/p} \quad &1 \leq p < \infty, \\ 
            \displaystyle\mathop{{\rm esssup}}\limits_{\Omega}\sum_{|\alpha| \leq k}|D^{\alpha}u| \quad &p = \infty.
        \end{cases}
    \end{equation*}
    容易验证, $(W^{k, p}(\Omega), \Vert \cdot \Vert_{k, p})$是赋范线性空间, 且是完备的.
    称此空间为\textbf{Sobolev空间}.
\end{definition}

\begin{remark}
    我们有时也用下式定义$W^{k, p}(\Omega)$上的范数:
    \begin{equation*}
        \Vert u \Vert_{k, p}' := 
        \begin{cases}
            \displaystyle\sum_{|\alpha| \leq k}\left(\int_{\Omega}|D^{\alpha}u|^p \,{\rm d}x\right)^{1/p} = \sum_{|\alpha| \leq m}\Vert D^{\alpha}f\Vert_{L^p} \quad &1 \leq p < \infty, \\ 
            \displaystyle\sum_{|\alpha| \leq k}\mathop{{\rm esssup}}\limits_{\Omega}|D^{\alpha}u| = \sum_{|\alpha| \leq m}\Vert D^{\alpha}f\Vert_{L^{\infty}} \quad &p = \infty.
        \end{cases}
    \end{equation*}
    可以验证, $\Vert \cdot \Vert_{k, p}$和$\Vert \cdot \Vert_{k, p}'$是等价的.
\end{remark}

\begin{proposition}
    Sobolev空间$W^{k, p}(\Omega)$具有如下简单性质:
    \begin{enumerate}
        \item 对于有界区域, 有 
        \begin{gather*}
            W^{k, \infty}(\Omega) \subseteq W^{k, q}(\Omega) \subseteq W^{k, p}(\Omega) \subseteq W^{k, 1}(\Omega), \qquad 1 \leq p \leq q \leq \infty, \\ 
            W^{m, p}(\Omega) \subseteq W^{l, p}(\Omega), \qquad 0 \leq l \leq m.
        \end{gather*}
        \item 若$\Omega_1 \subseteq \Omega_2$, 则对任意的$u \in W^{k, p}(\Omega_2)$, 有$u|_{\Omega_1} \in W^{k, p}(\Omega_1)$.
        \item 设$u \in W^{k, p}(\Omega), \psi \in C_0^{\infty}(\Omega)$, 则有$\psi u \in W_0^{k, p}(\Omega)$, 并且对任意的$\alpha \in \mathbb{Z}_{\geq 0}\colon |\alpha| \leq k$, 有${\rm supp}(D^{\alpha}(\psi u)) \subseteq {\rm supp}\ \psi$, 
        \begin{equation*}
            D^{\alpha}(\psi u) = \sum_{\beta \leq \alpha}\binom{\alpha}{\beta}D^{\beta}\psi D^{\alpha - \beta}u,
        \end{equation*} 
        其中$\binom{\alpha}{\beta} = \frac{\alpha!}{\beta!(\alpha - \beta)!}$.
        \item $W_0^{k, p}(\mathbb{R}^n) = W^{k, p}(\mathbb{R}^n)$.
    \end{enumerate}
    \begin{proof}
        1和2是显然的. 对于3, 先利用数学归纳法归结于$k = 1$的情形, 再使用广义导数的定义和分部积分公式即可.
        
        4. 显然有$W_0^{k, p}(\mathbb{R}^n) \subseteq W^{k, p}(\mathbb{R}^n)$.
        另一方面, 注意到$C_0^{\infty}(\mathbb{R}^n)$在$L^p(\mathbb{R}^n)$中稠密, 且$W_0^{k, p}(\mathbb{R}^n)$是闭的, 故反包含关系也成立.
    \end{proof}
\end{proposition}

\begin{example}
    设$J = [a, b] \subseteq \mathbb{R}$, 则$W^{1, 1}(J) = {\rm AC(J)}$, 并且
    \begin{equation*}
        Du(x) = \dot u(x), \qquad a.e. \ x \in J.
    \end{equation*}

    事实上, 对于任意的$u \in {\rm AC(J)}$, 则其导数$\dot u$几乎处处存在, 并且属于$L^1(J)$, 因此$u \in W^{1, 1}(J)$.
    此外, 注意到对任意的$x, y \in J$, 我们有 
    \begin{equation*}
        u(x) = \int_y^x\dot u(t) \,{\rm d}t + u(y),
    \end{equation*}
    从而对任意的$\varphi \in C_0^{\infty}(J)$, 有
    \begin{equation*}
        \int_Ju(x)\dot \varphi(x) \,{\rm d}x = -\int_J\dot u(x)\varphi(x) \,{\rm d}x,
    \end{equation*}
    此即$Du = \dot u$, a.e.

    另一方面, 对任意的$u \in W^{1, 1}(J)$, 令 
    \begin{equation*}
        \varphi_n(t) = 
        \begin{cases}
            \displaystyle n(t - a) \quad &t \in \left[a, a + \frac{1}{n}\right], \\ 
            \displaystyle 1 \quad &t \in \left[a + \frac{1}{n}, x - \frac{1}{n}\right], \\ 
            \displaystyle -n(x - t) \quad &t \in \left[x - \frac{1}{n}, x\right], \\ 
            0 \quad &t \in [x, b]
        \end{cases}
        \qquad (n = 1, 2, \cdots),
    \end{equation*}
    再将$\varphi_n$磨光, 即对任意$\varphi_n$, 取$\xi_{n, k} \in C_0^{\infty}(J), \Vert \xi_{n, k} \Vert_{C^1} \leq 2n$, 使得$\xi_{n, k}$在$J$上一致收敛到$\varphi_n$, 且$\dot\xi_{n, k} \rightarrow \dot \varphi_n$, a.e. $t \in J$.
    在等式 
    \begin{equation*}
        \int_Ju(t)\dot\xi_{n, k}(t) \,{\rm d}t = -\int_JDu(t)\varphi_n(t) \,{\rm d}t
    \end{equation*}
    两端先令$k \rightarrow \infty$, 即得 
    \begin{equation*}
        \int_Ju(t)\dot\varphi_n(t) \,{\rm d}t = -\int_JDu(t)\varphi_n(t).
    \end{equation*}
    再令$n \rightharpoonup \infty$, 有 
    \begin{equation*}
        u(x) - u(a) = \int_a^xDu(t) \,{\rm d}t, \qquad \forall x \in J.
    \end{equation*}
    这表明$u \in {\rm AC(J)}$, 并且$\dot u = Du$, a.e. $x \in J$.

    此外, 利用绝对连续函数的Newton-Leibniz公式, 我们还可以证明$W^{1, \infty}(J) = {\rm Lip}(J)$.
\end{example}

最后我们来探究$W^{k, p}(\Omega)$的对偶空间具有何种形式. 这对于研究其上的弱收敛是有帮助的.
考虑如下映射:
\begin{equation*}
    i\colon W^{k, p}(\Omega) \rightarrow \mathop{\times}\limits_{|\alpha| \leq m}L^p(\Omega), u \mapsto (D^{\alpha}u)_{|\alpha| \leq m},
\end{equation*}
其中乘积空间$\times_{|\alpha| \leq k}L^p(\Omega)$上的范数规定为
\begin{equation*}
    \Vert u \Vert = \sum_{|\alpha| \leq k}\Vert D^{\alpha}u \Vert_{L^p}.
\end{equation*}
容易验证, 在此范数下, $\times_{|\alpha| \leq k}L^p(\Omega)$是Banach空间, 其对偶空间$(\times_{|\alpha| \leq k}L^p(\Omega))^* \cong \times_{|\alpha| \leq k}L^{p'}(\Omega)$.
这里的同构具体表现为: 规定$(\times_{|\alpha| \leq k}L^p(\Omega))^*$中的元素均具有形式$f = (f_{\alpha})_{|\alpha| \leq k}$, 且 
\begin{equation*}
    \langle f, u\rangle = \sum_{|\alpha| \leq k}\langle f_{\alpha}, D^{\alpha}u\rangle.
\end{equation*} 
那么有$f_{\alpha} \in L^{p'}(\Omega)$, 且$\Vert f \Vert = \max_{|\alpha| \leq k}\Vert f_{\alpha} \Vert$.

由上述分析可知, $i$是等距嵌入映射. 再根据Hahn-Banach定理和Riesz表示定理, 则有: $f \in (W^{k, p}(\Omega))^*$, 当且仅当存在$(\psi_{\alpha})_{|\alpha| \leq k} \in \times_{|\alpha| \leq k}L^{p'}(\Omega)$, 使得
\begin{equation*}
    \langle f, u\rangle = \sum_{|\alpha| \leq k}\int_{\Omega}D^{\alpha}u\psi_{\alpha} \,{\rm d}x.
\end{equation*} 
从而$W^{k, p}(\Omega)$中的弱收敛可表现为:
\begin{equation*}
    \boxed{u_j \rightharpoonup u \ \text{in}\ W^{k, p} \Longleftrightarrow \sum_{|\alpha| \leq k}D^{\alpha}(u_j - u)\psi_{\alpha} \,{\rm d}x \rightarrow 0, \forall (\psi_{\alpha}) \in \times_{|\alpha| \leq k}L^{p'}(\Omega).}
\end{equation*}

\begin{corollary}
    $W^{k, p}(\Omega)$是自反且可分的Banach空间.
    \begin{proof}
        注意到$i$是等距嵌入, 从而$W^{k, p}(\Omega)$可以看作是$\times_{|\alpha| \leq m}L^p(\Omega)$的闭子空间.
        由此足以证得所需结论.
    \end{proof}
\end{corollary}

\subsubsection{延拓, 逼近与嵌入}

\begin{itemize}
    \item \textbf{延拓}. 基本问题是: 给定$u \in W^{k, p}(\Omega)$, 是否总能存在$\widetilde{u} \in W^{k, p}(\mathbb{R}^n)$, 使得$\widetilde{u}|_{\Omega} = u$?
    事实上, 对于$W_0^{k, p}(\Omega)$型空间, 不论区域$\Omega$如何选取, 延拓总是可能的, 且此延拓是有界的.
    我们只需将$u$在$\Omega$外定义为零即可. 但对于$W^{k, p}$型空间, 由于函数在边界处可能趋于无穷, 故我们不能采取零延拓的方式.
    然而, 若$\Omega$具有充分光滑的边界, 那么延拓是可能的:
\end{itemize}

\begin{theorem}
    设$\Omega$是$\mathbb{R}^n$中的有界区域, 其中$\partial\Omega$是一致$C^k$的.
    那么对任意的$l \in [0, k], p \in [1, \infty)$, 存在有界线性算子$T\colon W^{l, p}(\Omega) \rightarrow W^{l, p}(\mathbb{R}^n)$, 使得$Tu(x) = u(x)$, a.e. $x \in \Omega$.
\end{theorem}

\begin{itemize}
    \item \textbf{逼近}. 我们已经知道$C_c^{\infty}(\Omega)$在$L^p(\Omega)$中稠密. 特别地, $C^{\infty}(\Omega) \cap L^p(\Omega)$在$L^p(\Omega)$中稠密.
    以下逼近定理将此结果推广到了Sobolev空间上: 
\end{itemize}

\begin{theorem}[Serrin-Meyers]
    设$\Omega$是$\mathbb{R}^n$中的有界区域. 若$p \in [1, \infty)$, 则$C^{\infty}(\Omega) \cap W^{k, p}(\Omega)$在$W^{k, p}(\Omega)$中稠密.
    \begin{proof}
        我们先证明局部的逼近, 再利用单位分解得到整体的逼近.

        取一族光滑化子$\{\eta_{\varepsilon}\}_{\varepsilon > 0}$. 令
        \begin{equation*}
            u_{\varepsilon}(x) = (\eta_{\varepsilon} \ast u)(x), \qquad x \in \Omega_{\varepsilon},
        \end{equation*}
        其中$\Omega_{\varepsilon} = \{x \in \Omega\colon {\rm dist}(x, \partial\Omega) > \varepsilon\}$.
        显然对任意的$\varepsilon > 0$, $u_{\varepsilon} \in C^{\infty}(\Omega_{\varepsilon})$. 再根据恒等逼近的理论可知, 当$\varepsilon \rightarrow 0$时, $u_{\varepsilon}$在$W_{{\rm loc}}^{k, p}(\Omega)$中收敛到$u$.

        现取$\Omega$的一族开覆盖$\{\Omega_i\}$, 使得$\Omega = \bigcup_{i = 1}^{\infty}\Omega_i$, 且对任意的$i$, 有$\overline{\Omega_i} \subseteq \Omega_{i + 1}$.
        再令$V_i = \Omega_{i + 1} \smallsetminus \overline{\Omega_{i - 1}}, i = 1, 2, \cdots$, 其中$U_0 = \varnothing$.
        显然有$\Omega = \bigcup_{i = 1}^{\infty}V_i$. 且对任意的$x \in \Omega$, 只有有限个$V_i$包含$x$.
        今取$\{V_i\}$对应的一族单位分解$\{\zeta_i\}$. 对任意的$u \in W^{k, p}(\Omega)$, 显然有$\zeta_iu \in W^{k, p}(\Omega)$, 且${\rm supp}\ \zeta_i \subseteq V_i$, 根据前述的局部逼近, 对任意的$\eta > 0$, 我们总可以选取足够小的$\delta_i > 0$, 使得 
        \begin{equation*}
            \Vert \eta_{\delta_i} \ast (\zeta_iu) - \zeta_iu\Vert_{k, p} \leq \frac{\eta}{2^i} \qquad (i = 1, 2, \cdots).
        \end{equation*}
        记$u^i = \eta_{\delta_i} \ast (\zeta_iu)$. 令$v = \sum_{i = 1}^{\infty}u^i$.
        一方面, 注意到对任意的$x \in \Omega$, $\sum_{i = 1}^{\infty}u^i$是有限和, 故$v \in C^{\infty}(\Omega)$.
        另一方面, 取$V \subseteq \Omega$使得$\overline{V} \subseteq \Omega$, 我们有
        \begin{equation*}
            \Vert v - u \Vert_{W^{k, p}(V)} \leq \sum_{i = 1}^{\infty}\Vert u^i - \zeta u\Vert_{W^{k, p}(V)} \leq \eta.
        \end{equation*}
        注意到$V$是任意的, 故有$\Vert v - u \Vert_{W^{k, p}(\Omega)} \leq \eta$. 由此同时表明$v \in W^{k, p}(\Omega)$.
    \end{proof}
\end{theorem}

利用光滑函数在Sobolev空间中的稠密性, 我们可以轻松地将Poincar\'e不等式推广到$W_0^{1, p}(\Omega) (1 \leq p < \infty)$上:

\begin{corollary}[Poincaré不等式]
    设$\Omega \subseteq \mathbb{R}^n$是有界区域. 若$u \in W_0^{1, p}(\Omega), 1 \leq p < \infty$, 那么存在常数$C = C(p, \Omega) > 0$, 使得 
    \begin{equation*}
        \Vert u \Vert_{L^p} \leq C\Vert \nabla u \Vert_{L^p}.
    \end{equation*}
\end{corollary}

回顾前几节中对于空间$H^1$的定义: $C^1(\overline{\Omega})$在范数$\Vert \cdot \Vert_{H^1}$下的完备化.
注意到$W^{1, 2}(\Omega)$是包含$C^1(\overline{\Omega})$的完备空间, 因此$H^1(\Omega) \subseteq W^{1, 2}(\Omega)$.
另一方面, 根据上述定理逼近可知, $W^{1, 2}(\Omega) \subseteq H^1(\Omega)$. 从而
\begin{equation*}
    \boxed{H^1(\Omega) = W^{1, 2}(\Omega).}
\end{equation*} 
这也是$H^1(\Omega)$的一种等价定义.

此外, 由Poincaré不等式可知, 
\begin{equation*}
    \boxed{u \mapsto \left(\int_{\Omega}|\nabla u|^p \,{\rm d}x\right)^{1/p}}
\end{equation*}
是$W_0^{1, p}(\Omega)$上的一个等价范数, 其中$\Omega$是$\mathbb{R}^n$中的有界区域, $1 \leq p < \infty$.

\begin{itemize}
    \item \textbf{嵌入.} 利用嵌入定理, 我们可以将某些特别的Sobolev空间看作是某些常见函数空间的闭子空间.
    特别地, 若该嵌入还是紧的(有界集映到列紧集), 则我们可以使用一些特殊的定理, 如Arzelà-Ascoli定理, 来判断有界集合的列紧性.
\end{itemize}

\begin{theorem}[Sobolev嵌入定理]
    设$\Omega$是$\mathbb{R}^n$中具有一致$C^m$边界的有界区域, $1 \leq q < \infty$, $k \in \mathbb{Z}_{\geq 0}$.
    则对任意的$j \in \mathbb{Z}_{\geq 0}$, 有嵌入关系:
    \begin{gather*}
        W^{k, p}(\Omega) \hookrightarrow L^r(\Omega), \qquad \frac{1}{r} \geq \frac{1}{p} - \frac{k}{n}\ \left(k < \frac{n}{p}\right), \\
        W^{k + j, p}(\Omega) \hookrightarrow C^{j, \lambda}(\overline{\Omega}), \qquad 0 < \lambda \leq k - \frac{n}{p}\ \left(k > \frac{n}{p}\right),
    \end{gather*}
    其中$C^{j, \lambda}(\overline{\Omega})$是H\"older型空间.
\end{theorem}

最常用的是$k = 1$的情形: 记$p^* = \frac{np}{n - p}$为$q$的\textbf{Sobolev型共轭指标}, 则 
\begin{gather*}
    W^{1, p}(\Omega) \hookrightarrow L^r(\Omega), \qquad 1 \leq r \leq p^*\ (n > p), \\ 
    W^{1, p}(\Omega) \hookrightarrow C(\overline{\Omega}) \qquad (p > n).
\end{gather*}

\begin{remark}
    当$k = n = 1, \Omega = (a, b) \subseteq \mathbb{R}$时, 嵌入定理的结论很容易从H\"older不等式推出.
    注意到此时$p^* = p'$, 且广义导数和几乎处处导数是一致的, 故对任意的$x, y \in (a, b)$, 有 
    \begin{equation*}
        |u(x) - u(y)| = \left|\int_a^b\dot u(t) \,{\rm d}t\right| \leq |x - y|^{1/p'}\Vert \dot u \Vert_{L^p},
    \end{equation*}
\end{remark}

\begin{theorem}[Rellich-Kondrachov]
    设$\Omega$是$\mathbb{R}^n$中具有一致$C^m$边界的有界区域, $1 \leq p \leq \infty$, $m \in \mathbb{Z}_{\geq 0}$, 则如下嵌入 
    \begin{gather*}
        W^{k, p}(\Omega) \hookrightarrow L^r(\Omega), \qquad 1 \leq r < \frac{np}{n - kp}\ \left(k < \frac{n}{p}\right), \\
        W^{k, p}(\Omega) \hookrightarrow C(\overline{\Omega}), \qquad \left(k > \frac{n}{p}\right)
    \end{gather*}
    均是紧的.
\end{theorem}

以下是一种常见的特殊情形:

\begin{proposition}
    设$\Omega$为$\mathbb{R}^n$中的有界区域, $1 \leq p < \infty$, 则$W_0^{1, p}(\Omega)$中的单位闭球是$L^p(\Omega)$中的列紧集.
    \begin{proof}
        记$\mathbb{B}$为$W_0^{1, p}(\Omega)$中的单位闭球. 以下证明对任意的$\varepsilon > 0$, 在$L^p$模下, $\mathbb{B}$是完全有界的, 即存在有限的$\varepsilon$-网. 

        令$S = C_0^{\infty}(\Omega) \cap \mathbb{B}$. 取一族光滑化子$\{\eta_{\delta}\}_{\delta > 0}$.
        对任意的$\delta > 0$, 记$S_{\delta} = \{v_{\delta}\colon v \in S\}$, 其中$v_{\delta} = v \ast \eta_{\delta}$.
        由恒等逼近的理论可知, 对任意的$v \in S$和$\varepsilon > 0$, 存在充分小的$\delta_0 = \delta(\varepsilon) > 0$, 使得 
        \begin{equation*}
            \Vert v - v_{\delta} \Vert_{L^p} < \varepsilon, \qquad \forall \delta \in (0, \delta_0].
        \end{equation*}
        现固定$\delta = \delta_0$. 一方面, 由卷积不等式可知, $S_{\delta_0}$是完全有界的; 另一方面, 注意到 
        \begin{equation*}
            |\nabla v_{\delta}(x)| = |(v \ast \nabla\eta_{\delta})(x)| \leq \Vert v \Vert_{L^p},
        \end{equation*}
        其中$C$是一个不依赖于$\delta$的常数, 由此表明$S_{\delta_0}$还是等度连续的. 
        根据Arzelà-Ascoli定理, 我们可以找到$\{w_i\}_{i = 1}^l \subseteq S_{\delta_0}$, 使得对任意的$v_{\delta_0} \in S_{\delta_0}$, 总存在$w_i$, 使得 
        \begin{equation}\label{36}
            \Vert v_{\delta_0} - w_i\Vert_C < \frac{\varepsilon}{5|\Omega|}.
        \end{equation} 
        注意到此时$w_i$不一定在$\mathbb{B}$内. 然而, 每个$w_i$都对应着一个$v_{\delta_0}^i$使得不等式\eqref{36}成立, 其中$v^i \in S$.
        若$v_{\delta_0}^i \notin \mathbb{B}$, 那么我们选取充分小的$\delta_i \in (0, \delta_0]$, 使得$w_i' = v_{\delta_i}^i$在支集落在$\Omega$内.
        此时显然有$w_i' \in \mathbb{B}$, 且 
        \begin{equation*}
            \Vert w_i - w_i'\Vert_{L^p} \leq \Vert w_i - v^i_{\delta_0}\Vert_{L^p} + \Vert v^i_{\delta_0} - v_i\Vert_{L^p} +  \Vert w_i' - v^i\Vert_{L^p} < \frac{3\varepsilon}{5}.
        \end{equation*}
        今对任意的$u \in B$, 我们选取$v \in S$使得$\Vert u - v \Vert_{1, p} \leq \varepsilon/5$, 从而有 
        \begin{equation*}
            \Vert u - w_i'\Vert_{L^p} \leq \Vert u - v \Vert_{L^p} + \Vert v - v_{\delta_0} \Vert_{L^p} + \Vert w_i - w_i'\Vert_{L^p} < \varepsilon.
        \end{equation*}
        由此表明$\{w_i'\} \subseteq \mathbb{B}$是$\mathbb{B}$的$L^p$意义下的有限$\varepsilon$-网.
    \end{proof}
\end{proposition}

\subsubsection{Euler-Lagrange方程}

在这一节中, 我们旨在将E-L方程推广到$W^{1, p}(\Omega)$上. 给定Lagrange函数$L \in C(\Omega \times \mathbb{R}^N \times \mathbb{R}^{nN})$, 其中$\Omega \subseteq \mathbb{R}^n$为一有界区域, $L$是可微的.
考虑泛函 
\begin{equation*}
    I(u) = \int_{\Omega}L(x, u(x), \nabla u(x)) \,{\rm d}x.
\end{equation*}
为了使泛函$I$是良定义的, 且E-L方程式有意义的, 我们还需要对$L$添加如下假设:

\begin{enumerate}
    \item $L, L_u, L_p$是连续的. \label{con1}
    \item $|L(x, u, p)| \leq C(1 + |u|^q + |p|^q)$.\label{con2}
    \item $|L_u(x, u, p)| + |L_p(x, u, p)| \leq C(1 + |u|^q + |p|^q)$.
\end{enumerate}

\begin{proposition}\label{prop2.25}
    在上述假设下, 对任意的$u \in W^{1, q}(\Omega)$和$\varphi \in C_0^1(\Omega)$, 一阶变分有表达式:
    \begin{equation}\label{37}
        \delta I(u^*, \varphi) = \int_{\Omega}(L_u(x, u(x), \nabla u(x))\varphi(x) + L_p(x, u(x), \nabla u(x))\nabla\varphi(x)) \,{\rm d}x.
    \end{equation}
    \begin{proof}
        直接计算即可. 其中极限号和积分号交换顺序的合理性由控制收敛定理保证.
    \end{proof}
\end{proposition}

我们还可以将$L$的假设条件再放松一些: 
\begin{itemize}
    \item 连续性假设. 事实上, 若$L, L_u, L_p$满足如下\textbf{Carath\'eodory条件}:
    \begin{align*}
            &\forall (u, p) \in \mathbb{R}^N \times \mathbb{R}^{nN}, x \mapsto L(x, u, p)\ \text{是可测的}, \\ 
            &\text{对}\ {\rm a.e.}\ x \in \Omega, (u, p) \mapsto L(x, u, p)\ \text{是连续的},
    \end{align*}
    则命题\ref{prop2.25}的结论仍成立.
    \item 增长性假设. 利用嵌入定理, $|L_u| + |L_p|$的增长幂次可以放松为:
    \begin{equation}\label{38}
        |L_u(x, u, p)| + |L_p(x, u, p)| \leq 
        \begin{cases}
            C(1 + |u|^r + |p|^q) \quad (r \leq q^*) &q < n, \\ 
            C(1 + |u|^r + |p|^q) \quad (r \geq 1) &q = n, \\ 
            C(1 + |p|^q) \quad &q > n.
        \end{cases}
    \end{equation}
\end{itemize} 

因此, 若$L, L_u, L_p$满足Carathéodory条件, 且增长幂次满足条件\ref{con2}和\eqref{37}, 则一阶变分仍具有如\eqref{38}所示的表达式.
特别地, 当$n = 1, \Omega = (a, b)$时, 若设$u^* \in M = \varphi + W_0^{1, q}(\mathbb{R}^n)$是$I$在$M$中的极小点, 其中$\varphi \in W^{1, q}(\Omega)$, 根据变分学基本引理, $u^*$还满足如下积分形式的E-L方程:
\begin{equation*}
    \boxed{\int_a^tL_u(t, u^*(t), \nabla u^*(t)) \,{\rm d}t - L_p(t, u^*(t), \nabla u^*(t)) = {\rm const.} \qquad a.e.}
\end{equation*}

\begin{remark}
    相较于前述转化成一元函数的情形, 我们还可以直接考虑Banach空间上的微分.
    \begin{definition}
        设$X$是一个Banach空间, $U \subseteq X$是一个开集. 给定$U$上的函数$f \in C(U)$.
        称$f$在$x_0 \in U$处是\textbf{G\^ateaux可微}的, 如果对任意的$h \in X$, 存在$c \in \mathbb{R}$, 使得 
        \begin{equation*}
            |f(x_0 + th) - f(x_0) - tc| = o(t) \qquad (t \rightarrow 0).
        \end{equation*}
        若上式成立, 则称实数$c$为$f$在$x_0$处的\textbf{G\^ateaux导数}, 记为${\rm d}f(x_0, h)$.
    \end{definition}
    由上述定义可知, 若考虑$W^{1, q}(\mathbb{R}^N)$上的泛函$I$, 则G\^ateaux导数${\rm d}I(u, \varphi)$中的$\varphi$属于$W_0^{1, q}(\Omega)$, 而不是变分$\delta I(u, \varphi)$中的$C_0^1(\Omega)$.
    为使${\rm d}I(u, \varphi)$中的积分有定义, 根据嵌入定理, 我们需要规定$|L_u| + |L_p|$的增长条件:
    \begin{equation}\label{39}
        |L_u(x, u, p)| + |L_p(x, u, p)| \leq 
        \begin{cases}
            C(1 + |u|^{r - 1} + |p|^{q - 1}) \quad (r \leq q^*) &q < n, \\ 
            C(1 + |u|^r + |p|^{q - 1}) \quad (r \geq 1) &q = n, \\ 
            C(1 + |p|^{q - 1}) \quad &q > n.
        \end{cases}
    \end{equation}
    在此约束下, 若更设$L$满足条件\ref{con1}和\ref{con2}, 则${\rm d}I(u, \varphi)$的表达式与$\delta I(u, \varphi)$的表达式相同; 如\eqref{37}所示.
    注意到增长条件\eqref{39}比\eqref{38}更强, 故在变分学中我们一般不使用G\^ateaux导数, 而直接使用变分导数.
\end{remark}

\subsection{弱下半连续性}

\subsubsection{凸性与弱下半连续性}

回顾下半连续性的定义: 称函数$f\colon X \rightarrow \mathbb{R}$是弱下半连续的, 如果对任意的$t \in \mathbb{R}$, 下方水平集
\begin{equation*}
    f_t = \{x \in X\colon f(x) \leq t\}
\end{equation*}
是闭的. 以下我们主要探究如何判断\textbf{弱序列下半连续性}.

注意到对于Banach空间中的凸子集而言, (强)闭等价于弱闭. 以下命题是对前述结论的推广:

\begin{proposition}\label{prop2.28}
    设$C$是Banach空间$X$中的凸子集, 则$C$是闭的, 当且仅当$C$是弱序列闭的.
    \begin{proof}
        只需证明弱序列闭 $\Rightarrow$ 闭. 现设$C$是弱序列闭的. 取$x \in X \smallsetminus C$.
        若$C$不是闭集, 则对任意的$r > 0$, $B_r(x) \cap C \neq \varnothing$. 
        现对任意的$n \in \mathbb{Z}_{>0}$, 取$x_n \in B_{1/n}(x) \cap C$, 则显然有$x_n \rightarrow x$, 从而有$x_n \rightharpoonup x$.
        再根据$C$的弱序列闭性, 故有$x \in C$, 矛盾. 
    \end{proof}
\end{proposition}

我们再来回顾凸函数的概念. 称函数$f\colon X \rightarrow \mathbb{R}$是凸的, 如果对于任意的$x, y \in X$和$t \in [0, 1]$, 有 
\begin{equation*}
    f(tx + (1 - t)y) \leq tf(x) + (1 - t)f(y).
\end{equation*}
可以证明, $f$是凸函数 $\Rightarrow$ $f_t$是凸集. 结合命题\ref{prop2.28}的结论和凸函数的定义, 我们便有如下结果:

\begin{proposition}
    设$X$是Banach空间. 若$f\colon X \rightarrow \mathbb{R}$是凸函数, 则 
    \begin{equation*}
        f\ \text{序列下半连续}\ \Longleftrightarrow f\ \text{弱序列下半连续}.
    \end{equation*}
    特别地, 凸 $+$ 强连续 $\Rightarrow$ 弱序列下半连续.
\end{proposition}

\begin{example}\label{ex2.30}
    设$\Omega$是$\mathbb{R}^n$中的有界区域. 考虑定义在$W_0^{1, p}(\Omega)$上的泛函
    \begin{equation*}
        I(u) = \int_{\Omega}(|\nabla u(x)|^p + c(x)|u(x)|^r) \,{\rm d}x,
    \end{equation*}
    其中$1 \leq p < \infty, 1 \leq r \leq p^*, c \in L^{\infty}(\Omega), c \geq 0$.
    容易验证, $I$是连续凸的, 所以也是弱下半连续的. 若去掉$c$的非负性限制, 此时$I$不再是凸的, 因此$I$未必是弱序列下半连续的.
    然而, 若进一步假设$r < p^*$, 那么$I$仍是弱序列下半连续的. 事实上, 若$u_j$在$W_0^{1, p}(\Omega)$中收敛到$u$, 则 
    \begin{equation*}
        \int_{\Omega}|\nabla u(x)|^p \,{\rm d}x \leq \liminf\limits_{j \rightarrow \infty}\int_{\Omega}|\nabla u_j(x)|^p \,{\rm d}x.
    \end{equation*}
    我们还有 
    \begin{equation*}
        \int_{\Omega}c(x)|\nabla u(x)|^r \,{\rm d}x \leq \liminf\limits_{j \rightarrow \infty}\int_{\Omega}c(x)|\nabla u_j(x)|^r \,{\rm d}x,
    \end{equation*}
    由此足以说明$I$的弱序列下半连续性. 因若不然, 则存在$\varepsilon_0 > 0$以及子列$\{u_{k_j}\}$, 使得 
    \begin{equation*}
        \int_{\Omega}c(x)|\nabla u(x)|^r \,{\rm d}x + \varepsilon_0 > \int_{\Omega}c(x)|\nabla u_{k_n}(x)|^r \,{\rm d}x.
    \end{equation*}
    根据紧嵌入定理, 存在子列$\{u_{j'}\} \subseteq \{u_{k_j}\}$, 使得$u_{j'}$在$L^r$中收敛到$u$, 从而有 
    \begin{equation*}
        \int_{\Omega}c(x)|u_{j'}(x)|^r \,{\rm d}x \rightarrow \int_{\Omega}c(x)|u(x)|^r \,{\rm d}x,
    \end{equation*}
    矛盾!
\end{example}

我们现在回到对变分问题的讨论上去. 要使$I = I(u)$对于$u$是凸的, 则Lagrange函数$L = L(x, u, p)$对于$(u, p)$是凸的. 
下述命题表明, 对$u$的凸性要求可以由一定的增长性条件替代:

\begin{theorem}[Tonelli-Morrey]
    设$L\colon \overline{\Omega} \times \mathbb{R}^N \times \mathbb{R}^{nN} \rightarrow \mathbb{R}$, 且满足 
    \begin{itemize}
        \item $L \in C^1(\overline{\Omega} \times \mathbb{R}^N \times \mathbb{R}^{nN})$, 
        \item $L \geq 0$,
        \item 对任意的$(x, u) \in \Omega \times \mathbb{R}^N$, $p \mapsto L(x, u, p)$是凸的,
    \end{itemize}
    则$I(u) = \int_{\Omega}L(x, u(x), \nabla u(x)) \,{\rm d}x$在$W^{1, q}(\Omega)\ (1 \leq q < \infty)$上是弱下半连续的.
    \begin{proof}
        设$u_j$在$W^{1, q}$中弱收敛到$u$. 由紧嵌入定理可知, 存在子列, 不妨记为$\{u_j\}$, 使得$u_j$在$L^q$中收敛到$u$.
        再由Riesz定理, 存在子列的子列, 仍记为$\{u_j\}$, 使得$u_j(x) \rightarrow u(x)$ a.e. $x \in \Omega$.
        现对任意的$\varepsilon > 0$, 选取紧集$K \subseteq \Omega$使得$|\Omega \smallsetminus K| < \varepsilon$, 且 
        \begin{itemize}
            \item $u_j$在$K$上一致收敛到$u$ (Egorov定理), 
            \item $u$和$\nabla u$在$K$上是连续的 (Luzin定理), 
            \item 若$I(u) < +\infty$, 则 
            \begin{equation*}
                \int_KL(x, u(x), \nabla u(x)) \,{\rm d}x \geq \int_{\Omega}L(x, u(x), \nabla u(x)) - \varepsilon
            \end{equation*}
            (Lebesgue积分的绝对连续性). 若$I(u) = +\infty$, 则取 
            \begin{equation*}
                \int_KL(x, u(x), \nabla u(x)) \,{\rm d}x > \frac{1}{\varepsilon}.
            \end{equation*}
        \end{itemize}
        现利用$L$的凸性, 我们有 
        \begin{align*}
            I(u_j) &\geq \int_KL(x, u_j(x), \nabla u_j(x)) \,{\rm d}x \\ 
            &\geq \int_KL_p(x, u_j(x), \nabla u(x))(\nabla u_j(x) - \nabla u(x)) \,{\rm d}x + \int_KL(x, u_j(x), \nabla u(x)) \,{\rm d}x \\ 
            &= \int_KL(x, u_j(x), \nabla u(x)) \,{\rm d}x + \int_KL_p(x, u(x), \nabla u(x))(\nabla u_j(x) - \nabla u(x)) \,{\rm d}x \\
            + &\int_K(L_p(x, u_j(x), \nabla u(x)) - L_p(x, u(x), \nabla u(x)))(\nabla u_j(x) - \nabla u(x)) \,{\rm d}x \\ 
            &= I_1 + I_2 + I_3.
        \end{align*}
        对于$I_1$, 注意到$u_j$在$K$上一致收敛到$u$, 故由$L$的连续性可知, 
        \begin{equation*}
            I_1 \rightarrow \int_KL(x, u(x), \nabla u(x)) \,{\rm d}x.
        \end{equation*}
        对于$I_2$, 由于$L_p \in L^{\infty}(\Omega)$, 从而有$\chi_KL_p \in L^{\infty}(\Omega) \subseteq L^{q'}(\Omega)$.
        利用条件$u_j \rightharpoonup u \ \text{in}\ W^{1, q}(\Omega)$, 我们有$\nabla u_j \rightharpoonup \nabla u\ \text{in}\ L^q(\Omega)$.
        由此表明 
        \begin{equation*}
            \lim\limits_{j \rightarrow \infty}I_2 = \lim\limits_{j \rightarrow \infty}\int_KL_p(x, u(x), \nabla u(x))(\nabla u_j(x) - \nabla u(x)) \,{\rm d}x = 0.
        \end{equation*}
        最后, 注意到弱收敛列是有界列, 故存在$C_1, C_2 > 0$使得 
        \begin{equation*}
            \Vert \nabla u_j - \nabla u\Vert_{L^1} \leq C_1(\Vert \nabla u_j - \nabla u\Vert_{L^q}) \leq C_1(\Vert \nabla u_j\Vert_{L^q} + \Vert \nabla u\Vert_{L^q}) \leq C_2.
        \end{equation*}
        再注意到$L_p(x, u_j(x), \nabla u(x))$在$K$上一致收敛到$L_p(x, u(x), \nabla u(x))$, 从而有 
        \begin{equation*}
            \lim\limits_{j \rightarrow \infty}I_3 = 0.
        \end{equation*}
        综上所述, 当$I(u) < +\infty$时, 有 
        \begin{equation*}
            \liminf\limits_{j \rightarrow \infty}I(u_j) \geq \int_KL(x, u(x), \nabla u(x)) \geq I(u) - \varepsilon.
        \end{equation*}
        由$\varepsilon > 0$的任意性, 即可表明$I$的弱序列下半连续性. $I(u) = +\infty$的情形的证明是类似的.
    \end{proof}
\end{theorem}

\begin{remark}
    事实上, 上述定理中对$L$可微性的假设可以减弱为:
    \begin{itemize}
        \item 对a.e. $(x, u) \in \Omega \times \mathbb{R}^N$, $p \mapsto L(x, u, p)$是可微的.
        \item $L$和$L_p$满足Carathéodory条件.
    \end{itemize}
\end{remark}

\begin{corollary}[存在性]\label{coro2.33}
    设$\Omega$是$\mathbb{R}^n$中的有界区域. $\varphi \in W^{1, q}(\Omega)\ (1 < q < \infty)$.
    又设 
    \begin{itemize}
        \item $L$和$L_p$满足Carathéodory条件,
        \item 存在$a \in L^1(\Omega)$和$b > 0$, 使得对任意的$(x, u, p) \in \Omega \times \mathbb{R}^N \times \mathbb{R}^{nN}$, 有$L(x, u, p) \geq -a(x) + b|p|^q$,
        \item 对任意的$(x, u) \in \Omega \times \mathbb{R}^N$, $p \mapsto L(x, u, p)$是凸的,
    \end{itemize}
    那么泛函 
    \begin{equation*}
        I(u) = \int_{\Omega}L(x, u(x), \nabla u(x)) \,{\rm d}x
    \end{equation*}
    在$\varphi + W_0^{1, p}(\Omega)$上达到极小值.
    \begin{proof}
        在自反的Banach空间$W_0^{1, q}(\Omega)$上考虑泛函$v \mapsto I(\varphi + v)$.
        将Tonelli-Morrey定理运用到泛函$I' = I + \int_{\Omega}a(x) \,{\rm d}x$上, 即可说明$I'$是弱序列下半连续的, 从而$I$也是弱序列下半连续的.
        为证明极小值点的存在性, 现只需证明$I$是强制的. 事实上, 对任意的$v \in W_0^{1, q}(\Omega)$, 由Poincaré不等式可知, 存在$\alpha, \beta > 0$使得 
        \begin{equation*}
            I(\varphi + v) \geq -\int_{\Omega}a(x) \,{\rm d}x + b\int_{\Omega}|\nabla (u_0 + v)|^q \,{\rm d}x \geq \alpha\Vert v \Vert_{1, q}^q - \beta.
        \end{equation*}
        由此表明$I$是强制的.
    \end{proof}
\end{corollary}

\begin{example}
    给定$\mathbb{R}^n$上的有界区域$\Omega$. 那么对任意的$f \in L^2(\Omega)$, 存在$u \in H_0^1(\Omega)$, 满足等式$-\Delta u = f$, 即Poisson方程
    \begin{equation*}
        \begin{cases}
            -\Delta u = f \quad &\text{in}\ \Omega, \\ 
            u = 0 \quad &\text{on}\ \partial\Omega
        \end{cases}
    \end{equation*}
    存在广义解. 事实上, 先将Poisson方程转化成等价的泛函 
    \begin{equation*}
        I(u) = \int_{\Omega}\left(\frac{1}{2}|\nabla u|^2 - fu\right) \,{\rm d}x.
    \end{equation*}
    容易验证, $I$满足推论\ref{coro2.33}的假设条件, 故$I$存在极小值点.
\end{example}

\subsubsection{拟凸性}

\emph{Motivation}: 降低对Lagrange函数凸性的要求. 例如, 设$\Omega$是$\mathbb{R}^n$中的有界区域. 
$u = (u^1, \cdots, u^n) \in C^1(\Omega)$, $f \in C^1(\mathbb{R})$是凸函数.
考虑$u$的Jacobi行列式$A = \det (\nabla u)$, 定义Lagrange函数 
\begin{equation*}
    L\colon \mathbb{R}^{n \times n} \rightarrow \mathbb{R}, A \mapsto f(\det A).
\end{equation*}
可以验证, $L$对$p$不是凸的.

考虑特殊情形: $L$只依赖于$p$.

\begin{proposition}
    设$\Omega \subseteq \mathbb{R}^n$是一个区域, $L \in C(\mathbb{R}^{n \times N})$. 
    如果 
    \begin{equation*}
        I(u) = \int_{\Omega}L(\nabla u(x)) \,{\rm d}x
    \end{equation*}
    在$W^{1, \infty}(\Omega)$上是弱\textsuperscript{*}下半连续的, 那么对于任意的立方体$D \subset \subset \Omega$和$A \in \mathbb{R}^{n \times N}$, 有 
    \begin{equation*}
        \int_DL(A + \nabla\varphi(x)) \,{\rm d}x \geq L(A)|D|, \quad \forall \varphi \in W_0^{1, \infty}(\mathbb{R}^N).
    \end{equation*}
    \begin{proof}
        不妨设$D = [0, 1]^n$. 对任意的$k \in \mathbb{Z}_{> 0}$, 将$D$作$2^k$等分: $D = \bigcup_{l = 1}^{2^{kn}}D_l^k$, 其中$D_l^k$是边长为$2^{-k}$, 中心在$c_l^k = 2^{-k}(y_1^l + 1/2 + \cdots y_n^l)$的立方体,
        其中$(y_1^l, \cdots, y_n^l)$遍历$(0, 1, \cdots, 2^k - 1)^n$. 对任意的$v \in W_0^{1, \infty}(D)$, 将其周期扩展到$\mathbb{R}^n$上. 
        令 
        \begin{equation*}
            w_k(x) = \frac{1}{2^k}v(2^k(x - c_l^k)), \quad x \in D_l^k, \forall l = 1, 2, \cdots, 2^{kn}.
        \end{equation*}  
        容易验证,  
        \begin{equation*}
            \begin{cases}
                w_k \rightharpoonup 0, \\
                \nabla w_k \rightharpoonup^* 0
            \end{cases}
            \quad \text{in}\ L^{\infty}(D).
        \end{equation*}
        现对任意的$A \in \mathbb{R}^{n \times N}$, 定义$u_k(x)= Ax + w_k(x), k = 1, 2, \cdots$, 并且令其在$D$外为0.
        由前述分析可知, $u_k \rightharpoonup^* u = Ax$. 一方面, 我们有$I(u) = |\Omega|L(A)$; 另一方面, 
        \begin{align*}
            I(u_k) &= \int_DL(A + \nabla w_k(x)) \,{\rm d}x + \int_{\Omega\smallsetminus D}L(A) \,{\rm d}x \\ 
            &= \sum_{l = 1}^{2^{kn}}\int_{D^k_l}L(A + \nabla v(2^k(x - c_l^k))) \,{\rm d}x + L(A)|\Omega \smallsetminus D| \\ 
            &= \int_DL(A + \nabla v(x)) \,{\rm d}x + L(A)|\Omega \smallsetminus D|.
        \end{align*}
        再利用$I$的弱\textsuperscript{*}序列下半连续性, 我们便证得了所需结果.
    \end{proof}
\end{proposition}

\begin{definition}
    函数$f$称为\textbf{拟凸的}, 如果对任意的$A \in \mathbb{R}^{n \times N}$和立方体$D \subseteq \mathbb{R}^n$, 有 
    \begin{equation*}
        \boxed{|D|f(A) \leq \int_Df(A + \nabla u(x)) \,{\rm d}x, \qquad \forall v \in W_0^{1, \infty}(D).}
    \end{equation*}
\end{definition}

\begin{itemize}
    \item 设$p \mapsto L(p)$是凸的, 则对任意的$\varphi \in W_0^{1, \infty}(\Omega)$, 由Jensen不等式可得: 
    \begin{align*}
        L(p) = L\left(|D|^{-1}\int_D(p + \nabla \varphi(x)) \,{\rm d}x\right) \leq |D|^{-1}\int_DL(p + \nabla\varphi(x)) \,{\rm d}x.
    \end{align*}
    这表明$L$也是拟凸的. 即拟凸的确是凸的推广.
    \item 事实上, 当$n = 1$或$N = 1$时, 拟凸和凸是等价的.
    \begin{itemize}
        \item $n = 1$. 取$\xi, \eta \in \mathbb{R}^N$. 对任意的$\lambda \in [0, 1]$, 令 
        \begin{equation*}
            \xi_1 = \xi + (1 - \lambda)\eta, \xi_2 = \xi - \lambda\eta,
        \end{equation*}
        即有 
        \begin{equation*}
            \xi = \lambda\xi_1 + (1 - \lambda)\xi_2, \eta = \xi_1 - \xi_2.
        \end{equation*}
        定义函数 
        \begin{equation*}
            \varphi(t) = 
            \begin{cases}
                t(1 - \lambda)\eta \quad &t \in [0, \lambda), \\ 
                (1 - t)\lambda\eta \quad &t \in [\lambda, 1],
            \end{cases}
        \end{equation*}
        再利用$L$的拟凸性, 有 
        \begin{align*}
            L(\xi) \leq \int_0^1L(\xi + D\varphi(t)) \,{\rm d}t &= \int_0^{\lambda}L(\xi_1) \,{\rm d}t + \int_{\lambda}^1L(\xi_2) \\
            &= \lambda L(\xi_1) + (1 - \lambda)L(\xi_2).
        \end{align*}
        这表明$p \mapsto L(p)$是凸的.
    \end{itemize}
\end{itemize}

\begin{example}
    事实上, 存在不是凸的拟凸函数. 例如, 设$n = N = 2$, 考虑函数$f\colon \varphi \mapsto \det(\nabla\varphi)$, 则$f$不是凸的.
    注意到 
    \begin{equation*}
        \det(\nabla\varphi) = \partial_1\varphi^1\partial_2\varphi^2 - \partial_1\varphi^2\partial_2\varphi^1 = \partial_1(\varphi^1\partial_2\varphi^2) - \partial_2(\varphi^1\partial_1\varphi^2), 
    \end{equation*}
    所以 
    \begin{equation*}
        \int_D\det(\nabla\varphi) \,{\rm d}x = 0, \qquad \forall \varphi \in W_0^{1, \infty}(\Omega).
    \end{equation*}
    从而对任意的$A = (a_{ij})$, 有 
    \begin{align*}
        |D|^{-1}\int_Df(A + \nabla\varphi) \,{\rm d}x = &|D|^{-1}\int_D(\det A + \det(\nabla\varphi) + a_{11}\partial_2\varphi^2 + a_{22}\partial_1\varphi^1 \\
        &- a_{12}\partial_1\varphi^2 - a_{21}\partial_2\varphi^1) \,{\rm d}x \\ 
        &= f(A).
    \end{align*}
    这表明$f$是拟凸的.
\end{example}

下述定理表明, 拟凸性同样可以保证泛函的弱序列下半连续性.

\begin{theorem}[Morrey-Acerbi-Fusco]
    当$1 \leq p < \infty$时, 若 
    \begin{equation*}
        I(u) = \int_{\Omega}L(\nabla u(x)) \,{\rm d}x
    \end{equation*}
    在$W^{1, p}(\Omega)$上是弱序列下半连续的(或当$p = \infty$时, $I$是弱\textsuperscript{*}序列下半连续的), 则$L$是拟凸的.
    反之, 若$L$是拟凸的, 且满足如下增长性条件:
    \begin{equation*}
        \begin{cases}
            |L(A)| \leq \alpha(1 + |A|) \quad &p = 1, \\ 
            -\alpha(1 + |A|^q) \leq L(A) \leq \alpha(1 + |A|^p) \quad &1 \leq q < p < \infty, \\
            |L(A)| \leq \eta(|A|) \quad &p = \infty,
        \end{cases}
    \end{equation*}
    其中$\alpha > 0$是常数, $\eta$是一个递增的连续函数. 则当$1 \leq p < \infty$时, $I$在$W^{1, p}(\Omega)$是弱序列下半连续的(或当$p = \infty$时, $I$是弱\textsuperscript{*}序列下半连续的).
\end{theorem}

\begin{corollary}[存在性]
    设$L \in C(\mathbb{R}^{n \times N})$是拟凸的, 并存在常数$C_2 > C_1 > 0$使得 
    \begin{equation*}
        C_1|A|^p \leq L(A) \leq C_2(1 + |A|^p), \qquad 1 < p < \infty.
    \end{equation*}
    那么泛函 
    \begin{equation*}
        I(u) = \int_{\Omega}L(\nabla u(x)) \,{\rm d}x
    \end{equation*}
    在$M = \varphi + W_0^{1, p}(\Omega)$上达到极小值, 其中$\varphi \in W^{1, p}(\Omega)$.
\end{corollary}

\subsection{正则性\texorpdfstring{$(n = 1)$}{}}

在得到极小点的存在性后, 还需要验证极小点是E-L方程的经典解. 这便是\textbf{正则性}问题.

以下恒假设Lagrange函数$L \in C^2$.

\begin{proposition}
    设$u^* \in C^1(J)$是泛函 
    \begin{equation*}
        I(u) = \int_JL(t, u(t), \dot u(t)) \,{\rm d}t
    \end{equation*}
    的极小点. 若对任意的$t \in J$, $\det(L_{p_ip_j}(t, u^*(t), \dot u^*(t))) \neq 0$, 那么$u^* \in C^2(J)$.
    \begin{proof}
        由积分形式的E-L方程可知, 存在常数$C$, 使得 
        \begin{equation*}
            L_p(t, u^*(t), \dot u^*(t)) = \int_{t_0}^tL_u(\tau, u^*(\tau), \dot u^*(\tau)) \,{\rm d}\tau - C.
        \end{equation*} 
        将上式等号右侧的函数记为$q = q(t)$. 令
        \begin{equation*}
            \varphi(t, p) = L_p(t, u^*(t), p) - q(t) \qquad ((t, p) \in J \times \mathbb{R}^N).
        \end{equation*}
        显然, $\varphi \in C^1$, 且有$\varphi(t, u^*(t)) = 0$. 此外, 由于
        \begin{equation*}
            \det(\partial_p\varphi(t, p)) = \det(L_{p_ip_j}(t, u^*(t), \dot u^*(t))) \neq 0,
        \end{equation*} 
        故由隐函数定理可知, 对任意的$t \in J$, 存在$t$的邻域$U$以及唯一的$\lambda \in C^1(U)$, 等式 
        \begin{equation*}
            \varphi(t, \lambda(t)) = 0
        \end{equation*}
        对任意的$t \in U$成立. 注意到整体上有等式$\varphi(t, u^*(t)) = 0$, 故$\dot u \in C^1$, 从而$u \in C^2$. 
    \end{proof}
\end{proposition}

\begin{example}
    若条件$\det(L_{p_ip_j}(t, u^*(t), \dot u^*(t))) \neq 0$不被满足, 则存在这样的泛函$I$, 其极值函数是$C^1$的, 而不是$C^2$的.
    事实上, 设$M = \{u \in C^1[-1, 1]\colon u(-1) = 0, u(1) = 1\}$, Lagrange函数$L = L(t, u, p) = u^2(p - 2t)^2$.
    容易验证, $L$对应的变分积分有极小点 
    \begin{equation*}
        u^*(t) = 
        \begin{cases}
            0 \quad &t < 0, \\ 
            t^2 \quad &t \geq 0.
        \end{cases}
    \end{equation*}
    此时$u^* \in C^1 \smallsetminus C^2$, 而当$t < 0$时, $L_{pp}(t, u^*(t), \dot u^*(t)) = 2(u^*(t))^2 = 0$.
\end{example}

\begin{theorem}\label{th2.42}
    设$L$满足如下增长条件:
    \begin{equation*}
        \begin{cases}
            |L| + |L_u| + |L_p| \leq C(1 + |p|^r) \quad &1 < r < \infty, \\ 
            {\rm None} \quad &r = \infty.
        \end{cases}
    \end{equation*}
    又设对任意的$(t, u, p) \in J \times \mathbb{R}^N \times \mathbb{R}^N$, $(L_{p_ip_j}(t, u, p))$是正定的.
    若$u^* \in W^{1, r}(J)$是泛函 
    \begin{equation*}
        I(u) = \int_JL(t, u(t), \dot u(t)) \,{\rm d}t
    \end{equation*}
    的极小点, 则可以改变$u^*$在一个零测集上的值, 使得$u^* \in C^2$.
    \begin{proof}
        只需证明, 改变$u^*$在一个零测集上的值以后, 有$u^* \in C^1$.

        由题设条件可知, 存在常数$C$, 使得 
        \begin{equation*}
            L_p(t, u^*(t), \dot u^*(t)) = q(t), \qquad {\rm a.e.}\ t \in J,
        \end{equation*}
        其中 
        \begin{equation*}
            q(t) = \int_{t_0}^tL_u(\tau, u^*(\tau), \dot u^*(\tau)) \,{\rm d}\tau - C.
        \end{equation*}
        注意到对$L$的增长性假设, 我们还有$q \in {\rm AC(J)}$.
        定义函数 
        \begin{equation*}
            \varphi\colon J \times \mathbb{R}^N \times \mathbb{R}^N \times \mathbb{R}^N \rightarrow \mathbb{R}, (t, u, p, q) \mapsto L_p(t, u, p) - q.
        \end{equation*}
        根据隐函数定理, 方程$\varphi(t, u, p, q) = 0$存在唯一的局部$C^1$解$p = \lambda(t, u, q)$.
        尽管$\dot u$可能不是连续的, 但若能说明上述隐函数定理得到的局部解还是整体唯一的, 则有 
        \begin{equation*}
            \dot u(t) = \lambda(t, u^*(t), q(t)), \qquad {\rm a.e.}\ t \in J,
        \end{equation*}
        再注意到嵌入关系$W^{1, r}(J) \hookrightarrow C(\overline{J})$, 故$\dot u^*$是几乎处处连续的. 事实上, 设$p_1$和$p_2$是方程$\varphi = 0$的整体解, 那么 
        \begin{equation*}
            q = L_p(t, u, p_1) = L_p(t, u, p_2),
        \end{equation*}
        从而有 
        \begin{equation*}
            0 = (L_p(t, u, p_1) - L_p(t, u, p_2)) \cdot (p_1 - p_2) = (B(p_1 - p_2)) \cdot (p_1 - p_2),
        \end{equation*}
        其中 
        \begin{equation*}
            B = \int_0^1L_{pp}(t, u, p_1 + \tau(p_2 - p_1)) \,{\rm d}\tau.
        \end{equation*}
        由题设条件可知, $B$是正定的, 从而$p_1 = p_2$.
    \end{proof}
\end{theorem}

\begin{example}
    若去掉定理\ref{th2.42}中对于矩阵$(L_{p_ip_j}(t, u^*(t), \dot u^*(t)))$整体的正定性假设, 则极小点未必是$C^1$的.
    考虑Lagrange函数$L = L(p) = (p^2 - 1)^2$, 以及定义域
    \begin{equation*}
        M = \{u \in {\rm Lip}[0, 1]\colon u(0) = u(1) = 0\}.
    \end{equation*}
    容易验证, $L$对应的变分积分有极小值$0$, 此时$L_{pp} = 4(3p^2 - 1)$不是正定的.
    若$u \in C^1$是极小点, 则有$\dot u(t) = \pm 1$, 但无论处于何种情况, 都不可能有满足边值条件$u(0) = u(1) = 0$的解.
\end{example}

总结:

\begin{equation*}
    \text{直接方法} \leadsto \text{找极小化序列}
    \begin{cases}
        \text{存在性}
        \begin{cases}
            \text{合适的空间: Sobolev空间} \\ 
            \text{弱下半连续性}
        \end{cases} \\ 
        \text{正则性}
    \end{cases}
\end{equation*}

最后, 我们以几个具体的变分问题为例, 探究其极小点的存在性与正则性.

\begin{example}[两点边值问题]
    记区间$J = (t_0, t_1)$, 环面$\mathbb{T}^2 = \mathbb{R}^2/\mathbb{Z}^2$.
    给定$a_0, a_1 \in \mathbb{T}^2, F \in C^2(J \times \mathbb{T}^2)$. 
    考虑如下方程 
    \begin{equation}\label{40}
        \ddot u(t) = \nabla_uF(t, u(t))
    \end{equation}
    的$C^2$解, 且满足边值条件$u(t_i) = a_i, i = 0, 1$.
\end{example}

先将$F$周期延拓到全空间上去, 即对任意的$t \in J$, 定义 
\begin{equation*}
    F(t, u_1 + 1, u_2) = F(t, u_1 + 1, u_2) = F(t, u_1, u_2).
\end{equation*}
将扩张后的函数仍记为$F$, 定义泛函 
\begin{equation*}
    I(u) = \int_J\left(\frac{1}{2}|\dot u(t)|^2 + F(t, u(t))\right) \,{\rm d}t
\end{equation*}
以及定义域$M = u^* + H_0^1(J)$, 这里 
\begin{equation*}
    u^*(t) = \frac{a_0(t_1 - t) + a_1(t - t_0)}{t_1 - t_0}.
\end{equation*}
容易验证, $I$对应的E-L方程正是\eqref{40}. 在$H_0^1(J)$上赋予等价范数$v \mapsto \left(\int|\dot v|^2 \,{\rm d}x\right)^{1/2}$.
\begin{itemize}
    \item 显然$M$是弱序列闭的. 此外, $I$在$M$上还是强制的.
    \item 设$u_n$在$H^1(\Omega)$上弱收敛到$u$, 则$\{u_n\}$是有界集.
    注意到嵌入$H^1(J) \hookrightarrow C(\overline{J})$是紧的, 且$F$是连续的, 故$I$还是弱序列下半连续的(参考例\ref{ex2.30}中使用的方法). 
\end{itemize}
综上所述, $I$在$M$上有极小点$u^*$. 此时$L$满足如定理\ref{th2.42}中所列出的增长性限制, 且$(L_{p_ip_j})$为单位阵, 当然是正定的.
因此$u^* \in C^2$, 从而$u^*$是\eqref{40}的经典解.

\begin{example}[强迫振动的周期解]
    设$e$为周期为$T$, 且在$[0, T]$上的平均值为零的连续函数, 即 
    \begin{equation}\label{41}
        \int_0^Te(t) \,{\rm d}t = 0.
    \end{equation}
    求下列方程周期为$T$的解:
    \begin{equation*}
        \ddot u(t) + a\sin u(t) = e(t).
    \end{equation*}
\end{example}

记$H_{{\rm per}}^1(0, T)$为周期为$T$的$C^{\infty}$函数在$H^1(0, T)$中的闭包.
在其上考虑泛函 
\begin{equation*}
    I(u) = \int_0^T\left(\frac{1}{2}|\dot u(t)|^2 + a\cos u(t) + e(t)u(t)\right) \,{\rm d}t.
\end{equation*}
容易验证, $I$对应的E-L方程即为\eqref{42}式. 注意到$I$按照$H^1$范数并不是强制的, 故我们需要在$H_{{\rm per}}^1(0, T)$上选取其它的等价范数.
事实上, 若将$I$改写为
\begin{equation*}
    I(u) = \int_0^T\left(\frac{1}{2}|\dot u(t)|^2 + a\cos u(t) - E(t)u(t)\right) \,{\rm d}t,
\end{equation*}
其中 
\begin{equation*}
    E(t) = \int_0^te(\tau) \,{\rm d}\tau,
\end{equation*}
则$I$有估计 
\begin{equation*}
    I(u) \geq \frac{1}{2}\Vert \dot u \Vert_{L^2}^2 - |a| - C\Vert E \Vert_{L^{\infty}}\Vert \dot u \Vert_{L^2},
\end{equation*}
其中$C$是一个正常数. 由上式可以看出, 若能说明$u \mapsto \Vert \dot u \Vert_{L^2}$是$H_{{\rm per}}^1(0, T)$上的一个等价范数, 则$I$便是强制的.

\begin{lemma}[Wirtinger不等式]
    设$u \in H_{{\rm per}}^1(0, T)$. 若 
    \begin{equation*}
        \overline{u} = \int_0^Tu(t) \,{\rm d}t = 0,
    \end{equation*}
    那么 
    \begin{equation}\label{42}
        \int_0^T|\dot u(t)|^2 \,{\rm d}t \geq \frac{4\pi^2}{T^2}\int_0^T|u(t)|^2 \,{\rm d}t.
    \end{equation}
    \begin{proof}
        先对光滑函数证明\eqref{42}式. 由于$\overline{u} = 0$, 故$u$有Fourier展开 
        \begin{equation*}
            u(t) = \sum_{k = 1}^{\infty}\left(a_k\cos\frac{2\pi k}{T}t + b_k\sin\frac{2\pi k}{T}t\right).
        \end{equation*}
        从而 
        \begin{equation*}
            \dot u(t) = \frac{2\pi}{T}\sum_{k = 1}^{\infty}\left(-ka_k\sin\frac{2\pi k}{T}t + kb_k\cos\frac{2\pi k}{T}t\right).
        \end{equation*}
        再利用Parseval等式, 便有 
        \begin{align*}
            \int_0^T|\dot u(t)|^2 \,{\rm d}t = \frac{4\pi^2}{T^2}\sum_{k = 1}^{\infty}k^2(|a_k|^2 + |b_k|^2) \geq \frac{4\pi^2}{T^2}\sum_{k = 1}^{\infty}(|a_k|^2 + |b_k|^2) = \int_0^T|u(t)|^2 \,{\rm d}t.
        \end{align*}

        最后, 对任意的$u \in H_{{\rm per}}^1(0, T)$且满足$\overline{u} = 0$, 取一族周期为$T$的光滑函数$\{u_n\}$, 使得$u_n$在$H^1(0, T)$中收敛到$u$.
        注意到此时$\overline{u_n}$不一定等于零. 令$v_n = u_n - \overline{u_n}$, 则$\overline{v_n} = 0$.
        此外, 由H\"older不等式容易推出, $\overline{u_n} \rightarrow \overline{u} = 0$, 从而$v_n \rightarrow u$.
        利用第一段中证明的结果, 我们便在一般情况下证明了\eqref{42}式.
    \end{proof}
\end{lemma}

为使用Wirtinger不等式, 对任意的$u \in H_{{\rm per}}^1(0, T)$, 作分解$u = \widetilde{u} + \overline{u}$.
显然, $\overline{\widetilde{u}} = 0$. 此外, 注意到$I(u) = I(u + 2\pi)$, 这表明我们不需在整个$H_{{\rm per}}^1(0, T)$上考虑$I$, 而是取集合 
\begin{equation*}
    M = \{u = \xi + \eta\colon \xi \in H_{{\rm per}}^1(0, T), \overline{\xi} = 0, \eta \in [0, 2\pi]\},
\end{equation*}
并将$I$限制在$M$上. 这样做的好处是$\overline{u}$只在$[0, 2\pi]$上变化.

最后, 我们在$M$上考虑$I$. 显然$M$是弱序列闭的. 由上述分析可知, $I$在$M$上还是强制的.
此外, 容易验证$I$是弱序列下半连续的, 故由存在性定理可知, $I$在$M \subseteq H_{{\rm per}}^1$上存在极小点.
再利用正则性定理, 此极小点还是$C^2$的.

\subsection{专题一: 正交投影}

对于某类对应于线性微分方程的特殊变分问题, 其极小点的存在性可以由Hilbert空间的某些性质所保证, 而不需要使用弱拓扑的工具.

\begin{example}
    考虑Poisson方程
    \begin{equation*}
        \begin{cases}
            -\Delta u = f, \quad &\text{in}\ \Omega, \\ 
            u = 0, \quad &\text{on}\ \partial\Omega,
        \end{cases}
    \end{equation*}
    其中$\Omega \subseteq \mathbb{R}^n$为有界区域, $f \in L^2(\Omega)$.
    在等式$-\Delta u = f$两边与$\varphi \in H_0^1(\Omega)$相乘后在积分, 并利用Green公式和边值条件, 即得 
    \begin{equation*}
        (\varphi, u)_{H_0^1} = \int_{\Omega}\nabla u \cdot \nabla\varphi \,{\rm d}x = \int_{\Omega}f\varphi \,{\rm d}x.
    \end{equation*}
    由Cauchy-Schwarz不等式和Poincar\'e不等式, 有 
    \begin{equation*}
        \left|\int_{\Omega}f\varphi \,{\rm d}x\right| \leq \Vert f \Vert_{L^2}\Vert \varphi\Vert_{L^2} \leq C\Vert f \Vert_{L^2}\nabla\varphi\Vert_{L^2},
    \end{equation*}
    其中$C$是一个常数. 这表明$\varphi \mapsto \int_{\Omega}f\varphi \,{\rm d}x$可以看作是$H_0^1(\Omega)$上的有界线性泛函.
    按Riesz表示定理, 存在$u^* \in H_0^1(\Omega)$, 使得 
    \begin{equation*}
        \int_{\Omega}f\varphi \,{\rm d}x = (\varphi, u^*)_{H_0^1}.
    \end{equation*}
    从而对任意的$\varphi \in H_0^1(\Omega)$, 有$(\varphi, u)_{H_0^1} = (\varphi, u^*)_{H_0^1}$.
    故$u = u^* \in H_0^1(\Omega)$便是Poisson方程的解.
\end{example}

在上述的推导过程中, Riesz表示定理起到了决定性的作用, 而Riesz定理的证明依赖于下述正交投影定理:

\begin{theorem}[正交投影]
    设$M$为Hilbert空间$H$的闭子空间, 则对任意的$x \in H$, 存在$y \in M$, 使得$(x - y) \bot M$.
\end{theorem}

值得注意的是, Hilbert空间$H$中正交投影的存在性可以化归为一个变分问题:
\begin{align*}
    &{\rm min}\ \Vert x - z \Vert, \\ 
    &{\rm s.t.}\ z \in M.
\end{align*}
这里$x \in H$. 事实上, 若存在$z^* \in M$使得$\Vert x - z^* \Vert = \min_{z \in M}\Vert x - z \Vert$, 对任意的$y \in H$, 考虑定义在$B_r(0)$上的函数 
\begin{align*}
    g(t) = \Vert x - (ty + (1 - t)z^*) \Vert^2 = \Vert y - z^* \Vert^2t^2 - 2(x - z^*, y - z^*)t + \Vert x - z^*\Vert^2.
\end{align*}
这是一个关于$t$的二次函数, 它在$t = 0$处达到极小值. 从而有 
\begin{equation*}
    \dot g(0) = -2(x - z^*, y - z^*) = 0,
\end{equation*}
即$(x - z^*) \bot M$.

极小点$z^* \in H$的存在性也可以用初等方法证明. 具体地, 令$m = \inf_{z \in M}\Vert x - z \Vert$, 取极小化序列$\{z_j\} \subseteq M$, 使得 
\begin{equation*}
    \Vert z_j - x \Vert < m + \frac{1}{j}, \qquad \forall j.
\end{equation*}
对任意的$\varepsilon > 0$, 当$j, k$充分大时, 平行四边形法则给出 
\begin{align*}
    \Vert z_j - z_k\Vert^2 = 2(\Vert z_j - x\Vert^2 + \Vert z_j - x \Vert^2) - 4\left\Vert \frac{z_j + z_k}{2} - x \right\Vert^2 \leq 4(m + \varepsilon)^2 - 4m^2,
\end{align*}
由此表明$\{z_j\}$是Cauchy列, 从而存在$z^* \in H$, 使得$z_j \rightarrow z^*$.
又因为$M$是闭的, 故$z^* \in M$, 且有$\Vert z^* - x \Vert = m$.

综上所述, 利用Hilbert空间的完备性和几何性质, 我们验证了Dirichlet原理, 而不引入弱拓扑的概念.

%\begin{example}[变分不等式]
    %设$\Omega$为$\mathbb{R}^n$中的有界区域. 取$H_0^1(\Omega)$上的一个闭凸集 
    %\begin{equation*}
    %    C = \{u \in H_0^1(\Omega)\colon u(x) \leq \psi(x)\ {\rm a.e.}\}.
    %\end{equation*}
    %求解$v \in C$, 使得 
    %\begin{equation}\label{43}
    %    (u, v - u)_{H_0^1} - \int_{\Omega}f(v - u) \,{\rm d}x \geq 0, \qquad \forall v \in C.
    %\end{equation}

    %利用Riesz表示定理, \eqref{43}可转化为 
    %\begin{equation}\label{44}
    %    (u - u^*, v - u)_{H_0^1} \geq 0, \qquad \forall v \in C,
    %\end{equation}
    %这里$u^* \in H_0^1(\Omega)$, 而\eqref{44}中$u \in H_0^1(\Omega)$的存在性可以归结为变分问题 
    %\begin{equation}
    %    \min_{v \in C}\Vert u^* - v \Vert.
    %\end{equation}
    %至于\eqref{45}的解的存在性, 可以使用同样的方法得到.
%\end{example}

\subsection{专题二: 特征值问题}

设$\Omega \subseteq \mathbb{R}^n$是一个有界区域. 考虑边值问题(算子$-\Delta$的\textbf{特征值问题}) 
\begin{equation}\label{43}
    \begin{cases}
        -\Delta u = \lambda u \quad &{\rm in}\ \Omega, \\ 
        u = 0 \quad &{\rm on}\ \partial\Omega.
    \end{cases}
\end{equation}
问对于哪些$\lambda \in \mathbb{R}$, 上述特征值问题存在非零解? 
称对应着非零解的数$\lambda$为\textbf{谱}. 若这个非零解$u \in L^2(\Omega)$, 则称其为\textbf{特征函数}, 并称相应的$\lambda$为\textbf{特征值}.

利用约束变分方法, 我们有如下结果:

\begin{theorem}\label{th2.49}
    设$\Omega$为$\mathbb{R}^n$中的有界区域, 则方程\eqref{43}有一列特征值$\{\lambda_k\}$, 且满足 
    \begin{itemize}
        \item $0 < \lambda_1 \leq \lambda_2 \leq \cdots \leq \lambda_i \leq \cdots$;
        \item $\lambda_i \rightarrow +\infty$,
    \end{itemize}
    以及对应的特征函数$\{\varphi_i\} \subseteq H_0^1(\Omega)$, 且满足 
    \begin{itemize}
        \item $-\Delta\varphi_j = \lambda_j\varphi_j, \Vert \varphi_j \Vert_{L^2} = 1, \forall i = 1, 2, \cdots$;
        \item $(\varphi_j, \varphi_k)_{H_0^1} = 0 = (\varphi_j, \varphi_k)_{L^2}, \forall i \neq j$.
    \end{itemize}
    进一步地, $\{\varphi_i/\sqrt{\lambda_i}\}$是$H_0^1(\Omega)$中的一组正规正交基, 从而$\{\varphi_i\}$是$L^2(\Omega)$中的一组正规正交基.
    \begin{proof}
        考虑泛函 
        \begin{equation*}
            I(u) = \int_{\Omega}|\nabla u|^2 \,{\rm d}x
        \end{equation*}
        以及 
        \begin{equation*}
            N(u) = \int_{\Omega}|u|^2 \,{\rm d}x.
        \end{equation*}
        容易验证, 若$\varphi_1 \in C^2(\overline{\Omega})$是变分问题 
        \begin{equation}\label{44}
            \min_{u \in H_0^1(\Omega) \cap N^{-1}(1)}I(u)
        \end{equation}
        的极小点, 则由Lagrange乘子法可知, 存在$\lambda_1 \in \mathbb{R}$满足$-\Delta\varphi_1 = \lambda_1\varphi_1$.
        以下验证\eqref{44}解的存在性. 显然$I$是强制的, 并且是弱序列下半连续的.
        结合存在性和正则性定理, 以下只需验证集合 
        \begin{equation*}
            M = H_0^1(\Omega) \cap N^{-1}(1)
        \end{equation*}
        是弱序列闭的. 具体地, 设$\{u_j\}$在$H_0^1$中收敛到$u$, 注意到紧嵌入$H_0^1(\Omega) \hookrightarrow L^2(\Omega)$, 故存在子列$\{u_{j'}\} \subseteq \{u_j\}$, 使得$u_{j'}$在$L^2$中收敛到$u$.
        由于$\Vert u_{j'} \Vert_{L^2} = 1$, 从而有$\Vert u \Vert_{L^2} = 1$.
        这表明$u \in M_1$, 故$M_1$是弱序列闭的. 

        再取集合 
        \begin{equation*}
            M_2 = \left\{u \in M_1\colon \int_{\Omega}u\varphi_1 \,{\rm d}x = 0\right\},
        \end{equation*}
        并考虑变分问题$\min_{u \in M_2}I(u)$. 若$\varphi_2 \in C^2$是极小点, 则存在$\lambda_2, \mu_2 \in \mathbb{R}$使得 
        \begin{equation}\label{45}
            -\Delta\varphi_2 = \lambda_2\varphi_2 + \mu_2\varphi_1.
        \end{equation} 
        一方面, 在等式$-\Delta\varphi_1 = \lambda_1\varphi_1$两边乘以$\varphi_2$再积分, 并利用Green公式, 即得 
        \begin{equation*}
            \int_{\Omega}\nabla\varphi_1 \cdot \nabla\varphi_2 \,{\rm d}x = -\int_{\Omega}\Delta\varphi_1\varphi_2 \,{\rm d}x = \lambda_1\int_{\Omega}\varphi_1\varphi_2 \,{\rm d}x = 0;
        \end{equation*}
        另一方面, 在\eqref{45}等式两侧乘以$\varphi_1$后积分, 得 
        \begin{equation*}
            \int_{\Omega}\nabla\varphi_1 \cdot \varphi_2 \,{\rm d}x = \mu_2\int_{\Omega}|\varphi_1|^2 \,{\rm d}x = \mu_2.
        \end{equation*}
        故$\mu_2 = 0$. 这表明$\lambda_2$是一个特征值, $\varphi_2$是相应的特征函数, $\varphi_1 \neq \varphi_2$, 且 
        \begin{equation*}
            \lambda_2 = I(\varphi_2) = \min\{I(u)\colon u \in M_2\} \geq \lambda_1.
        \end{equation*}
        以下只需要说明$\varphi \in C^2$的极小点. 同理, 我们只需说明$M_2$是弱序列闭的, 而这一点可以由紧嵌入定理保证.

        如此进行下去, 在第$i$步时, 令 
        \begin{equation*}
            M_i = \left\{u \in M_{i - 1}\colon \int_{\Omega}u\varphi_{i - 1} \,{\rm d}x = 0\right\}.
        \end{equation*}
        可以证明它是弱序列闭的, 从而变分问题$\min_{u \in M_i}I(u)$有解$\varphi_i \neq 0$, 满足 
        \begin{equation*}
            -\Delta\varphi_i = \lambda_i\varphi_i + \sum_{j = 1}^{i - 1}\mu_j\varphi_j.
        \end{equation*}
        同理, 我们有$\mu_1 = \cdots = \mu_{i - 1} = 0$, 由此即表明$\lambda_i$是一个特征值, $\varphi_i$是对应的特征函数, 且 
        \begin{equation*}
            \lambda_i = I(\varphi_i) = \min\{I(u)\colon u \in M_i\} \geq \lambda_{i - 1}.
        \end{equation*}

        以下说明$\lambda_i \rightarrow +\infty$. 若存在$C > 0$使得$\lambda_i \leq C, \forall i$, 那么 
        \begin{equation*}
            \int_{\Omega}|\nabla\varphi_i|^2 \,{\rm d}x = \lambda_i\int_{\Omega}|\varphi_i|^2 \,{\rm d}x = \lambda_i \leq C, \qquad \forall i.
        \end{equation*}
        由此表明$\{\varphi_i\}$是$H_0^1(\Omega)$中的有界列, 从而有弱收敛子列$\{\varphi_{j'}\}$.
        一方面, 根据紧嵌入定理, 有子列$\{\varphi_{j''}\}$在$L^2$中收敛到$\varphi$; 另一方面, 注意到$(\varphi_j, \varphi_k)_{L^2} = 0, j \neq k$, 故有 
        \begin{equation*}
            \int_{\Omega}|\varphi_j - \varphi_k|^2 \,{\rm d}x = \int_{\Omega}(\varphi_j^2 + \varphi_k^2) \,{\rm d}x = 2, \qquad i \neq j.
        \end{equation*}
        从而有 
        \begin{equation*}
            \int_{\Omega}|\varphi^* - \varphi_{j'}|^2 \,{\rm d}x = 2,
        \end{equation*}
        矛盾.

        最后证明$\{\varphi_i\}$是完备的, 即对任意的$u \in H_0^1(\Omega)$, 部分和
        \begin{equation*}
            s_m = \sum_{i = 1}^mc_i\varphi_i
        \end{equation*}
        在$H_0^1$中收敛到$u$, 这里$c_i = (u, \varphi_i)_{H_0^1}$. 一方面, 由正交性直接计算得 
        \begin{align*}
            \int_{\Omega}|\nabla(u - s_m)|^2 \,{\rm d}x = \int_{\Omega}|\nabla u|^2 \,{\rm d}x - \sum_{n = 1}^m|c_n|^2.
        \end{align*}
        另一方面, 注意到$u - s_m \in M_{m + 1}$, 故有 
        \begin{equation*}
            \int_{\Omega}|\nabla(u - s_m)|^2 \,{\rm d}x \geq \lambda_{m + 1}\int_{\Omega}|u - s_m|^2 \,{\rm d}x.
        \end{equation*}
        因此 
        \begin{equation*}
            \int_{\Omega}|u - s_m|^2 \,{\rm d}x \leq \frac{1}{\lambda_{m + 1}}\int_{\Omega}|\nabla u|^2 \,{\rm d}x \rightarrow 0 \qquad (m \rightarrow \infty),
        \end{equation*}
        即$s_m$在$L^2$意义下收敛到$u$. 进一步地, 由Bessel不等式可知, 当$m > n$时, 有 
        \begin{equation*}
            \int_{\Omega}|\nabla(s_m - s_n)|^2 \,{\rm d}x = \sum_{i = n + 1}^m|c_i|^2 \rightarrow 0 \qquad (n \rightarrow +\infty).
        \end{equation*}
        结合上述分析可知, $s_m$也在$H_0^1$中收敛到$u$.
    \end{proof}
\end{theorem}

\begin{example}
    在区间$J = [a, b] \subseteq \mathbb{R}$上给定$p \in C^1(J), q \in C(J)$, 并假设存在$\alpha \in \mathbb{R}$使得$p(x) \geq \alpha$, 且$q(x) \geq 0$.
    考虑具有Dirichlet边界条件的Sturm-Liouville问题 
    \begin{equation*}
        \begin{cases}
            \mathcal{L}u = -(pu')' + qu = \lambda u \quad &{\rm in}\ J, \\ 
            u(a) = u(b) = 0.
        \end{cases}
    \end{equation*}
    和定理\ref{th2.49}中的讨论过程类似, 在空间$H_0^1(J)$上考察泛函 
    \begin{equation*}
        I(u) = \frac{1}{2}\int_J(p|u'|^2 + q|u|^2) \,{\rm d}x,
    \end{equation*}
    此时$H_0^1(\Omega)$上的内积规定为 
    \begin{equation*}
        (u, v) = \int_J(pu'v' + quv) \,{\rm d}x.
    \end{equation*}
    通过逐次引入约束$M_1 = \{u \in H_0^1(J)\colon \Vert u \Vert^2 = 1\}$, $M_2 = \{u \in M_1\colon (u, \varphi_1) = 0\} \cdots$, 同样可以得到一列恒正递增区域无穷的特征值$\{\lambda_i\}$, 以及对应的特征函数$\{\varphi_i\} \subseteq C^2(J)$.
    此外, $\{\varphi_i\}$也是$H_0^1(\Omega)$中的一组正交基.
\end{example}

\begin{remark}
    上述讨论均涉及的是Dirichlet边界的情形. 对于Neumann边界, 由于Poincar\'e不等式在$H^1(\Omega)$上不再成立, 对应的泛函未必是强制的.
    然而, 注意到如下结果:
    \begin{proposition}[Poincaré不等式]
        设$\Omega$为$\mathbb{R}^n$中的有界区域, 其边界是$C^1$的. 那么对任意的$u \in W^{1, p}(\Omega)\ (1 \leq p \leq \infty)$, 存在常数$C = C(n, p, \Omega) > 0$, 使得 
        \begin{equation*}
            \left\Vert u - \int_{\Omega}u \,{\rm d}x \right\Vert_{L^p} \leq C\Vert \nabla u \Vert_{L^p}.
        \end{equation*}
        由此可知, 我们可以将对应的泛函限制到子空间 
        \begin{equation*}
            X = \left\{u \in H^1(\Omega)\colon \int_{\Omega}u \,{\rm d}x = 0\right\}
        \end{equation*}
        上, 此时Poincaré不等式在$X$上仍成立, 我们可以在$X$上(相当于添加一个约束条件, 其对应的Lagrange乘子事实上为零)求对应泛函的极小值.
    \end{proposition}
\end{remark}

在前述的讨论中, 方程\eqref{43}的特征值$\{\lambda_i\}$是通过递归给出的.
下述定理直接给出了$\lambda_i$的刻画:

\begin{theorem}[Courant极小极大定理]
    记$\{\lambda_n\}$为方程\eqref{43}的特征值, 则有 
    \begin{equation*}
        \lambda_n = \max_{E_{n - 1}}\min_{u \in E_{n - 1}^{\bot} \smallsetminus \{0\}}\frac{\displaystyle\int_{\Omega}|\nabla u|^2 \,{\rm d}x}{\displaystyle \int_{\Omega}|u|^2 \,{\rm d}x},
    \end{equation*}
    其中$E_{n - 1}$是$H_0^1(\Omega)$中任意$(n - 1)$维线性子空间.
    \begin{proof}
        记$E_{n - 1} = {\rm span}\{v_1, \cdots, v_n\}$是$H_0^1(\Omega)$中任一$(n - 1)$维线性子空间, 其中$\{v_n\} \subseteq H_0^1(\Omega)$线性无关.
        再记 
        \begin{equation*}
            \mu(E_{n - 1}) = \min_{u \in E_{n - 1}^{\bot} \smallsetminus \{0\}}\frac{\displaystyle\int_{\Omega}|\nabla u|^2 \,{\rm d}x}{\displaystyle \int_{\Omega}|u|^2 \,{\rm d}x}.
        \end{equation*}
        一方面, 由于$\lambda_1 \leq \lambda_2 \leq \cdots \leq \lambda_n$, 
        \begin{equation*}
            \mu(E_{n - 1}) \leq \frac{\displaystyle\int_{\Omega}|\nabla u|^2 \,{\rm d}x}{\displaystyle \int_{\Omega}|u|^2 \,{\rm d}x} = \frac{\displaystyle\sum_{i = 1}^n\lambda_i|c_i|^2}{\displaystyle\sum_{i = 1}^n|c_i|^2} \leq \lambda_n;
        \end{equation*}
        另一方面, 令$\{\varphi_1, \cdots, \varphi_n\}$为前$n$个特征函数, 注意到 
        \begin{equation*}
            (E_{n - 1}^{\bot} \smallsetminus \{0\}) \cap {\rm span}\{\varphi_1, \cdots, \varphi_n\} \neq \varnothing,
        \end{equation*}
        取$\widetilde{E_{n - 1}} = {\rm span}\{\varphi_1, \cdots, \varphi_{n - 1}\}$, 便有 
        \begin{equation*}
            \max_{E_{n - 1}}\mu(E_{n - 1}) \geq \mu(\widetilde{E_{n - 1}}) = \lambda_n.
        \end{equation*}
    \end{proof}
\end{theorem}

\subsection{专题三: Ekeland变分原理}

下述的Ekeland变分原理给出了选取一串近似极小点的方法. 此外, 若将其与近代变分学常用的紧性条件Palais-Smale条件结合起来, 则提供了一个求解更广意义下的极小点(即临界点)的方法. 

\begin{theorem}[Ekeland]
    设$(X, d)$是一个完备的度量空间. 又设$f\colon X \rightarrow (-\infty, +\infty]$, 且$f \not\equiv +\infty$.
    若$f$是下方有界且下半连续的, 则对任意的$\varepsilon > 0$和$x_{\varepsilon} \in X$使得$f(x_{\varepsilon}) < \inf_X f + \varepsilon$, 存在$y_{\varepsilon} \in X$, 使得 
    \begin{itemize}
        \item $f(y_{\varepsilon}) \leq f(x_{\varepsilon})$; 
        \item $d(y_{\varepsilon}, x_{\varepsilon}) \leq 1$;
        \item $f(x) > f(y_{\varepsilon}) - \varepsilon d(y_{\varepsilon}, x), \forall x \in X \smallsetminus \{y_{\varepsilon}\}$.
        即对于给定的$y_{\varepsilon}$, 函数$x \mapsto f(x) + \varepsilon d(y_{\varepsilon}, x)$以$y_{\varepsilon}$为严格极小点.
    \end{itemize}
    \begin{proof}
        先选择$u_0 = x_{\varepsilon}$. 假设$u_i$已经选定, 令 
        \begin{equation*}
            S_n = \{x \in X\colon f(x) \leq f(u_n) - \varepsilon d(x, u_n)\}.
        \end{equation*}
        显然$S_n$是非空的. 今选取$u_{i + 1} \in S_n$满足 
        \begin{equation}\label{46}
            f(u_{i + 1}) - \inf_{S_n}f \leq \frac{1}{2}\left(f(u_i) - \inf_{S_n}f\right).
        \end{equation}
        由此我们便得到了$X$中的一组序列$\{u_i\}$. 此外, 由$\{u_i\}$的构造方式可知, 
        \begin{equation*}
            \varepsilon d(x, u_i) \leq f(u_i) - f(u_{i + 1}), \qquad \forall i = 0, 1, \cdots.
        \end{equation*}
        将上述不等式累加, 并利用三角不等式, 即得 
        \begin{equation}\label{47}
            \varepsilon d(u_i, u_j) \leq f(u_i) - f(u_j), \qquad \forall j \geq i.
        \end{equation}
        注意到$\{f(u_i)\}$是递减的, 故上式表明$\{u_i\}$是Cauchy列. 记其极限为$u^*$, 以下验证$u^*$满足定理中列出的三条性质.

        首先, 由于$\{f(u_i)\}$是递减的, 从而有 
        \begin{equation*}
            f(u^*) \leq f(u_i) \leq f(u_0) = f(x_{\varepsilon}).
        \end{equation*}
        其次, 由\eqref{47}可得, 
        \begin{align*}
            \varepsilon d(x_{\varepsilon}, u^*) \leq f(x_{\varepsilon}) - f(u^*) \leq f(x_{\varepsilon}) - \inf_Xf < \varepsilon,
        \end{align*}
        即$d(x_{\varepsilon}, u^*) \leq 1$. 最后利用反证法证明$u^*$满足第三条性质. 设存在$x \neq u^*$使得 
        \begin{equation}\label{48}
            f(x) \leq f(u^*) - \varepsilon d(u^*, x).
        \end{equation} 
        首先注意到\eqref{47}给出不等式 
        \begin{equation}\label{49}
            f(u^*) \leq f(u_i) - \varepsilon d(u_i, u^*).
        \end{equation}
        联立\eqref{48}和\eqref{49}, 即有 
        \begin{equation*}
            f(x) \leq f(u_i) - \varepsilon d(u^*, x), \qquad \forall i.
        \end{equation*}
        这表明$x \in S_i, \forall i$. 在\eqref{46}不等式两侧取下极限, 并利用$f$的下半连续性, 我们有 
        \begin{equation*}
            f(u^*) \leq \varliminf\limits_{i \rightarrow \infty}\inf_{S_n}f \leq f(x)
        \end{equation*}
        这与\eqref{48}矛盾.
    \end{proof}
\end{theorem}

\begin{corollary}
    设$(X, d)$是一个完备的度量空间. 又设$f\colon X \rightarrow (-\infty, +\infty]$是下方有界和下半连续的, 且$f \not\equiv +\infty$.
    那么对任意的$\varepsilon > 0$, 存在$y_{\varepsilon} \in S$, 使得对任意的$x \neq y$, 有$f(x) > f(y_{\varepsilon}) - \varepsilon d(x, y_{\varepsilon})$.
\end{corollary}

以下引入Palais-Smale条件. 首先回顾一些概念:

\begin{definition}
    设$f$是Banach空间$X$上的实值函数, $x_0 \in U \subseteq X$, 其中$U$是一个开集.
    称$f$在$x_0$处是\textbf{Fr\'echet可微}的, 如果存在$\xi \in X^*$, 使得 
    \begin{equation*}
        |f(x) - f(x_0) - \langle \xi, x - x_0\rangle| = o(\Vert x - x_0\Vert) \qquad (x \rightarrow x_0).
    \end{equation*}
    称$\xi$为$f$在$x_0$处的\textbf{Fr\'echet导数}, 记为$f'(x_0)$.

    若Fr\'echet导数$f'(x)$处处存在, 并且$x \mapsto f'(x)$是连续的, 那么称$f$是连续可微的, 记作$f \in C^1$.
\end{definition}

容易看出, G\^ateaux导数和Fr\'echet导数分别是欧氏空间中方向导数和全微分在Banach空间中的推广.
此外, 若$f$有Fr\'echet导数$f'(x_0)$, 那么它必有G\^ateaux导数${\rm d}f(x_0, h)$, 且 
\begin{equation*}
    {\rm d}f(x_0, h) = \langle f'(x_0), h\rangle, \qquad \forall h \in X.
\end{equation*}
同时有
\begin{equation*}
    \Vert f'(x_0) \Vert = \sup_{0 \neq h \in X}\frac{|{\rm d}f(x_0, h)|}{\Vert h \Vert}
\end{equation*}
反之, 设$f$在$x_0$的一个邻域$U$内处处有G\^ateaux导数${\rm d}f(x, h)\ (x \in U)$, 并且有$\xi = \xi(x) \in X^*$满足 
\begin{equation*}
    \langle \xi(x), h\rangle = {\rm d}f(x, h), \qquad \forall x \in U, h \in X.
\end{equation*}
如果$x \mapsto \xi(x)$是连续的, 那么$f$在$x_0$处有Fr\'echet导数$f'(x_0)$.

以下定义是对经典极小点概念的推广:

\begin{definition}
    称满足条件$f'(x_0) = 0$的点$x_0$为\textbf{临界点}, 并称相应的函数值$f(x_0)$为\textbf{临界值}.
\end{definition}

由此可以看出, 在变分问题中, 极小点便是临界点, 而一切临界点都是E-L方程的解.

\begin{definition}
    设$X$是一个Banach空间, $f \in C^1(X)$. 若对任一满足条件 
    \begin{equation}\label{50}
        f(x_i) \rightarrow c, \qquad \Vert f'(x_i)\Vert \rightarrow 0
    \end{equation}
    的序列$\{x_i\} \subseteq X$都有收敛的子列, 那么称$f$在$c$处满足\textbf{Palais-Smale条件}, 记作${\rm PS_c}$.
    此外, 称满足\eqref{50}的序列为一个\textbf{Palais-Smale序列}(简称\textbf{PS序列}).
\end{definition}

\begin{corollary}
    设$X$是一个Banach空间. 设$f \in C^1(X)$, 并且是下方有界的. 记$c = \inf_Xf$.
    若$f$满足${\rm PS_c}$, 那么$f$能达到极小值.
    \begin{proof}
        根据Ekeland变分原理, 我们可以选取$X$中的一组序列$\{x_i\}$, 满足条件 
        \begin{equation*}
            \begin{cases}
                \displaystyle f(x) > f(x_i) - \frac{1}{i}\Vert x - x_i\Vert \qquad (\forall x \neq x_i), \\ 
                \displaystyle c \leq f(x_i) < c + \frac{1}{i}.
            \end{cases}
            \qquad \forall i.
        \end{equation*}
        第一个不等式表明 
        \begin{equation*}
            \Vert f'(x_i) \Vert \leq \frac{1}{i} \rightarrow 0 \qquad (i \rightarrow +\infty).
        \end{equation*}
        第二个不等式表明$f(x_i) \rightarrow c$. 按${\rm PS_c}$, $\{x_i\}$有收敛子列$\{x_{j_i}\}$.
        记其极限为$x^*$. 由$f$的连续性, 便有$f(x^*) = c = \inf_Xf$.
    \end{proof}
\end{corollary}

最后我们用Ekeland变分原理来推导一个最基本的临界点定理, 即山路定理.

问题的提出: 在一个四面环山的盆地, 从山外地面上一点$p_1$出发想要进入盆地中的一点$p_0$.
人们希望走的山路是这样一条连接$p_0$和$p_1$的道路, 其最高点不高于任何临近道路的最高点.
这条山路上的最高点未必是极值点, 一个自然的问题是: 该最高点是否是临界点?

具体地, 设$\Omega$是$\mathbb{R}^n$中的有界区域. 给定$p_0 \in \Omega, p_1 \in X \smallsetminus \overline{\Omega}$.
设函数$f \in C^1(X)$满足 
\begin{equation}\label{51}
    \alpha = \inf_{\partial\Omega}f(x) > \max\{f(p_0), f(p_1)\}.
\end{equation}
令 
\begin{equation}\label{52}
    \Gamma = \{\gamma \in C[0, 1]\colon \gamma(i) = p_i, i = 0, 1\}
\end{equation}
以及 
\begin{equation}\label{53}
    c = \inf_{\gamma \in \Gamma}\sup_{t \in [0, 1]}f \circ \gamma(t).
\end{equation}
问$c$是否是$f$的临界值? 即是否存在$x \in X$, 使得$f'(x_0) = 0$以及$f(x_0) = c$?
下述定理回答了这个问题:

\begin{theorem}[山路定理]
    设$X$是一个Banach空间, $f \in C^1(X)$. 又设$\Omega \subseteq X$是一个有界区域.
    给定$p_0 \in \Omega, p_1 \in X \smallsetminus \overline{\Omega}$满足\eqref{51}, 又按\eqref{52}和\eqref{53}定义$c$.
    若$f$满足${\rm PS_c}$, 那么$c \geq \alpha$是$f$的一个临界值.
    \begin{proof}
        在$\Gamma$上引入度量 
        \begin{equation*}
            d(\gamma_1, \gamma_2) = \max_{t \in [0, 1]}\Vert \gamma_1(t) - \gamma_2(t)\Vert.
        \end{equation*}
        容易验证$(\Gamma, d)$是一个完备的度量空间. 令 
        \begin{equation*}
            I(\gamma) = \max_{t \in [0, 1]}f \circ \gamma(t).
        \end{equation*}
        由假设条件可知, $I \geq \alpha$, 即$I$是下方有界的. 此外, 注意到 
        \begin{align*}
            |I(\gamma_1) - I(\gamma_2)| &\leq \max_{t \in [0, 1]}|f \circ \gamma_1(t) - f \circ \gamma_2(t)| \\ 
            &\leq \max_{t \in [0, 1]}\Vert \dot f(\theta\gamma_1(t) + (1 - \theta)\gamma_2(t))\Vert \Vert \gamma_1(t) - \gamma_2(t)\Vert \\ 
            &\leq Cd(\gamma_1, \gamma_2),
        \end{align*}
        故$I$是Lipschitz连续的, 自然也是下半连续的. 应用Ekeland变分原理于$I$, 我们可以选取$\Gamma$中的一组序列$\{\gamma_i\}$, 使得对任意的$i$, 有 
        \begin{gather}
            c \leq I(\gamma_i) < c + \frac{1}{i}, \label{54} \\ 
            I(\gamma) > I(\gamma_i) - \frac{1}{i}d(\gamma, \gamma_i) \qquad (\gamma \neq \gamma_i) \label{55},
        \end{gather}
        现令 
        \begin{equation*}
            M(\gamma) = \{t \in [0, 1]\colon f \circ \gamma(t) = I(\gamma)\}.
        \end{equation*}
        容易验证, $M \subseteq (0, 1)$是非空紧集. 记
        \begin{equation*}
            \Gamma_0 = \{\gamma \in C[0, 1]\colon \gamma(i) = 0, i = 0, 1\}.
        \end{equation*}
        对任意的$h \in C[0, 1]$, 任取一列递减趋于零的序列$\{\lambda_j\}$以及序列$\{\xi_j\} \subseteq M(\gamma_i + \lambda_jh)$, 由\eqref{54}可得 
        \begin{equation*}
            \lambda_j^{-1}(f \circ (\gamma_i + \lambda_jh)(\xi_j) - f \circ \gamma_i(\xi_j)) \geq -\frac{1}{i}.
        \end{equation*}
        由于$\{\xi_j\}$, 故存在收敛子列. 设收敛子列的极限为$\eta_i$, 从而有 
        \begin{equation}\label{56}
            {\rm d}f(\gamma_i(\eta_i), h(\eta_i)) \geq -\frac{1}{i}.
        \end{equation}
        若能证明存在$\eta_i^* \in M(\gamma_i)$, 使得 
        \begin{equation}\label{57}
            {\rm d}f(\gamma(\eta_i^*), x) \geq -\frac{1}{i}, \qquad \forall x \in X, \Vert x \Vert = 1,
        \end{equation}
        令$x_i = \gamma_i(\eta_i^*)$, 从而有
        \begin{gather*}
            c \leq f(x_i) = f \circ \gamma_i(\eta_i^*) < c + \frac{1}{i}, \\ 
            \Vert f'(x_i)\Vert = \sup_{\Vert x \Vert = 1}|{\rm d}f(x_i, x)| \leq \frac{1}{i}. 
        \end{gather*}
        注意到$f$满足${\rm PS}_c$, 故$\{x_i\}$有收敛子列. 若记子列的极限为$x^*$, 显然有$f'(x^*) = 0$, 且$f(x^*) = c$.
        即$c$是$f$的一个临界值.

        以下证明\eqref{57}. 若不存在$\eta_i^*$使得\eqref{57}成立, 则对任意的$\eta \in M(\gamma_i)$, 存在$y_{\eta} \in X, \Vert y_{\eta}\Vert = 1$, 使得 
        \begin{equation*}
            {\rm d}f(\gamma_i(\eta), y_{\eta}) < -\frac{1}{i}.
        \end{equation*} 
        根据$f$的连续性, 存在$\eta$的一个邻域$O_{\eta} \subseteq (0, 1)$, 使得 
        \begin{equation*}
            {\rm d}f(\gamma_i(\xi), y_{\eta}) < -\frac{1}{i}, \qquad \forall \xi \in O_{\eta}.
        \end{equation*}
        注意到$M(\gamma_i)$是紧的, 故开覆盖$\{O_{\eta}\} \supseteq M(\gamma_i)$存在有限子覆盖$\{O_{\eta_j}\}_{j = 1}^m$, 其对应着$v_{\eta_j} \in X$, 满足$\Vert y_{\eta_i}\Vert = 1$, 且 
        \begin{equation*}
            {\rm d}f(\gamma_i(\xi), y_{\eta_i}) < -\frac{1}{i}, \qquad \forall \xi \in O_{\eta_j}, j = 1, \cdots, m.
        \end{equation*}
        现取$\{O_{\eta_j}\}$对应的一组单位分解$\{\rho_j\}$, 并令
        \begin{equation*}
            y = y(\xi) = \sum_{j = 1}^m\rho_j(\xi)y_{\eta_j},
        \end{equation*}
        从而有 
        \begin{equation*}
            {\rm d}f(\gamma_i(\xi), y(\xi)) < -\frac{1}{i}, \qquad \forall \xi \in M(\gamma_i),
        \end{equation*}
        这与\eqref{56}矛盾.
    \end{proof}
\end{theorem}

\begin{example}
    在山路定理中, 若$f$不满足Palais-Smale条件, 则定理结论可能不成立. 以下反例来自Brezis和Nirenberg.
    
    在$\mathbb{R}^2$上考虑函数 
    \begin{equation*}
        f(x, y) = x^2 + (1 - x)^3y^2
    \end{equation*}
    令$\Omega = B_{1/2}(0)$, $c = \inf_{\partial\Omega}f > 0$. 显然$f(0, 0) = 0, f(4, 1) = -11$.
    但直接验证可知, $f$只有一个临界点$(0, 0)$.
\end{example}

最后, 我们利用山路定理来求解变分问题.

\begin{example}
    设$a$是$\mathbb{R}$上周期为$T$的连续函数. 定义函数 
    \begin{equation*}
        V(t, x) = -\frac{1}{2}|x|^2 + \frac{a(t)}{p + 1}|x|^{p + 1}
    \end{equation*}
    假设$p > 1, a(t) \geq \alpha > 0$. 求满足方程 
    \begin{equation}\label{58}
        \ddot x + \partial_xV(t, x) = 0
    \end{equation}
    的非平凡$C^2$周期解.

    在$H_{{\rm per}}^1(0, T)$上定义泛函 
    \begin{equation*}
        I(x) = \int_0^T\left(\frac{1}{2}(|\dot x|^2 + |x|^2) - \frac{a(t)}{p + 1}|x|^{p + 1}\right) \,{\rm d}t.
    \end{equation*}
    而\eqref{58}是$I$对应的E-L方程. 为使用山路定理, 取$p_0 = 0$, $p_1 = \lambda\xi\sin\frac{2\pi}{T}t$, 其中$\xi = (1, 1, \cdots, 1)$, $\lambda > 0$.
    事实上, 直接计算得$I(p_0) = 0$, 且 
    \begin{equation*}
        I(p_1) = O(\lambda^2) - O(\lambda^{p + 1}),
    \end{equation*}
    从而我们可以选取充分大的$\lambda > 0$, 使得$I(p_1) < 0$. 定义 
    \begin{equation*}
        \Gamma = \{\gamma \in C[0, 1]\colon \gamma(i) = p_i, i = 0, 1\}
    \end{equation*}
    以及 
    \begin{equation*}
        c = \inf_{\gamma \in \Gamma}\sup_{t \in [0, 1]}I \circ \gamma(t).
    \end{equation*}
    以下验证$f$满足${\rm PS_c}$. 具体地, 设$\{x_i\}$是一个${\rm PS}$序列, 即满足条件 
    \begin{equation*}
        \begin{cases}
            I(x_i) \rightarrow c, \\ 
            \Vert I'(x_i)\Vert = \sup_{\Vert x \Vert = 1}|{\rm d}I(x_i, x)| \rightarrow 0.
        \end{cases}
    \end{equation*}
    直接计算得 
    \begin{equation*}
        \begin{cases}
            I(x_i) = \int_0^T\left(\frac{1}{2}(|\dot x_i|^2 + |x_i|^2) - a(t)\frac{|x_i|^{p + 1}}{p + 1}\right) \,{\rm d}t \rightarrow c, \\
            {\rm d}I(x_i, x_i) = \int_0^T(|\dot x_i|^2 + |x_i|^2 - a(t)|x_i|^{p + 1}) \,{\rm d}t = o(\Vert x_i\Vert).
        \end{cases}
    \end{equation*}
    从而有
    \begin{equation*}
        \left(\frac{1}{2} - \frac{1}{p + 1}\right)\int_0^T(|\dot x_i|^2 + |x_i|^2) \,{\rm d}t = C + o(\Vert x_i \Vert),
    \end{equation*}
    其中$C$是一个常数. 上式表明$\{x_i\}$是有界序列, 从而存在子列, 仍记为$\{x_i\}$, 使得$x_i \rightharpoonup x^*\ {\rm in}\ H_{{\rm per}}^1(0, T)$.
    特别地, 结合${\rm PS}$序列的假设, 我们有 
    \begin{equation*}
        \int_0^Ta(t)|x_i|^{p - 1}x_i\varphi \,{\rm d}t \rightarrow \int_0^T(\dot x^*\dot\varphi + x^*\varphi) \,{\rm d}t = \int_0^T\left(\frac{{\rm d}^2}{{\rm d}t^2} + 1\right)x^*\varphi \,{\rm d}t.
    \end{equation*}
    此外, 直接利用分部积分公式可得
    \begin{equation*}
        \int_0^T\left(\left(\frac{{\rm d}^2}{{\rm d}t^2} + 1\right)x_i - a(t)|x_i|^{p - 1}x_i\right)\varphi \,{\rm d}t \rightarrow 0,
    \end{equation*}
    综上分析, 我们有$x_i \rightarrow x\ {\rm in}\ H_{{\rm per}}^1(0, T)$, 故$f$满足${\rm PS_c}$.
    由山路定理, 方程\eqref{58}有非平凡周期解$u^*$. 再由正则性定理, 此解还是$C^2$的.
\end{example}

\subsection{专题四: 对偶最小作用量原理}

Hamilton方程组对应的泛函不是下方有界的, 故不能采用直接求极值的方法. 然而, 若$H$是凸的, 我们可以通过Legendre变换将问题转化.
这便是\textbf{对偶最小作用量原理}.

\subsubsection{凸分析初步: Legendre变换}

首先回顾一些关于Legendre变换的基本概念. 在第五节中, 我们主要考虑的是$C^1$函数的情形.
在本节中, 我们集中于$\mathbb{R}^n$上的Legendre变换, 而对函数的可微性不做进一步的要求.
具体地, 设$f\colon \mathbb{R}^n \rightarrow (-\infty, +\infty]$, 且$f \not\equiv +\infty$ (此时称$f$为\textbf{真}函数).
定义 
\begin{equation*}
    \boxed{f^*(\xi) = \sup_{x \in \mathbb{R}^n}[(\xi, x)_{\mathbb{R}^n} - f(x)]}.
\end{equation*}
称$f^*$为$f$的\textbf{Legendre变换.}


\begin{proposition}
    设$f\colon \mathbb{R}^n \rightarrow (-\infty, +\infty]$是真函数. 记$f^*$为$f$的Legendre变换.
    那么 
    \begin{enumerate}
        \item $f^*$是下半连续的凸函数;
        \item 若$f$是真的, 下半连续的凸函数, 则$f^*$是真函数;
        \item 若$f \leq g$, 则$g^* \leq f^*$; \label{prop2.63-3}
        \item (Young不等式) $(x, \xi) \leq f(x) + f^*(\xi)$, 其中等号成立当且仅当$\xi \in \partial f(x)$. \label{prop2.63-4}
        这里$\partial f(x)$为$f$的在$x$处的次微分:
        \begin{equation*}
            \partial f(x) = \{\xi \in \mathbb{R}^n\colon f(y) \geq f(x) + (\xi, y - x), \forall y \in D(f)\}.
        \end{equation*}
    \end{enumerate}
    \begin{proof}
        1. 注意到$f^*$可以看作是一族凸的, 连续的(从而也是下半连续的)函数之上确界, 而凸且连续的函数族的上确界还是凸且连续的.

        2. 由于$f$是真的, 故可以选取$x_0 \in \mathbb{R}^n$使得$f(x_0) < +\infty$.
        由假设可知, 上图
        \begin{equation*}
            {\rm epi}\ f = \{(x, t) \in \mathbb{R}^n \times \mathbb{R}\colon f(x) \leq t\}
        \end{equation*}
        是非空凸闭集. 取$t_0 < f(x_0)$, 则$(x_0, t_0) \notin {\rm epi}\ f$. 再利用凸集分离定理, 存在$\Phi \in (\mathbb{R}^n \times \mathbb{R})^*$和$\alpha \in \mathbb{R}$, 使得超平面$\{\Phi = \alpha\}$严格分离${\rm epi}\ f$和$(x_0, t_0)$.
        利用$\mathbb{R}^n$的自共轭性, 我们可以找到$(\xi, \lambda) \in \mathbb{R}^n \times \mathbb{R}$, 使得对任意的$(x, t)$, 有 
        \begin{equation*}
            \Phi((x, t)) = (\xi, x)_{\mathbb{R}^n} + \lambda t.
        \end{equation*}
        综上所述, 我们有 
        \begin{equation*}
            (\xi, x)_{\mathbb{R}^n} + \lambda t > \alpha > (\xi, x_0)_{\mathbb{R}^n} + \lambda t_0, \qquad \forall (x, t) \in {\rm epi}\ f.
        \end{equation*}
        特别地, 
        \begin{equation*}
            (\xi, x_0)_{\mathbb{R}^n} + \lambda f(x_0) > \alpha > (\xi, x_0)_{\mathbb{R}^n} + \lambda t_0,
        \end{equation*}
        从而$\lambda > 0$, 并且 
        \begin{equation*}
            \left(-\frac{1}{\lambda}\xi, x\right)_{\mathbb{R}^n} - f(x) < -\frac{\alpha}{\lambda},
        \end{equation*}
        故有 
        \begin{equation*}
            f^*\left(-\frac{\xi}{\lambda}\right) < -\frac{\alpha}{\lambda} < +\infty.
        \end{equation*}
        由此即表明$f \not\equiv +\infty$.
    
        3. 显然.
    
        4. Young不等式可以由Legendre变换的定义直接得出. 此外, 我们有 
        \begin{align*}
            \xi \in \partial f(x) &\Longleftrightarrow (\xi, y - x)_{\mathbb{R}^n} \leq f(y) - f(x), \quad \forall y \in \mathbb{R}^n, \\ 
            &\Longleftrightarrow (\xi, y)_{\mathbb{R}^n} - f(y) \leq (\xi, x)_{\mathbb{R}^n} - f(x), \quad \forall y \in \mathbb{R}^n, \\ 
            &\Longleftrightarrow f^*(\xi) \leq (\xi, x)_{\mathbb{R}^n} - f(x).
        \end{align*}
    \end{proof}
\end{proposition}

\begin{theorem}[Fenchel-Moreau]\footnote{该定理结论对一般的Legendre变换也成立(利用凸集分离定理证明, 即Hahn-Banach定理的几何形式).}
    若$f$是一个真的, 下半连续的凸函数, 则$f^{**} = f$.
\end{theorem}

\begin{corollary}\label{coro2.65}
    对于一个真的, 下半连续的凸函数$f$, 我们有 
    \begin{equation*}
        \xi \in \partial f(x) \ \Longleftrightarrow \ x \in \partial f^*(\xi).
    \end{equation*}
    \begin{proof}
        利用Fenchel-Moreau定理和命题2.63\eqref{prop2.63-4}即可.
    \end{proof}
\end{corollary}

\begin{corollary}
    设$f\colon \mathbb{R}^n \rightarrow (-\infty, +\infty]$是真函数, 则 
    \begin{equation*}
        f^{**} = {\rm conv}\ f = \sup\{\varphi \colon \varphi\ \text{proper and convex}, \varphi (x) \leq f(x), \forall x \in \mathbb{R}^n\}.
    \end{equation*}
    \begin{proof}
        一方面, 设$g \leq f$是凸的, 则它还是真的. 由命题2.63\eqref{prop2.63-3}和Fenchel-Moreau定理, 即得 
        \begin{equation*}
            g = g^{**} \leq f^{**}.
        \end{equation*}
        另一方面, 注意到$f^{**}$本身便是凸的, 由此我们便证得了所需结论.
    \end{proof}
\end{corollary}

\subsubsection{对偶最小作用量原理}

\textbf{Question:} 给定一个凸的Hamilton函数$H \in C^1(\mathbb{R}^N \times \mathbb{R}^N)$, 并满足如下的增长条件:
\begin{equation*}
    0 \leq H(u, \xi) \leq C(|u|^2 + |\xi|^2).
\end{equation*}
求满足下列Hamilton方程组的周期解$(u(t), \xi(t))$:
\begin{equation}\label{59}
    \begin{cases}
        \dot \xi(t) = H_u(u(t), \xi(t)), \\ 
        \dot u(t) = -H_{\xi}(u(t), \xi(t)).
    \end{cases}
\end{equation}

\emph{Solution.} 先求解下列方程的以$2\pi$为周期的解$(v(t), \eta(t))$:
\begin{equation}\label{60}
    \begin{cases}
        \dot \eta(t) = \lambda H_v(v(t), \eta(t)), \\ 
        \dot v(t) = -\lambda H_{\eta}(v(t), \eta(t)).
    \end{cases}
\end{equation}
事实上, 若已求得满足方程\eqref{60}的解$(v(t), \eta(t))$, 令 
\begin{equation*}
    u(t) = v(\lambda^{-1}t), \quad \xi(t) = \eta(\lambda^{-1}t),
\end{equation*}
则$(u(t), \xi(t))$满足\eqref{59}, 它以$2\lambda\pi$为周期($\lambda > 0$).

现将\eqref{60}转化成一个约束极小值问题: 求泛函 
\begin{equation*}
    I(w, \rho) = \int_0^{2\pi}H^*(\dot\rho(t), -\dot w(t)) \,{\rm d}t
\end{equation*}
在约束 
\begin{equation*}
    G(w, \rho) = \int_0^{2\pi}(-\dot\rho(t) \cdot w(t) + \dot w(t) \cdot \rho(t)) \,{\rm d}t = -\pi
\end{equation*}
下的极小值. 这里$H^*$代表$H$的Legendre变换. 事实上, 若$(w_0, \rho_0)$是该约束极值问题的极小点, 则存在$\lambda \in \mathbb{R}$, 使得 
\begin{equation*}
    \begin{cases}
        \frac{{\rm d}}{{\rm d}t}(H_{\rho}^*(\dot\rho_0(t), -\dot w_0(t)) - \frac{\lambda}{2} w_0(t)) = \frac{\lambda}{2} \dot w_0(t), \\ 
        \frac{{\rm d}}{{\rm d}t}(-H_w^*(\dot\rho_0(t), -\dot w_0(t)) + \frac{\lambda}{2}\rho_0(t)) = -\frac{\lambda}{2} \dot\rho_0(t).
    \end{cases}
\end{equation*}
再利用\ref{coro2.65}, 上式等价于 
\begin{equation*}
    \begin{cases}
        \dot\rho_0(t) = H_w(\lambda w_0(t), \lambda\rho_0(t)), \\
        \dot w_0(t) = -H_{\rho}(\lambda w_0(t), \lambda\rho_0(t)).
    \end{cases}
\end{equation*}
令$\eta_0 = \lambda\rho_0, v_0 = \lambda w_0$, 容易验证$(\eta_0, v_0)$满足\eqref{60}.

以下利用直接方法来验证极小点$(w_0, \rho_0)$的存在性. 显然, $I$是连续的, 从而也是弱序列下半连续的.
其次, 由$H$的增长性条件可知 
\begin{equation*}
    H^*(w, \rho) \geq \frac{1}{C}(|w|^2 + |\rho|^2),
\end{equation*}
从而有 
\begin{equation*}
    I(w, \rho) \geq \frac{1}{C}\int_0^{2\pi}(|\dot w|^2 + |\dot \rho|^2) \,{\rm d}t.
\end{equation*}
这表明$I$是下方有界, 且是强制的. 此时定义域$M = G^{-1}(-\pi)$上的范数规定为 
\begin{equation*}
    \Vert (w, \rho)\Vert = \left(\int_0^{2\pi}(|\dot w|^2 + |\dot \rho|^2) \,{\rm d}t\right)^{1/2}.
\end{equation*}
最后, 设$\{(w_i, \rho_i)\} \subseteq M$在$H_{{\rm per}}^1(0, 2\pi)$上弱收敛到$(w^*, \rho^*)$.
由紧嵌入$H^1 \hookrightarrow L^2$可知, $w_j$和$\rho_j$在$L^2(0, 2\pi)$中分别收敛到$w^*$和$\rho^*$.
从而有 
\begin{align*}
    -\pi = \lim\limits_{j \rightarrow \infty}G(w_j, \rho_j) &= \lim\limits_{j \rightarrow \infty}\int_0^{2\pi}(-\dot\rho_j \cdot w_j + \dot w_j \cdot \rho_j) \,{\rm d}t \\ 
    &= \int_0^{2\pi}(-\dot\rho^* \cdot w^* + \dot w^* \cdot \rho^*) \,{\rm d}t = G(w^*, \rho^*).
\end{align*}
这表明$(w^*, \rho^*) \in M$, 故$M$是弱序列闭的. 综上所述, 我们利用直接方法验证了极小点$(w_0, \rho_0)$的存在性.

最后, 为使得到的极小点$(w_0, \rho_0)$是非平凡的, 我们还需说明$\lambda > 0$.
使用记号 
\begin{equation*}
    \langle (w, \rho), (u, \xi)\rangle = \int_0^{2\pi}(w \cdot u + \rho \cdot \xi) \,{\rm d}t.
\end{equation*}
在前述推导中我们得到了等式 
\begin{equation*}
    \nabla H^*(\dot \rho_0, -\dot w_0) = \lambda(w_0, \rho_0).
\end{equation*}
从而 
\begin{equation*}
    \langle \nabla H^*(\dot \rho_0, -\dot w_0), (\dot\rho_0, -\dot w_0)\rangle = -\lambda G(w_0, \rho_0) = \lambda\pi.
\end{equation*}
利用$H$的增长条件, 易得$\nabla H(0, 0) = (0, 0)$, 故 
\begin{equation*}
    H^*(0, 0) = -H(0, 0) = 0.
\end{equation*}
由于$H$是凸的, 从而有 
\begin{equation*}
    H^*(0, 0) - H^*(\dot\rho_0, -\dot w_0) \geq -\langle \nabla H^*(\dot\rho_0, -\dot w_0), (\dot \rho_0, -\dot w_0)\rangle,
\end{equation*}
即 
\begin{equation*}
    \lambda\pi = \langle \nabla H^*(\dot\rho_0, -\dot w_0), (\dot \rho_0, -\dot w_0)\rangle \geq H^*(\dot\rho_0, -\dot w_0) \geq 0.
\end{equation*}
这表明$\lambda \geq 0$. 以下验证$\lambda \neq 0$. 若$\lambda = 0$, 那么必有 
\begin{equation*}
    \nabla H^*(\dot\rho_0, -\dot w_0) = (0, 0).
\end{equation*}
从而 
\begin{equation*}
    \begin{cases}
        \dot\rho_0 = H_w(0, 0), \\ 
        \dot w_0 = -H_{\rho}(0, 0).
    \end{cases}
\end{equation*}
再利用$H$的增长条件, 即得$(\dot\rho_0, \dot w_0) = (0, 0)$, 而这与约束条件矛盾!

\end{document}
